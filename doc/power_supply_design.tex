\documentclass[10pt,twoside]{article}

% Math symbols
\usepackage{amsmath}
\usepackage{amssymb}

% Headers/Footers
\usepackage{fancyhdr}

% Colors
\usepackage[usenames,dvipsnames]{color}

% Importing and manipulating graphics
\usepackage{graphicx}
\usepackage{subfig}
\usepackage{lscape}

% Misc packages
\usepackage{verbatim}
\usepackage{dcolumn}
\usepackage{ifpdf}
\usepackage{enumerate}

% PDF Bookmarks and hyperref stuff
\usepackage[
  bookmarks=true,
  bookmarksnumbered=true,
  colorlinks=true,
  filecolor=blue,
  linkcolor=blue,
  urlcolor=blue,
  hyperfootnotes=true
  citecolor=blue
]{hyperref}

% Improved hyperlinking to figures
% (include after hyperref)
\usepackage[all]{hypcap}

% Improved citation handling
% (include after the hyperref stuff)
\usepackage{cite}

% Pretty-print code
\usepackage{listings}

% -----------------------------------------------------------------
% Hyper-references
% -----------------------------------------------------------------
%
% This page has lots of good advice on hyper-references, including
% the use of the hypcap package and \phantomsection for generating
% labels to text.
%
% http://en.wikibooks.org/wiki/LaTeX/Labels_and_Cross-referencing
%
% -----------------------------------------------------------------
% Setup the margins
% -----------------------------------------------------------------
% Footer Template

% Set left margin - The default is 1 inch, so the following
% command sets a 1.25-inch left margin.
\setlength{\oddsidemargin}{0.25in}
\setlength{\evensidemargin}{0.25in}

% Set width of the text - What is left will be the right
% margin. In this case, right margin is
% 8.5in - 1.25in - 6in = 1.25in.
\setlength{\textwidth}{6in}

% Set top margin - The default is 1 inch, so the following
% command sets a 0.75-inch top margin.
%\setlength{\topmargin}{-0.25in}

% Set height of the header
\setlength{\headheight}{0.3in}

% Set vertical distance between the header and the text
\setlength{\headsep}{0.2in}

% Set height of the text
\setlength{\textheight}{8.5in}

% Set vertical distance between the text and the
% bottom of footer
\setlength{\footskip}{0.4in}

% -----------------------------------------------------------------
% Allow floats to take up more space on a page.
% -----------------------------------------------------------------

% see page 142 of the Companion for this stuff and the
% documentation for the fancyhdr package
\renewcommand{\textfraction}{0.05}
\renewcommand{\topfraction}{0.95}
\renewcommand{\bottomfraction}{0.95}
% dont make this too small
\renewcommand{\floatpagefraction}{0.35}
\setcounter{totalnumber}{5}

% -----------------------------------------------------------------
% Abbreviated symbols
% -----------------------------------------------------------------
\newcommand{\sinc}{\ensuremath{\,\text{sinc}}}
\newcommand{\rect}{\ensuremath{\,\text{rect}}}

% =================================================================
% The document starts here
% =================================================================
%
\begin{document}
\title{Power Supply Design Examples}
\author{D. W. Hawkins (dwh@ovro.caltech.edu)}
\date{\today}
%\date{June 13, 2012}
\maketitle

% No header/footer on the first page
\thispagestyle{empty}

\tableofcontents

% start the intro on an odd page
\cleardoublepage
%\clearpage

% Set up the header/footer
\pagestyle{fancy}
\lhead{Power Supply Design Examples}
\chead{}
\rhead{\today}
\lfoot{}
\cfoot{}
\rfoot{\thepage}
\renewcommand{\headrulewidth}{0.4pt}
\renewcommand{\footrulewidth}{0.4pt}

% Set the listings package language to Tcl
\lstset{language=Tcl}

% =================================================================
\section{Introduction}
% =================================================================

This document contains power supply design examples created
and analyzed using LTspice and MATLAB.

% =================================================================
\section{Example Designs}
% =================================================================

The following power supply design examples follows the same
general flow;
%
\begin{enumerate}
\item Specify the requirements for the power supply.
\item Follow the data sheet design sequence.
\item Create LTspice simulations for the transient and
frequency (Bode) responses.
\item Review the Bode and transient responses.
\item Optimize the control loop using the measured Bode response.
\item Review and discuss the design.
\end{enumerate}
%
These example designs provide significantly more insight into
the design of the power supplies than the Linear Technology
data sheets or application notes.

% -----------------------------------------------------------------
% Include the example designs
% -----------------------------------------------------------------
%
% Each example design contains much the same sequence, so having
% them in a separate file makes it easier to create a new example
% using an earlier example as a template.
%
\clearpage
% =================================================================
\subsection{LTC1735: 12V to 3.3V@6A}
% =================================================================

Linear Technology power supply controller data sheets briefly
describe how to compensate the control loop, and then refer the
reader to Application Note AN76~\cite{Linear_AN76_1999} for more
details. AN76 discusses controller compensation with respect to 
the LTC1735, LTC1736, and LTC1628 controllers. The 
LTC1735~\cite{Linear_LTC1735_1998} is used as the 
controller in the following example design, so that the
recommendations in AN76 can be reviewed with respect to an
LTspice based design.

% -----------------------------------------------------------------
\subsubsection{Component Selection}
% -----------------------------------------------------------------

Power supply requirements;
%
\begin{itemize}
\item Input supply voltage, $V_{\rm IN} = 12$V
\item Output supply voltage, $V_{\rm OUT} = 3.3$V
\item Output supply current, $I_{\rm OUT} = 6$A
\item Output supply voltage ripple, $\Delta V_{\rm OUT} = \pm 165$mV ($\pm5$\%)
\item Controller switching frequency, $f_{\rm SW} = 500$kHz
\end{itemize}
%
The controller components are selected based on the data
sheet procedure (see p11~\cite{Linear_LTC1735_1998});
%
\begin{enumerate}
\item Select the output current-sense resistor.

\begin{equation}
R_{\rm SENSE} = \frac{50\text{mV}}{I_{\rm MAX}} =
\frac{50\text{mV}}{7\text{A}} = 7.14\text{m}\Omega \approx  7\text{m}\Omega 
\end{equation}
%
where $I_{\rm MAX} = I_{\rm OUT} + \Delta I_{\rm OUT}/2 \approx 7$A is calculated
shortly.
%
\item Select $C_{\rm OSC}$ to select the controller operating
frequency.

The LTspice model was used to determine the capacitor value that
resulted in a switching frequency of 500kHz, i.e., 
$C_{\rm OSC} = 30.1$pF.

\item Inductor selection.

The {\em nominal} inductor value is determined based on a ripple current of
30\% of the maximum output current, i.e., 
$\Delta I_{\rm OUT} = 0.3I_{\rm OUT} = 2$A. The inductor value 
is then
%
\begin{equation}
L = \frac{V_{\rm OUT}}{f_{\rm SW}\Delta I_{\rm OUT}}\left(
1 - \frac{V_{\rm OUT}}{V_{\rm IN}}\right) =
\frac{3.3}{500\text{k}\times2}\left(
1 - \frac{3.3}{12}\right) = 2.3925\mu\text{H} \approx 2.4\mu\text{H}
\end{equation}
%
A review of Vishay/Dale IHLP series inductors showed there were
no 2.4$\mu$H inductors that would support 6A of load current.
However, there were 2.2$\mu$H and 3.3$\mu$H parts,
eg., see the IHLP-2525CZ-01 and IHLP-3232DZ-01 series devices.
For $L = 2.2\mu$H, $\Delta I_{\rm OUT} = 2.2$A and 
$I_{\rm MAX} = 7.1$A, while for $L = 3.3\mu$H, 
$\Delta I_{\rm OUT} = 1.5$A and $I_{\rm MAX} = 6.75$A.

Although the larger inductor generates a smaller ripple current,
it increases the amount of output capacitance required to meet
the output voltage ripple requirement during a load transient,
so the smaller $L = 2.2\mu$H inductor is selected.
The LTC1735 does not support inductor DCR sensing, so the inductor
resistance just has to be low enough to keep the inductor
temperature increase under full load within reason.
A 6A load will generate a temperature increase of under $30^\circ$C
in either of the following Vishay/Dale devices;
IHLP-2525CZ-01 2.2$\mu$H $R_{\rm DCR(max)} = 20\text{m}\Omega$ or
IHLP-3232DZ-01 2.2$\mu$H $R_{\rm DCR(max)} = 17.7\text{m}\Omega$.

\item Power MOSFET and D1 selection.

The LTspice test jig for the LTC1735 used Vishay/Silconix
Si4410DY MOSFETs for both the top and bottom MOSFETs, and
used a 1N5818 diode over the synchronous (bottom) MOSFET.
These components were selected for use in this example.

Based on the LTC1735 data sheet equations, the power dissipated
in the synchronous MOSFET will be
%
\begin{equation}
P_{\rm SYNC} \approx \left(1 - \frac{V_{\rm OUT}}{V_{\rm IN}}\right)
I_{\rm MAX}^2R_{\rm DS(ON)} = 0.725\times7.1^2\times20\text{m} = 0.73\text{W}
\end{equation}
%
and with $\theta_{\rm JA} = 50^\circ$C/W, the junction temperature
rise will be $37^\circ$C. The power dissipated in the main (top) MOSFET
will be
%
\begin{equation}
\begin{split}
P_{\rm MAIN} &\approx \frac{V_{\rm OUT}}{V_{\rm IN}}I_{\rm MAX}^2R_{\rm DS(ON)} +
kV_{\rm IN}I_{\rm MAX}C_{\rm RSS}f_{\rm SW}\\
&= 0.275\times7.1^2\times20\text{m} + 
1.7\times12^2\times7.1\times ??? \times 500\text{k}\\
&= ???
\end{split}
\end{equation}
%
resulting in a junction temperature rise of ???

Page 26 has an example with an Si4412ADY and uses Css = 100pF. See if
I can find that data sheet. They then use an Si4410DY for the
synchronous MOSFET ... ah, so perhaps change the LTSpice simulation?

LTspice will be used to review these power dissipation estimates.

\item Input capacitance selection.

TODO: how do I calculate the value and ESR? Is it just based on
RMS handling? p21 recommends 20 to 40$\mu$F with an ESR of 10 
to 10m$\Omega$.

\item Output capacitance selection.

The {\em nominal} value of the output capacitance is determined by 
the amount required to absorb the energy in the inductor during a
load step from maximum current to minimum current, i.e.,
%
\begin{equation}
\begin{split}
C_{\rm OUT} &> \frac{LI_{\rm STEP}^2}
{(V_{\rm OUT} + \Delta V_{\rm OUT})^2 - V_{\rm OUT}^2} 
\approx \frac{LI_{\rm STEP}^2}{2V_{\rm OUT}\Delta V_{\rm OUT}}\\
&= \frac{2.2\mu\times 6^2}{2\times3.3\times0.1}\\
&= 120\mu\text{F}
\end{split}
\end{equation}
%
where $I_{\rm STEP}=6$A and $\Delta V_{\rm OUT} = 100$mV was
used, since there will be additional ripple voltage due to 
the output capacitance ESR.

The {\em nominal} value of the output capacitance ESR is determined by
the allowable output voltage dip during a load step from minimum current
to maximum current, i.e.,
%
\begin{equation}
R_{\rm ESR} = \frac{\Delta V_{\rm OUT}}{I_{\rm STEP}} = 
\frac{0.1}{6} \approx 17\text{m}\Omega
\end{equation}
%

The output capacitance requirements can be met using a single
Sanyo POSCAP TPE-series capacitor, eg., one of
150$\mu$F 6TPE150MI (6.3V, 18m$\Omega$, 2.8A$_{\rm RMS}$),
220$\mu$F 6TPE220MI (6.3V, 18m$\Omega$, 2.8A$_{\rm RMS}$), or
330$\mu$F 6TPE330MFL (6.3V, 15m$\Omega$, 3.1A$_{\rm RMS}$).

To minimize the location of the power-stage LC resonance,
the smallest value will be selected, i.e., 150$\mu$F with 
an ESR of 18m$\Omega$. The output voltage dip for a 6A output
current increase will be around 108mV, the output voltage peaking for
a 6A output current decrease will be around 67mV, and the ripple
voltage will be approximately 38mV.

\item Output voltage.

The output voltage is determined by the output feedback resistors
and the controller reference voltage, 
%
\begin{equation}
V_{\rm OUT} = V_{\rm REF}\left(1+\frac{R_{\rm a}}{R_{\rm b}}\right)
\end{equation}
%
where $V_{\rm REF} = 0.8$V, $R_{\rm a}$ is the resistor from the
output voltage to the controller $V_{\rm OSENSE}$ pin, and 
$R_{\rm b}$ is the resistor from the controller $V_{\rm OSENSE}$ 
pin to ground.

The nearest pair of 1-percent resistor values for a desired
output voltage of 3.3V are $R_{\rm a} = 3.57\text{k}\Omega$ and 
$R_{\rm b} = 1.15\text{k}\Omega$, resulting in an output
voltage of $V_{\rm OUT} = 3.28$V.

\item Soft-start capacitance.

A soft-start capacitance of $C_{\rm SS} = 800$pF was selected based
on an analysis of the LTspice transient simulation.
The simulation generates an output current load step once the 
controller has completed the soft-start sequence and the output
voltage has settled. The voltage on the soft-start capacitor 
sets the output current-limit for a short time after the
output voltage has settled. The output load step {\em should not}
be applied until after the soft-start capacitor voltage has exceeded
3.0V, otherwise the controller will enter current-limit, and
the voltage on the compensation pin will not reflect the
correct control loop dynamics.
This effect is shown shortly during the transient discussion.

\item Compensation components.

Page 21 of the LTC1735 data sheet indicates to use the compensation
components shown in Figure 1 as the starting point for the 
compensation design~\cite{Linear_LTC1735_1998}, i.e., 
$R_1  = 33\text{k}\Omega$, $C_1 = 330$pF, and
$C_2 = 100$pF. These components create a compensation
network with a zero frequency of
%
\begin{equation}
f_{\rm Z} = \frac{1}{2\pi R_1 C_1} = 14.6\text{kHz}
\end{equation}
%
and a pole frequency of
%
\begin{equation}
f_{\rm P} = \frac{1}{2\pi R_1 C_1C_2/(C_1+C_2)}
= 62.8\text{kHz}
\end{equation}
\end{enumerate}

\clearpage
% -----------------------------------------------------------------
\subsubsection{Transient Response Analysis}
% -----------------------------------------------------------------

Figure~\ref{fig:LTC1735_ex1_transient_circuit} shows an LTspice
circuit for analyzing the transient response of the
LTC1735 12V to 3.3V@6A power supply design.
Figure~\ref{fig:LTC1735_ex1_transient_response} shows the transient
response of several controller waveforms. The output voltage starts
to ramp up at $t = 0.93$ms, which is when the oscillator starts (the 
oscillator can be viewed by adding \verb+V(osc)+ to the plot).
The voltage on the soft-start capacitor sets the power-on 
current-limit, which the compensation voltage (error amplifier 
output), \verb+V(comp)+, then tracks;
in Figure~\ref{fig:LTC1735_ex1_transient_response}(a) \verb+V(comp)+
follows \verb+V(ss)-0.5+.

Figure~\ref{fig:LTC1735_ex1_transient_response} shows the transient
response for an output current load step from 1A to 6A (a 5A load step),
applied at $t = 2.0$ms and removed at $t = 2.4$ms. 
The load step is not applied until the soft-start voltage has 
risen above 3V, as otherwise the soft-start function
will limit \verb+V(comp)+  to \verb+V(ss)-0.5+, preventing the 
response at \verb+V(comp)+ being used to investigate the control-loop
dynamics. This relative timing of the soft-start and load step is only 
a concern in simulation. A hardware test of the controller would
apply the load step long after soft-start completes.

Figure~\ref{fig:LTC1735_ex1_transient_response}(b) shows a zoomed view of
the controller response to an output current load step. The compensation
response, \verb+V(comp)+, has no overshoot or ringing, indicating
the closed-loop response is stable with good phase margin.
The inductor current ripple of $\Delta I_{\rm OUT} = 2.1$A
can be seen in the inductor current waveform, \verb+I(L)+,
in the bottom trace in Figure~\ref{fig:LTC1735_ex1_transient_response}(b).
The AC component of the inductor current travels through the output
capacitance, and generates the output ripple voltage over the output
capacitance ESR.  This ripple voltage can be seen in \verb+V(out)+, in
the top trace in Figure~\ref{fig:LTC1735_ex1_transient_response}(b). 
The peak-to-peak output ripple voltage matches the expected value of 
$2.1\text{A}\times18\text{m}\Omega=38$mV.
The output voltage has an average value of 3.283V, as expected due to 
the use of 1-percent output resistors to set the output voltage. 
The output voltage dip and peaking during the application and removal
of the output current step is about -140mV and +120mV respectively.
These values are within the output voltage ripple specification,
however, due to the offset of the output voltage, the voltage dip
during the load current increase marginally violates the specification.
This issue can be rectified by using $R_{\rm a} = 10.7\text{k}\Omega$
and $R_{\rm b} = 3.4\text{k}\Omega$ for $V_{\rm OUT} = 3.318$V,
or  $R_{\rm a} = 46.4\text{k}\Omega$
and $R_{\rm b} = 14.7\text{k}\Omega$ for $V_{\rm OUT} = 3.325$V.
With this change, the power supply meets the design requirements.

The transient response of the output voltage changes depending on
the control-loop bandwidth. If the control-loop has high bandwidth,
then the output voltage dip would be dominated by the output load
step generating a voltage drop over the output capacitor ESR. 
The magnitude of the voltage dip for this example would have been
$5\text{A}\times18\text{m}\Omega=90$mV.
Since the output voltage dip of 140mV in 
Figure~\ref{fig:LTC1735_ex1_transient_response}(b) is larger than
the ESR voltage dip, the dip magnitude is being dominated by
the controller response time.
Although this dip could be reduced, since the output voltage
ripple is within the specifications, it can be accepted.


% -----------------------------------------------------------------
% LTspice circuit
% -----------------------------------------------------------------
%
\begin{landscape}
\setlength{\unitlength}{1mm}
\begin{figure}[p]
  \begin{center}
    \includegraphics[width=210mm]
    {figures/LTC1735_ex1_transient_circuit.pdf}
  \end{center}
  \caption{LTC1735 12V to 3.3V@6A LTspice transient response analysis circuit.}
  \label{fig:LTC1735_ex1_transient_circuit}
\end{figure}
\end{landscape}

% -----------------------------------------------------------------
% Transient response waveforms
% -----------------------------------------------------------------
%
\setlength{\unitlength}{1mm}
\begin{figure}[p]
  \begin{picture}(155,205)(0,0)
    \put(10,105){
    \includegraphics[width=0.87\textwidth]
    {figures/LTC1735_ex1_transient_response_a.pdf}}
    \put(10,0){
    \includegraphics[width=0.87\textwidth]
    {figures/LTC1735_ex1_transient_response_b.pdf}}
    \put(77,105){(a)}
    \put(77, 0){(b)}
  \end{picture}
  \caption{LTC1735 12V to 3.3V@6A supply LTspice transient response;
  (a) from $t=0$, and (b) zoomed view from $t=1.7$ms to 2.7ms.}
  \label{fig:LTC1735_ex1_transient_response}
\end{figure}

\clearpage
% -----------------------------------------------------------------
\subsubsection{Frequency Response Analysis}
% -----------------------------------------------------------------

This section uses an LTspice circuit to measure the power-supply
open-loop, control-to-output, compensation, and power-stage
frequency response. The frequency responses are then analyzed to
determine whether any circuit adjustments are required.

Figure~\ref{fig:LTC1735_ex1_bode_circuit} shows an LTspice
circuit for analyzing the Bode response of the
LTC1735 12V to 3.3V@6A power supply design. The {\em frequency}
response of the circuit is analyzed using multiple transient
({\em time}) analysis runs. The frequency response analysis
must be performed using multiple transient analysis runs, as
switched-mode power supplies are non-linear devices, and 
there is no linearized small-signal AC model (which is the model
needed for an LTspice \verb+.ac+ analysis). The frequency response
of a {\em physical} power supply can be measured using a
{\em frequency response analyzer} (FRA). The circuit in
Figure~\ref{fig:LTC1735_ex1_bode_circuit} uses the same
technique as an FRA; the feedback loop is broken at a 
low-impedance point, a small amplitude sine wave is injected 
into the loop, and the response at various locations around the
control loop is measured. The LTspice \verb+.measure+ statements in 
Figure~\ref{fig:LTC1735_ex1_bode_circuit} calculate the
gain and phase at each measurement frequency in the \verb+.step+
list, for the locations \verb+A+, \verb+B+, \verb+C+, and \verb+D+
in the circuit. The measurements at those locations are then 
converted into open-loop, control-to-output, compensation, and
LC response measurements.

Figure~\ref{fig:LTC1735_ex1_bode_response} shows the Bode response
for the circuit in Figure~\ref{fig:LTC1735_ex1_bode_circuit}.
The Bode response shows that the loop gain has a
cross-over frequency of 46kHz with 60-degrees of phase margin.
This is consistent with the good transient response observed in
Figure~\ref{fig:LTC1735_ex1_transient_response}. The Bode response
allows you to investigate the main control loop components;
the control-to-output (including the current-feedback loop) 
and the compensation responses.

The current-feedback loop converts the resonant LC response of 
the power-stage into a control-to-output response that is more
like a single-pole response (at least at low-frequencies). 
The control-to-output phase is a good indicator of how well the 
current-loop has been designed.
%
Figure~\ref{fig:LTC1735_ex1_bode_response}(b) shows how the close to
-180-degree phase in the LC response is reduced to under -90-degrees
in the control-to-output loop. This indicates a reasonable 
closed-loop current-feedback response.

The magnitude and phase of the control-to-output loop can be adjusted
using the current-sense resistor. A slightly smaller resistor will
cause slightly less signal to be fed back into the current-loop,
causing the control-to-output gain to increase and the control-to-output
phase to decrease. The increase in control-to-output gain can be
used to increase the cross-over frequency, however, due to the loss
of control-to-output phase, the compensation phase must be adjusted
to provide sufficient phase-margin. This adjustment in compensation
phase might not be possible due to the desire to keep the compensation
gain at the switching frequency at or below 0dB. Hence, there is a
tradeoff in the design of the current-loop and compensation
responses.

The compensation response in Figure~\ref{fig:LTC1735_ex1_bode_response}
meets the typical requirements for a compensation network; it
provides sufficient mid-band gain and phase to produce an
open-loop response with a high cross-over frequency with sufficient
phase-margin, and the gain at the switching frequency is no more than
0dB (this results in the same amount of ripple voltage on the error
amplifier output as on the power supply output).
Table~\ref{fig:LTC1735_ex1_compensation} shows the results of 
{\em designing} a compensation network using a MATLAB script.
The MATLAB script imports the control-to-output gain measured 
using LTspice, and then allows the user to adjust the compensation
response by specifying the desired cross-over frequency and
zero and pole locations. The script then calculates the nearest
1-percent resistor value and standard capacitor values.
Table~\ref{fig:LTC1735_ex1_compensation} shows the compensation
parameters for the circuit in 
Figure~\ref{fig:LTC1735_ex1_bode_response} and an alternative
set of parameters (that result in a very similar Bode response).
A transient response using the alternative parameters appeared
identical to that shown in Figure~\ref{fig:LTC1735_ex1_transient_response}.
The fact that the compensation could not be improved significantly
is a reflection that the original compensation was reasonable.

\clearpage
% -----------------------------------------------------------------
% LTspice circuit
% -----------------------------------------------------------------
%
\begin{landscape}
\setlength{\unitlength}{1mm}
\begin{figure}[p]
  \begin{center}
    \includegraphics[width=200mm]
    {figures/LTC1735_ex1_bode_circuit.pdf}
  \end{center}
  \caption{LTC1735 12V to 3.3V@6A LTspice Bode response analysis circuit.}
  \label{fig:LTC1735_ex1_bode_circuit}
\end{figure}
\end{landscape}

\clearpage
% -----------------------------------------------------------------
% Bode response
% -----------------------------------------------------------------
%
\setlength{\unitlength}{1mm}
\begin{figure}[p]
  \begin{picture}(155,190)(0,0)
    \put(15,100){
    \includegraphics[width=0.75\textwidth]
    {figures/LTC1735_ex1_bode_response_mag.pdf}}
    \put(15,4){
    \includegraphics[width=0.75\textwidth]
    {figures/LTC1735_ex1_bode_response_phase.pdf}}
    \put(75,  97){(a)}
    \put(75,   0){(b)}
  \end{picture}
  \caption{LTC1735 12V to 3.3V@6A supply Bode response;
  (a) magnitude, and (b) phase. The LC and compensation
  {\em calculated} responses are shown by the black and red
  solid lines, while the LTspice {\em simulated} responses are
  shown by the black and red crosses. The 
  control-to-output and open-loop LTspice {\em simulated} responses
  are shown using both solid lines and crosses.
  The open-loop gain has a cross-over frequency of 46kHz with
  60-degrees of phase margin.}
  \label{fig:LTC1735_ex1_bode_response}
\end{figure}

\clearpage
% -----------------------------------------------------------------
% Compensator design
% -----------------------------------------------------------------
%
\begin{table}
\caption{LTC1735 12V to 3.3V@6A supply compensation design.}
\label{fig:LTC1735_ex1_compensation}
\begin{center}
\begin{tabular}{|l|c|c|}
\hline
\rule{0cm}{4mm}Component or Parameter & \multicolumn{2}{c|}{Compensation Design Number}\\
\cline{2-3}
\rule{0cm}{4mm}     & \#1 & \#2\\
\hline
\hline
\multicolumn{3}{|l|}{\bf Compensation response}\\
\hline
& \hspace {20mm} &  \hspace {20mm} \\
$R_1$       & 33.0k$\Omega$ & 40.2k$\Omega$ \\
$C_1$       &  330pF        &  270pF        \\
$C_2$       &  100pF        &  100pF        \\
&&\\
$f_{\rm Z}$ & 14.6kHz       & 14.7kHz       \\
$f_{\rm P}$ & 62.8kHz       & 54.3kHz       \\
&&\\
Compensation gain at $f_{\rm SW}$ & 0.01dB & 0.02dB\\
&&\\
\hline
\multicolumn{3}{|l|}{\bf Open-loop response}\\
\hline
&&\\
Cross-over frequency    & 47.0kHz      & 49.8kHz      \\ 
Cross-over phase-margin & 61.0$^\circ$ & 56.8$^\circ$ \\
&&\\
\hline
\end{tabular}
\end{center}
\end{table}

\clearpage
% -----------------------------------------------------------------
\subsubsection{Review and Discussion}
% -----------------------------------------------------------------

\noindent{\bf TODO}:
\begin{itemize}
\item Review the power dissipation in the MOSFETs.
\item Can I get an efficiency report?
\item Add a comment about output loads based on current sinks rather
than resistive loads, per the comments in~\cite{Linear_DC247_1999}
\item Find ESR of ceramic caps and add a couple on the output.
\end{itemize}


\clearpage
% =================================================================
\subsection{LTC3851A: 12V to 1.5V@15A}
% =================================================================

The Linear Technology web page for the LTC1735 indicates that
the device is not recommended for new designs, and recommends
using the LTC3851 controller, which in turn recommends using
the LTC3851A~\cite{Linear_LTC3851A_2010}. The power supply 
design in this section is based on the reference design that
can be downloaded from the
\href{http://www.linear.com/product/LTC3851A}{LTC3851A web page}.

% -----------------------------------------------------------------
\subsubsection{Component Selection}
% -----------------------------------------------------------------

Power supply requirements;
%
\begin{itemize}
\item Input supply voltage, $V_{\rm IN} = 12$V
\item Output supply voltage, $V_{\rm OUT} = 1.5$V
\item Output supply current, $I_{\rm OUT} = 15$A
\item Output supply voltage ripple, $\Delta V_{\rm OUT} = \pm 75$mV ($\pm5$\%)
\item Controller switching frequency, $f_{\rm SW} = 500$kHz
\end{itemize}
%
The controller components are selected based on the data
sheet procedure (see p11~\cite{Linear_LTC3851A_2010});
%
\begin{enumerate}
\item Select the output current-sense resistor.

\begin{equation}
R_{\rm SENSE} = 0.8\cdot\frac{V_{\rm MAX}}{(I_{\rm OUT} + \Delta I_{\rm OUT}/2)} =
0.8\cdot\frac{53\text{mV}}{17.8\text{A}} = 2.38\text{m}\Omega \approx  2\text{m}\Omega 
\end{equation}
%
where $I_{\rm MAX} = I_{\rm OUT} + \Delta I_{\rm OUT}/2 \approx 17.8$A is
calculated shortly.
%
\item Select $R_{\rm FREQ}$ to select the controller operating
frequency.

$R_{\rm FREQ} = 60\text{k}\Omega$ selects a switching frequency of
500kHz (p4~\cite{Linear_LTC3851A_2010}). The LTspice model was confirmed
to generate this switching frequency.

\item Inductor selection.

The {\em nominal} inductor value is determined based on a ripple current of
30\% of the maximum output current, i.e., 
$\Delta I_{\rm OUT} = 0.3I_{\rm OUT} = 5$A. The inductor value 
is then
%
\begin{equation}
L = \frac{V_{\rm OUT}}{f_{\rm SW}\Delta I_{\rm OUT}}\left(
1 - \frac{V_{\rm OUT}}{V_{\rm IN}}\right) =
\frac{1.5}{500\text{k}\times5}\left(
1 - \frac{1.5}{12}\right) = 0.525\mu\text{H} \approx 0.5\mu\text{H}
\end{equation}
%
A review of Vishay/Dale IHLP series inductors showed that the nearest
value is the IHLP-3232DZ-01 $0.47\mu\text{H}$ 
($R_{\rm DCR(MAX)} = 2.62\text{m}\Omega$).
For $L = 0.47\mu$H, $\Delta I_{\rm OUT} = 5.8$A and 
$I_{\rm MAX} = 17.8$A. The IHLP-3232DZ-01 inductor can easily
support a peak load current of 17.8A, resulting in a temperature 
rise of about $20^\circ$C.

The LTC3851A supports current-sensing via either a current-sense
resistor, or via inductor DCR sensing. The DCR of the $0.47\mu\text{H}$
is slightly higher than the target sense resistance of $2\text{m}\Omega$,
so the DCR sense network needs to attenuate the current-sense
voltage by about 0.8. The use of a sense resistor versus the inductor
DCR will be compared during power (efficiency) analysis.

\item Power MOSFET and D1 selection.

The LTspice example downloaded from the LTC3851A web page uses a
Renesas RJK0305DP for the main (top) MOSFET 
($R_{\rm DS(ON)} = 6.7\text{m}\Omega$), and a Renesas RJK0301DP
for the synchronous (bottom) MOSFET ($R_{\rm DS(ON)} = 2.3\text{m}\Omega$).
These components are also used in this example.

Based on the LTC3851A data sheet equations, the power dissipated
in the synchronous MOSFET will be
%
\begin{equation}
P_{\rm SYNC} \approx \left(1 - \frac{V_{\rm OUT}}{V_{\rm IN}}\right)
I_{\rm MAX}^2R_{\rm DS(ON)} = 0.875\times17.8^2\times2.3\text{m} = 0.64\text{W}
\end{equation}
%
and with a channel-to-case thermal resistance of $\theta_{\rm ch-C} = 1.93^\circ$C/W,
the channel temperature rise will be $1.2^\circ$C. The power dissipated in the
main (top) MOSFET will be
%
\begin{equation}
\begin{split}
P_{\rm MAIN} &\approx \frac{V_{\rm OUT}}{V_{\rm IN}}I_{\rm MAX}^2R_{\rm DS(ON)} +
\frac{V_{\rm IN}^2I_{\rm MAX}R_{\rm DR}C_{\rm MILLER}}{2}\cdot
\left[ \frac{1}{V_{\rm INTVCC} - V_{\rm TH(MIN)}} + \frac{1}{V_{\rm TH(MIN)}}\right]\\
&= 0.125\times17.8^2\times6.7\text{m} +  ???\\
&= ???
\end{split}
\end{equation}
%
resulting in a junction temperature rise of ???

{\bf TODO}: fill in these numbers.

LTspice will be used to review these power dissipation estimates.

\item Input capacitance selection.

TODO: how do I calculate the value and ESR? Is it just based on
RMS handling? LTC3851A data sheet p17 has comments on ESR, but
nothing on the amount of capacitance.

\item Output capacitance selection.

The {\em nominal} value of the output capacitance is determined by 
the amount required to absorb the energy in the inductor during a
load step from maximum current to minimum current, i.e.,
%
\begin{equation}
\begin{split}
C_{\rm OUT} &> \frac{LI_{\rm STEP}^2}
{(V_{\rm OUT} + \Delta V_{\rm OUT})^2 - V_{\rm OUT}^2} 
\approx \frac{LI_{\rm STEP}^2}{2V_{\rm OUT}\Delta V_{\rm OUT}}\\
&= \frac{0.47\mu\times 15^2}{2\times1.5\times60\text{m}}\\
&= 588\mu\text{F}
\end{split}
\end{equation}
%
where $I_{\rm STEP}=15$A and $\Delta V_{\rm OUT} = 60$mV was
used, since there will be additional ripple voltage due to 
the output capacitance ESR.

The {\em nominal} value of the output capacitance ESR is determined by
the allowable output voltage dip during a load step from minimum current
to maximum current, i.e.,
%
\begin{equation}
R_{\rm ESR} = \frac{\Delta V_{\rm OUT}}{I_{\rm STEP}} = 
\frac{75\text{m}}{15} \approx 5\text{m}\Omega
\end{equation}
%
The output capacitance requirements can be met using a pair
of 330$\mu$F 2R5TPE330M9 (2.5V, 9m$\Omega$, 3.9A$_{\rm RMS}$)
Sanyo POSCAP TPE-series capacitors.
%
The output voltage dip for a 15A output current increase will
be around 68mV, the output voltage peaking for a 15A output
current decrease will be around 54mV, and the ripple
voltage will be approximately 25mV.

\item Output voltage.

The output voltage is determined by the output feedback resistors
and the controller reference voltage, 
%
\begin{equation}
V_{\rm OUT} = V_{\rm REF}\left(1+\frac{R_{\rm a}}{R_{\rm b}}\right)
\end{equation}
%
where $V_{\rm REF} = 0.8$V, $R_{\rm a}$ is the resistor from the
output voltage to the controller $V_{\rm OSENSE}$ pin, and 
$R_{\rm b}$ is the resistor from the controller $V_{\rm OSENSE}$ 
pin to ground.

The LTC3851A example design used $R_{\rm a} = 43.2\text{k}\Omega$ and 
$R_{\rm b} = 49.9\text{k}\Omega$, resulting in an output
voltage of $V_{\rm OUT} = 1.493$V.
%
The nearest pair of 1-percent resistor values for an output voltage
of 1.5V are actually $R_{\rm a} = 9.31\text{k}\Omega$ and 
$R_{\rm b} = 10.7\text{k}\Omega$, resulting in an output
voltage of $V_{\rm OUT} = 1.496$V.
%
This design retains the same resistor values as the LTC3851A example.

\item Soft-start capacitance.

The soft-start time is 
%
\begin{equation}
t_{\rm SS} = 0.8\text{V}\cdot\frac{C_{\rm SS}}{1.0\mu\text{A}}
\end{equation}
%
A soft-start capacitance of $C_{\rm SS} = 2$nF was selected 
to give a soft-start time of 1.6ms. This results in the 
LTspice simulation having similar power-on timing to the
LTC1735 example.

The voltage on the soft-start capacitor ramps the output voltage
during power on, and thus limits the power-on current.
The ramping of the soft-start voltage and the output voltage
can be viewed in the LTspice simulation by plotting the
soft-start voltage \verb+V(ss)+ and the voltage on the
error amplifier feedback input \verb+V(fb)+.

\item Compensation components.

The LTC3851A example design uses the compensation components
$R_1  = 6.49\text{k}\Omega$, $C_1 = 3300$pF, and
$C_2 = 470$pF. These components create a compensation
network with a zero frequency of
%
\begin{equation}
f_{\rm Z} = \frac{1}{2\pi R_1 C_1} = 7.4\text{kHz}
\end{equation}
%
and a pole frequency of
%
\begin{equation}
f_{\rm P} = \frac{1}{2\pi R_1 C_1C_2/(C_1+C_2)}
= 59.6\text{kHz}
\end{equation}
\end{enumerate}

\clearpage
% -----------------------------------------------------------------
\subsubsection{Transient Response Analysis}
% -----------------------------------------------------------------

Figure~\ref{fig:LTC3851A_ex1_transient_circuit} shows an LTspice
circuit for analyzing the transient response of the
LTC3851A 12V to 1.5V@15A power supply design.
Figure~\ref{fig:LTC3851A_ex1_transient_response} shows the transient
response of several controller waveforms. 
The top trace in Figure~\ref{fig:LTC3851A_ex1_transient_response}
show the output voltage, \verb+V(out)+, the feedback voltage
\verb+V(fb)+, and the soft-start voltage, \verb+V(ss)+. The
feedback voltage tracks the soft-start voltage until the
feedback voltage reaches the reference voltage of 
$V_{\rm REF} = 0.8$V, after which point, the output voltage is
in regulation. During the soft-start power-on time, the
error amplifier (compensation) output voltage, \verb+V(comp)+,
and the inductor current, \verb+I(L)+ are well below their
maximum values. If these signals were observed to be too high,
then the soft-start time would need to be increased.

Figure~\ref{fig:LTC3851A_ex1_transient_response} shows the transient
response for an output current load step from 1A to 15A (a 14A load step),
applied at $t = 2.0$ms and removed at $t = 2.4$ms. 
Figure~\ref{fig:LTC3851A_ex1_transient_response}(b) shows a zoomed view of
the controller response to the load step. The compensation
response, \verb+V(comp)+, has no overshoot or ringing, indicating
the closed-loop response is stable with good phase margin.
The inductor current ripple of $\Delta I_{\rm OUT} = 5.6$A
can be seen in the inductor current waveform, \verb+I(L)+,
in the bottom trace in Figure~\ref{fig:LTC3851A_ex1_transient_response}(b).
The AC component of the inductor current travels through the output
capacitance, and generates the output ripple voltage over the output
capacitance ESR.  This ripple voltage can be seen in \verb+V(out)+, in
the top trace in Figure~\ref{fig:LTC3851A_ex1_transient_response}(b). 
The peak-to-peak output ripple voltage matches the expected value of 
$5.6\text{A}\times4.5\text{m}\Omega=25$mV.
The output voltage has an average value of 1.493V.
The output voltage dip and peaking during the application and removal
of the output current step is about -110mV and +90mV respectively.
The top plot in
Figure~\ref{fig:LTC3851A_ex1_transient_response}(b)
contains traces showing the voltage transient limits at
1.425V and 1.575V. The output current load transient causes
output voltage transients that both violate the power supply
output voltage specification.

The transient response in 
Figure~\ref{fig:LTC3851A_ex1_transient_response}(b) is
control loop limited, i.e., the loop bandwidth needs to be increased
so that the voltage dip transient is output capacitor ESR dominated.
The next section measures the power supply control loop to determine
whether the control loop bandwidth can be increased, while maintaining
adequate phase margin.

% -----------------------------------------------------------------
% LTspice circuit
% -----------------------------------------------------------------
%
\begin{landscape}
\setlength{\unitlength}{1mm}
\begin{figure}[p]
  \begin{center}
    \includegraphics[width=210mm]
    {figures/LTC3851A_ex1_transient_circuit.pdf}
  \end{center}
  \caption{LTC3851A 12V to 1.5V@15A LTspice transient response analysis circuit.}
  \label{fig:LTC3851A_ex1_transient_circuit}
\end{figure}
\end{landscape}

% -----------------------------------------------------------------
% Transient response waveforms
% -----------------------------------------------------------------
%
\setlength{\unitlength}{1mm}
\begin{figure}[p]
  \begin{picture}(155,205)(0,0)
    \put(10,105){
    \includegraphics[width=0.87\textwidth]
    {figures/LTC3851A_ex1_transient_response_a.pdf}}
    \put(10,0){
    \includegraphics[width=0.87\textwidth]
    {figures/LTC3851A_ex1_transient_response_b.pdf}}
    \put(77,105){(a)}
    \put(77, 0){(b)}
  \end{picture}
  \caption{LTC3851A 12V to 1.5V@15A supply LTspice transient response;
  (a) from $t=0$, and (b) zoomed view from $t=1.7$ms to 2.7ms.}
  \label{fig:LTC3851A_ex1_transient_response}
\end{figure}

\clearpage
% -----------------------------------------------------------------
\subsubsection{Frequency Response Analysis}
% -----------------------------------------------------------------

Figure~\ref{fig:LTC3851A_ex1_bode_circuit} shows an LTspice
circuit for analyzing the Bode response of the
LTC3851A 12V to 1.5V@15A power supply design.
Figure~\ref{fig:LTC3851A_ex1_bode_response} shows the Bode 
response.
The Bode response shows that the loop gain has a
cross-over frequency of 41kHz with 69-degrees of phase margin.
This is consistent with the transient response observed in
Figure~\ref{fig:LTC3851A_ex1_transient_response}. 
The problem with the transient response though, is that it
does not meet the output voltage regulation requirements.
The Bode magnitude plot in
Figure~\ref{fig:LTC3851A_ex1_bode_response}(a) shows that the
compensation response has less than 0dB gain at the switching
frequency, indicating that there is some room for 
compensation network adjustment.

Table~\ref{fig:LTC3851A_ex1_compensation} shows the results of 
{\em designing} a compensation network using a MATLAB script.
The MATLAB script imports the control-to-output gain measured 
using LTspice, and then allows the user to adjust the compensation
response by specifying the desired cross-over frequency and
zero and pole locations. The script then calculates the nearest
1-percent resistor value and standard capacitor values.
Table~\ref{fig:LTC3851A_ex1_compensation} shows the compensation
parameters for the circuit in 
Figure~\ref{fig:LTC3851A_ex1_bode_response} and an alternative
set of parameters.

To improve the transient response of the power supply, the 
circuit in Figure~\ref{fig:LTC3851A_ex1_bode_circuit} was modified;
%
\begin{itemize}
\item The feedback resistors were changed to 
$R_{\rm a} = 1.65\text{k}\Omega$ and 
$R_{\rm b} = 1.87\text{k}\Omega$. This results in a slightly higher
output voltage of $V_{\rm OUT} = 1.506$V. This change helps 
make the transient response symmetric.
\item The compensation components were changed per 
Table~\ref{fig:LTC3851A_ex1_compensation} design number \#2.
\item A 100$\mu$F ceramic output capacitor was added
(TDK C5750X5R0J107M X5R $R_{\rm ESR} = 2\text{m}\Omega$)~\footnote{LTspice
has an internal database of capacitor manufacturer parts that is
accessed by right-clicking on the capacitor schematic symbol, and
then clicking on the {\em Select Capacitor} button.}.
\end{itemize}
%
The transient response of the modified circuit is shown in
Figure~\ref{fig:LTC3851A_ex2_transient_response}, while 
Figure~\ref{fig:LTC3851A_ex2_bode_response} shows the Bode response.
A comparison of the original transient response with the new
shows that the new response reacts much quicker to the load step
due to the increase in loop bandwidth and in the compensation
zero frequency, but has slightly more ringing in the response
due to the slightly lower phase-margin.

The modified power supply design has an improved transient response that
marginally violates the output voltage specification for the design.
At this point, the performance could be accepted subject to hardware
testing, or additional simulation could be performed. The next
section discussions some of the options.

\clearpage
% -----------------------------------------------------------------
% LTspice circuit
% -----------------------------------------------------------------
%
\begin{landscape}
\setlength{\unitlength}{1mm}
\begin{figure}[p]
  \begin{center}
    \includegraphics[width=200mm]
    {figures/LTC3851A_ex1_bode_circuit.pdf}
  \end{center}
  \caption{LTC3851A 12V to 1.5V@15A LTspice Bode response analysis circuit.}
  \label{fig:LTC3851A_ex1_bode_circuit}
\end{figure}
\end{landscape}

\clearpage
% -----------------------------------------------------------------
% Bode response
% -----------------------------------------------------------------
%
\setlength{\unitlength}{1mm}
\begin{figure}[p]
  \begin{picture}(155,190)(0,0)
    \put(15,100){
    \includegraphics[width=0.75\textwidth]
    {figures/LTC3851A_ex1_bode_response_mag.pdf}}
    \put(15,4){
    \includegraphics[width=0.75\textwidth]
    {figures/LTC3851A_ex1_bode_response_phase.pdf}}
    \put(75,  97){(a)}
    \put(75,   0){(b)}
  \end{picture}
  \caption{LTC3851A 12V to 1.5V@15A supply Bode response;
  (a) magnitude, and (b) phase. The LC and compensation
  {\em calculated} responses are shown by the black and red
  solid lines, while the LTspice {\em simulated} responses are
  shown by the black and red crosses. The 
  control-to-output and open-loop LTspice {\em simulated} responses
  are shown using both solid lines and crosses.
  The open-loop gain has a cross-over frequency of 41kHz with
  69-degrees of phase margin.}
  \label{fig:LTC3851A_ex1_bode_response}
\end{figure}

\clearpage
% -----------------------------------------------------------------
% Compensation design
% -----------------------------------------------------------------
%
\begin{table}[p]
\caption{LTC3851A 12V to 1.5V@15A supply compensation design.}
\label{fig:LTC3851A_ex1_compensation}
\begin{center}
\begin{tabular}{|l|c|c|c|c|}
\hline
\rule{0cm}{4mm}Component or Parameter & \multicolumn{4}{c|}{Compensation Design Number}\\
\cline{2-5}
\rule{0cm}{4mm}     & \#1 & \#2 (3,4) & \#5 & \#6\\
\hline
\hline
\multicolumn{5}{|l|}{\bf Compensation response}\\
\hline
& \hspace {20mm} &  \hspace {20mm} &  \hspace {20mm} &  \hspace {20mm} \\
$R_1$       & 6.49k$\Omega$ &  7.5k$\Omega$ &  7.15k$\Omega$ &  7.15k$\Omega$\\
$C_1$       & 3300pF        &  680pF        &  560pF         &  2700pF       \\
$C_2$       &  470pF        &  330pF        &  180pF         &  180pF        \\
&&&&\\
$f_{\rm Z}$ &  7.4kHz       & 31.2kHz       & 39.8kHz        &  8.24kHz       \\
$f_{\rm P}$ & 59.6kHz       & 95.5kHz       & 163.4kHz       & 131.9kHz      \\
&&&&\\
Compensation gain at $f_{\rm SW}$ & -2.84dB & 0.07dB & 5.06dB & 5.19dB\\
&&&&\\
\hline
\multicolumn{5}{|l|}{\bf Open-loop response}\\
\hline
&&&&\\
Cross-over frequency    & 41.6kHz      & 49.4kHz      & 71.6kHz      & 76.1kHz      \\ 
Cross-over phase-margin & 69.9$^\circ$ & 57.0$^\circ$ & 56.2$^\circ$ & 72.0$^\circ$ \\
&&&&\\
\hline
\end{tabular}
\end{center}
\end{table}

% -----------------------------------------------------------------
% Transient response (Example#2)
% -----------------------------------------------------------------
%
\setlength{\unitlength}{1mm}
\begin{figure}[p]
  \begin{center}
    \includegraphics[width=0.87\textwidth]
    {figures/LTC3851A_ex2_transient_response.pdf}
  \end{center}
  \caption{LTC3851A 12V to 1.5V@15A supply transient response for design \#2.
  The circuit modifications were; a slight increase in the output voltage,
  a new compensation network, and a ceramic output capacitor was added.
  Compare the response in this figure to that in
  Figure~\ref{fig:LTC3851A_ex1_transient_response}(b).
  Figure~\ref{fig:LTC3851A_ex2_bode_response} shows the Bode response.}
  \label{fig:LTC3851A_ex2_transient_response}
\end{figure}

\clearpage
% -----------------------------------------------------------------
% Bode response (Example#2)
% -----------------------------------------------------------------
%
\setlength{\unitlength}{1mm}
\begin{figure}[p]
  \begin{picture}(155,190)(0,0)
    \put(15,100){
    \includegraphics[width=0.75\textwidth]
    {figures/LTC3851A_ex2_bode_response_mag.pdf}}
    \put(15,4){
    \includegraphics[width=0.75\textwidth]
    {figures/LTC3851A_ex2_bode_response_phase.pdf}}
    \put(75,  97){(a)}
    \put(75,   0){(b)}
  \end{picture}
  \caption{LTC3851A 12V to 1.5V@15A supply Bode response for design \#2;
  (a) magnitude, and (b) phase. The modified circuit added additional
  output capacitance, which causes the control-to-output response to
  move lower in frequency. The open-loop gain has a cross-over frequency
  of 43kHz with 48-degrees of phase margin. The slight ringing seen in the
  transient response is a reflection of the lower phase-margin.}
  \label{fig:LTC3851A_ex2_bode_response}
\end{figure}

\clearpage
% -----------------------------------------------------------------
\subsubsection{Control Loop Optimization}
% -----------------------------------------------------------------

The transient response in Figure~\ref{fig:LTC3851A_ex2_transient_response}
marginally violates the voltage specification of the design.
This violation can be avoided by increasing the loop bandwidth.
The loop bandwidth is increased by first modifying the current-loop
gain, and then adjusting the compensation network.

The compensation voltage waveform in Figure~\ref{fig:LTC3851A_ex2_transient_response},
\verb+V(comp)+, peaks at slightly over 1.4V. The LTC3851A data sheet
{\em Maximum Peak Current Sense Threshold vs $I_{\rm TH}$ Voltage} plot 
(p6~\cite{Linear_LTC3851A_2010}) shows that 1.4V is close to the limit
of the linear region of the error amplifier output.
The compensation voltage response can be reduced by reducing the current feedback.
Reducing the current feedback causes a slight increase
in the control-to-output gain, due to the lowered current feedback,
causing a slight increase in the open-loop gain cross-over frequency.
The LTC3851A has two options for adjusting the current feedback;
the $I_{\rm LIM}$ pin and the current sense resistance. These two
options can be considered coarse adjustment and fine adjustment.

Coarse adjustment of the current feedback can be controlled using
the LTC3851A $I_{\rm LIM}$ pin; 0V, floating, and $\text{INTV}_{\rm CC}$
select the {\em nominal} current-sense limits of 30mV, 53mV, and 80mV. 
These correspond to high, medium, and low current gain settings.
The effect of the $I_{\rm LIM}$ pin can be understood by considering
the power supply just analyzed. The supply was designed for the
53mV current-sense range ($I_{\rm LIM}$ floating), resulting in the
selection of a 2m$\Omega$ current-sense resistor. If the
80mV current-sense range ($I_{\rm LIM}$ connected to $\text{INTV}_{\rm CC}$)
is now selected {\em without} changing the sense resistor, then the
compensation voltage waveform will be scaled by $53\text{mV}/80\text{mV} = 0.663$,
i.e., the current-sense gain has been lowered.

Fine adjustment of the current feedback by adjusting the
current-sense resistance, is not very practical in designs that use an
actual {\em resistor} for current-sensing, due to the fact that the
low-ohmic value resistors come in limited increments. Current-sensing using
inductor DC resistance (DCR) sensing is much more flexible, as the RC-sense network
can also include attenuation control, providing the flexibility
to scale the current-sense gain arbitrarily. 
%
For example, the power supply just analyzed can be converted to use 
inductor DCR sensing. The design of a DCR sensing network is discussed
on pages 12 and 13 of the LTC3851A data 
sheet~\cite{Linear_LTC3851A_2010}. Defining $\alpha$ as the required
current-sense attenuation, and using the reference designators
from Figure 2 in the data sheet, the component values are calculated
via;
%
\begin{equation}
\begin{split}
\alpha &= \frac{R_{\rm SNS}}{R_{\rm DCR}}\\
C_1 &= 100\text{nF}\\
R_1 &= \frac{L}{\alpha R_{\rm DCR}C_1}\\
R_2 &= R_1\cdot\frac{\alpha}{(1-\alpha)}
\end{split}
\end{equation}
%
where $C_1$ is arbitrarily selected. Table~\ref{tab:LTC3851A_ex3_dcr_sensing}
shows the components required to implement an effective current-sense
value of 2.0m$\Omega$ as used in the original supply design, and the component
values needed for an effective current-sense resistance of 1.5m$\Omega$.
The resistance values in the table have been converted to the
nearest 1-percent values. 

\begin{table}[t]
\caption{LTC3851A 12V to 1.5V@15A supply inductor DCR current-sense components.}
\label{tab:LTC3851A_ex3_dcr_sensing}
\begin{center}
\begin{tabular}{|c|c|c|}
\hline
\rule{0cm}{4mm}Parameter or & \multicolumn{2}{c|}{Nominal Sense Resistance} \\
\cline{2-3}
\rule{0cm}{4mm}Component    & $2.0\text{m}\Omega$ & $1.5\text{m}\Omega$ \\
\hline\hline
         &\hspace {20mm} &\hspace {20mm}\\
$\alpha$ & 0.769         & 0.577         \\
$C_1$    & 100nF         & 100nF         \\
$R_1$    & 2.37k$\Omega$ & 3.16k$\Omega$ \\
$R_2$    & 7.87k$\Omega$ & 4.32k$\Omega$ \\
&&\\
\hline
\end{tabular}
\end{center}
\end{table}

Figures~\ref{fig:LTC3851A_ex3_transient_response} 
and~\ref{fig:LTC3851A_ex3_bode_response} show the transient and Bode
response for the $R_{\rm SNS} = 2.0\text{m}\Omega$ design.
A comparison between the transient responses in
Figures~\ref{fig:LTC3851A_ex2_transient_response} 
and~\ref{fig:LTC3851A_ex3_transient_response} 
shows the transient responses are virtually indistinguishable,
i.e., DCR current-sensing performs identically to resistive
current-sensing.
A comparison between the Bode responses in
Figures~\ref{fig:LTC3851A_ex2_bode_response} 
and~\ref{fig:LTC3851A_ex3_bode_response} shows that the
control-to-output response differs in amplitude and phase 
at low-frequencies. The DCR sensing design has higher gain near 
DC, which will result in better output regulation.

Figure~\ref{fig:LTC3851A_ex4_transient_response} 
and~\ref{fig:LTC3851A_ex4_bode_response} show the transient and Bode
response for the $R_{\rm SNS} = 1.5\text{m}\Omega$ design.
A comparison between the transient responses in
Figures~\ref{fig:LTC3851A_ex3_transient_response} 
and~\ref{fig:LTC3851A_ex4_transient_response} shows a slightly
faster transient in the $R_{\rm SNS} = 1.5\text{m}\Omega$ design,
due to a slight increase in the open-loop cross-over frequency.
A comparison between the Bode responses in
Figures~\ref{fig:LTC3851A_ex3_bode_response} 
and~\ref{fig:LTC3851A_ex4_bode_response} shows how the
control-to-output response shifts up slightly in the
$1.5\text{m}\Omega$ design.

The $R_{\rm SNS} = 1.5\text{m}\Omega$ design still does not meet
the output voltage regulation specification.
The compensation network used to produce the transient responses in
Figures~\ref{fig:LTC3851A_ex2_transient_response},
~\ref{fig:LTC3851A_ex3_transient_response},
and~\ref{fig:LTC3851A_ex4_transient_response}
was designed so that the compensation gain at the switching-frequency
was 0dB. This feature can be seen in the Bode responses in
Figures~\ref{fig:LTC3851A_ex2_bode_response}(a),
~\ref{fig:LTC3851A_ex3_bode_response}(a),
and~\ref{fig:LTC3851A_ex4_bode_response}(a).
Since the output voltage ripple is low (due to the use of the
additional ceramic output capacitor), this requirement can be
relaxed, allowing the open-loop bandwidth to be increased.
Figures~\ref{fig:LTC3851A_ex5_transient_response}
and~\ref{fig:LTC3851A_ex5_bode_response} show the transient
and Bode response for a compensation network design that
allows gain at the switching frequency (see 
Table~\ref{fig:LTC3851A_ex1_compensation} design \#5 for the
component values). The transient response
now meets the output voltage regulation requirement.

Figures~\ref{fig:LTC3851A_ex6_transient_response}
and~\ref{fig:LTC3851A_ex6_bode_response} show the transient
and Bode response for design \#6, the {\em final} supply design.
Table~\ref{fig:LTC3851A_ex1_compensation} shows that relative to
design \#5, the value for $C_1$ was increased to 2700pF,
causing the compensation zero to move to lower frequency.
This change was made to show that it was the {\em increase
in open-loop cross-over frequency} that led to the design
meeting the output voltage regulation specification.
The location of the low-frequency zero determines the responsiveness
of the controller after the initial transient. You can see
the effect of moving the zero by comparing the transient
responses in Figures~\ref{fig:LTC3851A_ex5_transient_response}
and~\ref{fig:LTC3851A_ex6_transient_response}; note how the
time taken for the output voltage to settle back to the nominal
output voltage takes longer in design \#6 than in design \#5.
Moving the zero to lower frequency provides additional phase
margin, so the transient response of the compensation voltage
and output current in 
Figure~\ref{fig:LTC3851A_ex6_transient_response}
have virtually no ringing.

The 1ms transient responses shown in the figures in this section simulate
very quickly in LTspice ($\sim$1 minute). The Bode responses are
constructed from multiple transient simulation runs; each of the 19 
points in the Bode plot is a transient simulation of 10 periods of the
sinusoid stimulus or 1ms, whichever time is greater. The lowest frequencies
of 100Hz and 300Hz dominate the simulation time.
The typical run time of the Bode analysis on an Intel Core i7 Q820 
1.73GHz quad-core machine with 16GB of RAM was in excess of 4 hours,
i.e., the Bode plots for the 6 designs required over 24 hours of 
simulation time. The Bode plots are however critical to the 
{\em design} of the compensation network, so the simulation time
cannot be avoided (just try to avoid too many iterations!).

The subtle changes in the transient and Bode figures is best
viewed using the electronic version of this document. The figures
have all been scaled and located on the page identically, so that
you can see the change in response when changing pages.

% -----------------------------------------------------------------
% Transient response (Example#3)
% -----------------------------------------------------------------
%
\setlength{\unitlength}{1mm}
\begin{figure}[p]
  \begin{center}
    \includegraphics[width=0.87\textwidth]
    {figures/LTC3851A_ex3_transient_response.pdf}
  \end{center}
  \caption{LTC3851A 12V to 1.5V@15A supply transient response for design \#3
  (inductor DCR sensing, $R_{\rm SNS} = 2.0\text{m}\Omega$).
  The response is virtually indistinguishable from the sense resistor
  version in Figure~\ref{fig:LTC3851A_ex2_transient_response}.\newline}
  \label{fig:LTC3851A_ex3_transient_response}
\end{figure}

% -----------------------------------------------------------------
% Transient response (Example#4)
% -----------------------------------------------------------------
%
\setlength{\unitlength}{1mm}
\begin{figure}[p]
  \begin{center}
    \includegraphics[width=0.87\textwidth]
    {figures/LTC3851A_ex4_transient_response.pdf}
  \end{center}
  \caption{LTC3851A 12V to 1.5V@15A supply transient response for design \#4
  (inductor DCR sensing, $R_{\rm SNS} = 1.5\text{m}\Omega$).
  The smaller effective current-sense resistance reduces the response
  of {\tt V(comp)}, relative to that in
  Figure~\ref{fig:LTC3851A_ex3_transient_response}. The transient response
  is now slightly faster due to the increase in open-loop bandwidth,
  however, the transient extrema still exceed the specification.}
  \label{fig:LTC3851A_ex4_transient_response}
\end{figure}

% -----------------------------------------------------------------
% Transient response (Example#5)
% -----------------------------------------------------------------
%
\setlength{\unitlength}{1mm}
\begin{figure}[p]
  \begin{center}
    \includegraphics[width=0.87\textwidth]
    {figures/LTC3851A_ex5_transient_response.pdf}
  \end{center}
  \caption{LTC3851A 12V to 1.5V@15A supply transient response for design \#5
  (inductor DCR sensing, $R_{\rm SNS} = 1.5\text{m}\Omega$, with modified
   compensation network). The modified compensation network moves the
   low frequency zero in the compensation response higher, and the
   cross-over frequency higher. The transient response now meets
   the output voltage regulation specification.}
  \label{fig:LTC3851A_ex5_transient_response}
\end{figure}

% -----------------------------------------------------------------
% Transient response (Example#6)
% -----------------------------------------------------------------
%
\setlength{\unitlength}{1mm}
\begin{figure}[p]
  \begin{center}
    \includegraphics[width=0.87\textwidth]
    {figures/LTC3851A_ex6_transient_response.pdf}
  \end{center}
  \caption{LTC3851A 12V to 1.5V@15A supply transient response for design \#6
  (the {\em final} design). The zero in the compensation network was moved
   lower, resulting in a slightly longer settling time after each transient,
   but less ringing (more phase-margin), while still meeting the output voltage 
   regulation specification.}
  \label{fig:LTC3851A_ex6_transient_response}
\end{figure}

\clearpage
% -----------------------------------------------------------------
% Bode response (Example#3)
% -----------------------------------------------------------------
%
\setlength{\unitlength}{1mm}
\begin{figure}[p]
  \begin{picture}(155,190)(0,0)
    \put(15,100){
    \includegraphics[width=0.75\textwidth]
    {figures/LTC3851A_ex3_bode_response_mag.pdf}}
    \put(15,4){
    \includegraphics[width=0.75\textwidth]
    {figures/LTC3851A_ex3_bode_response_phase.pdf}}
    \put(75,  97){(a)}
    \put(75,   0){(b)}
  \end{picture}
  \caption{LTC3851A 12V to 1.5V@15A supply Bode response for design \#3
  (inductor DCR sensing, $R_{\rm SNS} = 2.0\text{m}\Omega$);
  (a) magnitude, and (b) phase.
  The open-loop gain has a cross-over frequency of 43kHz with
  48-degrees of phase margin.\newline\newline}
  \label{fig:LTC3851A_ex3_bode_response}
\end{figure}

\clearpage
% -----------------------------------------------------------------
% Bode response (Example#4)
% -----------------------------------------------------------------
%
\setlength{\unitlength}{1mm}
\begin{figure}[p]
  \begin{picture}(155,190)(0,0)
    \put(15,100){
    \includegraphics[width=0.75\textwidth]
    {figures/LTC3851A_ex4_bode_response_mag.pdf}}
    \put(15,4){
    \includegraphics[width=0.75\textwidth]
    {figures/LTC3851A_ex4_bode_response_phase.pdf}}
    \put(75,  97){(a)}
    \put(75,   0){(b)}
  \end{picture}
  \caption{LTC3851A 12V to 1.5V@15A supply Bode response for design \#4
  (inductor DCR sensing, $R_{\rm SNS} = 1.5\text{m}\Omega$);
  (a) magnitude, and (b) phase.
  The open-loop gain has a cross-over frequency of 55kHz with
  49-degrees of phase margin.\newline\newline}
  \label{fig:LTC3851A_ex4_bode_response}
\end{figure}

\clearpage
% -----------------------------------------------------------------
% Bode response (Example#5)
% -----------------------------------------------------------------
%
\setlength{\unitlength}{1mm}
\begin{figure}[p]
  \begin{picture}(155,190)(0,0)
    \put(15,100){
    \includegraphics[width=0.75\textwidth]
    {figures/LTC3851A_ex5_bode_response_mag.pdf}}
    \put(15,4){
    \includegraphics[width=0.75\textwidth]
    {figures/LTC3851A_ex5_bode_response_phase.pdf}}
    \put(75,  97){(a)}
    \put(75,   0){(b)}
  \end{picture}
  \caption{LTC3851A 12V to 1.5V@15A supply Bode response for design \#5
  (inductor DCR sensing, $R_{\rm SNS} = 1.5\text{m}\Omega$, compensation
   gain at the switching-frequency);
  (a) magnitude, and (b) phase.
  The open-loop gain has a cross-over frequency of 71kHz with
  56-degrees of phase margin.\newline\newline}
  \label{fig:LTC3851A_ex5_bode_response}
\end{figure}

\clearpage
% -----------------------------------------------------------------
% Bode response (Example#6)
% -----------------------------------------------------------------
%
\setlength{\unitlength}{1mm}
\begin{figure}[p]
  \begin{picture}(155,190)(0,0)
    \put(15,100){
    \includegraphics[width=0.75\textwidth]
    {figures/LTC3851A_ex6_bode_response_mag.pdf}}
    \put(15,4){
    \includegraphics[width=0.75\textwidth]
    {figures/LTC3851A_ex6_bode_response_phase.pdf}}
    \put(75,  97){(a)}
    \put(75,   0){(b)}
  \end{picture}
  \caption{LTC3851A 12V to 1.5V@15A supply Bode response for design \#6
  (inductor DCR sensing, $R_{\rm SNS} = 1.5\text{m}\Omega$, compensation
   gain at the switching-frequency, compensation zero moved to $\sim$10kHz);
  (a) magnitude, and (b) phase.
  The open-loop gain has a cross-over frequency of 76kHz with
  72-degrees of phase margin.\newline\newline}
  \label{fig:LTC3851A_ex6_bode_response}
\end{figure}

\clearpage
% -----------------------------------------------------------------
\subsubsection{Review and Discussion}
% -----------------------------------------------------------------

\noindent{\bf TODO}:
\begin{itemize}
\item What other design changes could be looked at? Go back to
Rdcr = 2.0mOhm, so that the compensation voltage was close to
full-scale, and then redo the compensation components to have
a high cross-over, with say 6dB of gain at the switching frequency. 
\item Sense resistor versus inductor DCR sensing
\item NTC inductor temperature compensation.
\item Review the power dissipation in the MOSFETs.
\item Can I get an efficiency report?
\item Add a comment about output loads based on current sinks rather
than resistive loads, per the comments in~\cite{Linear_DC247_1999}
\item Find ESR of ceramic caps and add a couple on the output.
\end{itemize}


\clearpage
% =================================================================
\subsection{LTC3855: 12V to 0.95V@40A}
% =================================================================
\label{sec:LTC3855}

The Altera Stratix IV GT series of FPGAs uses a 0.95V core voltage,
with a voltage tolerance of $\pm$30mV. The EP4S100G5F45I1 device 
can generate load current steps of up to 40A. The following
sections contain the design of a dual-phase supply for conversion
of 12V to 0.95V@40A. The design analyses the requirements of a
single-phase for conversion of 12V to 0.95V@20A, and then
converts that to a dual-phase design. A comparison of the
single-phase to dual-phase design shows the advantages of
multi-phase controllers.

% -----------------------------------------------------------------
\subsubsection{Component Selection}
% -----------------------------------------------------------------

Power supply requirements;
%
\begin{itemize}
\item Input supply voltage, $V_{\rm IN} = 12$V
\item Output supply voltage, $V_{\rm OUT} = 0.95$V
\item Number of phases, $N_{\rm PH} = 2$
\item Output supply current (total), $I_{\rm OUT(TOTAL)} = 40$A
\item Output supply current (per phase), $I_{\rm OUT} = 20$A
\item Output supply voltage ripple, $\Delta V_{\rm OUT} = \pm 30$mV ($\pm3$\%)
\item Controller switching frequency, $f_{\rm SW} = 750$kHz
\end{itemize}
%
The controller components are selected based on the data
sheet procedure (see p11~\cite{Linear_LTC3855_2009});
%
\begin{enumerate}
\item Select the {\em target} current-sense resistor (implemented
via inductor DCR sensing).

\begin{equation}
R_{\rm SENSE} = 0.8\cdot\frac{V_{\rm MAX}}{(I_{\rm OUT} + \Delta I_{\rm OUT}/2)} =
0.8\cdot\frac{50\text{mV}}{22.7\text{A}} = 1.76\text{m}\Omega 
\end{equation}
%
where the 0.8 scale factor accounts for component tolerances (per the LTC3851A design),
and the per phase maximum current, $I_{\rm MAX} = I_{\rm OUT} + \Delta I_{\rm OUT}/2 = 22\text{A} + 5.3\text{A}/2 \approx 22.7$A, is calculated shortly.
%
\item Select $R_{\rm FREQ}$ to select the controller operating
frequency.

$R_{\rm FREQ} = 180\text{k}\Omega$ selects a switching frequency of
750kHz (the LTspice model was used to confirm this selection).
Because this controller is used for high load currents, with a 
tight output voltage tolerance, the open-loop bandwidth will need
to be high (60kHz or higher). Using the rule-of-thumb that the cross-over
is $f_{\rm SW}/10$, the switching frequency should be over 600kHz.
The higher switching frequency will result
in lower inductor ripple current, but higher MOSFET switching losses.

\item Inductor selection.

The {\em nominal} inductor value is determined based on a ripple current of
30\% of the maximum per phase output current, i.e., 
$\Delta I_{\rm OUT} = 0.3I_{\rm OUT} = 6.7$A. The inductor value 
is then
%
\begin{equation}
L = \frac{V_{\rm OUT}}{f_{\rm SW}\Delta I_{\rm OUT}}\left(
1 - \frac{V_{\rm OUT}}{V_{\rm IN}}\right) =
\frac{0.95}{750\text{k}\times6.7}\left(
1 - \frac{0.95}{12}\right) = 0.17\mu\text{H}
\end{equation}
%
A review of Vishay/Dale IHLP series inductors showed that the nearest
value is the IHLP-3232DZ-01 $0.22\mu\text{H}$ 
($R_{\rm DCR(MAX)} = 1.68\text{m}\Omega$).
For $L = 0.22\mu$H, $\Delta I_{\rm OUT} = 5.3$A and 
$I_{\rm MAX} = 22.7$A. The IHLP-3232DZ-01 inductor can easily
support a peak load current of 22.7A, resulting in a temperature 
rise of about $20^\circ$C.

The LTC3855 supports current-sensing via current-sense resistor or
inductor DCR sensing. This design will use DCR sensing to minimize
power loss. Transient tests with the LTspice model showed a DCR
sense RC-network with an attenuation of 0.8 resulted in a transient
response on the \verb+V(comp)+ pin that remained within the
linear region of the $I_{\rm TH}$ voltage 
(see p7~\cite{Linear_LTC3855_2009}). The sense network components
are
%
\begin{equation}
\begin{split}
C_{\rm S} &= 100\text{nF}\\
R_{\rm S} &= 1.3\text{k}\Omega\\
R_{\rm ATTEN} &= 5.23\text{k}\Omega
\end{split}
\end{equation}
%
where $C_{\rm S}$ is selected to give resistors on the k$\Omega$
range, and the resistances are the nearest 1-percent values.

\item Power MOSFET and D1 selection.

The LTspice example downloaded from the LTC3855 web page uses a
Renesas RJK0305DP for the main (top) MOSFET 
($R_{\rm DS(ON)} = 6.7\text{m}\Omega$), and a Renesas RJK0301DP
for the synchronous (bottom) MOSFET ($R_{\rm DS(ON)} = 2.3\text{m}\Omega$).
These components are also used in this example.

{\bf TODO}: power estimates from LTspice. Compare these
Renesas MOSFETs to the Fairchild and TI dual-asymmetric packaged
devices. Do they have lower power dissipation? They certainly
have lower PCB real-estate requirements.

\item Input capacitance selection.

TODO: how do I calculate the value and ESR? Is it just based on
RMS handling? LTC3855 data sheet p17 has comments on ESR, but
nothing on the amount of capacitance.

\item Output capacitance selection.

The {\em nominal} value of the output capacitance is determined by 
the amount required to absorb the energy in the inductor during a
load step from maximum current to minimum current, i.e., the amount
of capacitance {\em per phase} is
%
\begin{equation}
\begin{split}
C_{\rm OUT} &> \frac{LI_{\rm STEP}^2}
{(V_{\rm OUT} + \Delta V_{\rm OUT})^2 - V_{\rm OUT}^2} 
\approx \frac{LI_{\rm STEP}^2}{2V_{\rm OUT}\Delta V_{\rm OUT}}\\
&= \frac{0.22\mu\times 20^2}{2\times0.95\times25\text{m}}\\
&= 1853\mu\text{F}
\end{split}
\end{equation}
%
where $I_{\rm STEP}=20$A and $\Delta V_{\rm OUT} = 25$mV was
used, since there will be additional ripple voltage due to 
the output capacitance ESR.

The {\em nominal} value of the output capacitance ESR is determined by
the allowable output voltage dip during a load step from minimum current
to maximum current, i.e.,
%
\begin{equation}
R_{\rm ESR} = \frac{\Delta V_{\rm OUT}}{I_{\rm STEP}} = 
\frac{30\text{m}}{20} = 1.5\text{m}\Omega
\end{equation}
%
The output capacitance requirements are dominated by the low ESR.
The per phase requirements can be met using four 
470$\mu$F 2R5TPF470M6L (2.5V, 6m$\Omega$, 4.4A$_{\rm RMS}$)
Sanyo POSCAP TPF-series capacitors.
%
The output voltage dip (peaking) for a 20A output current increase
(decrease) will be about 30mV. The single-phase ripple voltage will
be $1.5\text{m}\Omega \times 5.3\text{A} = 8$mV.
The dual-phase supply will have reduced output ripple voltage.

\item Output voltage.

The output voltage is determined by the output feedback resistors
and the controller reference voltage, 
%
\begin{equation}
V_{\rm OUT} = V_{\rm REF}\left(1+\frac{R_{\rm a}}{R_{\rm b}}\right)
\end{equation}
%
where $V_{\rm REF} = 0.6$V, $R_{\rm a}$ is the resistor from the
output voltage to the controller $V_{\rm OSENSE}$ pin, and 
$R_{\rm b}$ is the resistor from the controller $V_{\rm OSENSE}$ 
pin to ground.
%
The nearest pair of 1-percent resistor values for an output voltage
of 0.95V are $R_{\rm a} = 10.2\text{k}\Omega$ and 
$R_{\rm b} = 17.4\text{k}\Omega$, resulting in an output
voltage of $V_{\rm OUT} = 0.952$V.
%
The output voltage can be adjusted in approximately 2mV steps
using 1-percent resistors. For example, $R_{\rm a} = 9.31\text{k}\Omega$ and 
$R_{\rm b} = 15.8\text{k}\Omega$, results in an output
voltage of $V_{\rm OUT} = 0.954$V, and 
$R_{\rm a} = 6.34\text{k}\Omega$ and 
$R_{\rm b} = 10.7\text{k}\Omega$, results in an output
voltage of $V_{\rm OUT} = 0.956$V.  
%
Selecting an output voltage that is slightly higher than the nominal
0.95V, helps to trade off the ESR dip voltage, against the capacitance
peaking voltage, which can be more easily reduced by increasing
the output capacitance value.

\item Soft-start capacitance.

The soft-start time is 
%
\begin{equation}
t_{\rm SS} = 0.6\text{V}\cdot\frac{C_{\rm SS}}{1.2\mu\text{A}}
\end{equation}
%
The soft-start time should be configured such that the current 
required to charge the output capacitance is within normal
operating conditions. The average current required to charge
the output capacitance to the output voltage within the
soft-start time (assuming no load current) is
%
\begin{equation}
I_{\rm SS} = \frac{C_{\rm OUT}V_{\rm OUT}}{t_{\rm SS}}
\end{equation}
%
A soft-start capacitance of $C_{\rm SS} = 1800$pF results in a
soft-start time of 0.9ms, and an average output current
of $4\times470\mu\times0.95/0.9\text{m} = 2$A, which is
well within operating conditions. The power-on current
waveform during the soft-start time can be seen in 
Figure~\ref{fig:LTC3855_ex1_transient_response_power_on}(a).

\item Compensation components.

The LTC3855 example design uses the compensation components
$R_1  = 18.2\text{k}\Omega$, $C_1 = 1000$pF, and
$C_2 = 150$pF. These components create a compensation
network with a zero frequency of
%
\begin{equation}
f_{\rm Z} = \frac{1}{2\pi R_1 C_1} = 8.7\text{kHz}
\end{equation}
%
and a pole frequency of
%
\begin{equation}
f_{\rm P} = \frac{1}{2\pi R_1 C_1C_2/(C_1+C_2)}
= 67.7\text{kHz}
\end{equation}
%
These components were used as the initial compensation components,
and LTspice transient analysis runs were used to adjust the 
components until an acceptable transient response was obtained.
\end{enumerate}


\clearpage
% -----------------------------------------------------------------
\subsubsection{Transient Response Analysis}
% -----------------------------------------------------------------

Figures~\ref{fig:LTC3855_ex1_transient_circuit} 
and~\ref{fig:LTC3855_ex2_transient_circuit} show the LTspice circuits
used for analyzing the LTC3855 single-phase and dual-phase transient
response\footnote{The transient analysis run time for the
single-phase controller was about 8 minutes, while the run time
for the dual-phase controller was about 15 minutes.}.
In the single-phase circuit, the second channel is disabled
by grounding the \verb+RUN+ pin. Relative to the single-phase
circuit, the dual-phase circuit has the following changes;
extra power-stage MOSFETs and inductor, pins are tied together
per the data sheet recommendations (\verb+COMP+, \verb+FB+, 
\verb+ILIM+, \verb+RUN+, \verb+SS+, and \verb+TEMP+), the soft-start
capacitor value was doubled (since the \verb+SS+ pins are tied
together, the soft-start current doubles), the compensation 
network was modified, and the output current load step was
doubled.

Figure~\ref{fig:LTC3855_ex1_transient_response_power_on} shows
the single-phase power-on transient response (with $4\times470\mu$F
output capacitance). The dual-phase response is not shown, as it
looks similar. The transient response of the single-phase circuit in
Figure~\ref{fig:LTC3855_ex1_transient_response_power_on}(b) shows
that the load current transients cause violations of the output 
voltage regulation specification.
The load current increase causes the output voltage to drop to 
912mV (8mV violation), and the load current decrease causes the
output voltage to peak to 988mV (8mV violation).
The poor transient response is due to low control loop bandwidth.

Figure~\ref{fig:LTC3855_ex1_transient_response} shows the transient
response with an improved compensation network; $R_1 = 40.2\text{k}\Omega$,
$C_1 = 560$pF and $C_2 = 82$pF. The improved compensation network
values were determined by increasing the compensation mid-band gain
by increasing $R_1$, and then reducing $C_1$ and $C_2$ by a similar
amount ($C_1 = 470$pF and $C_2 = 68$pF produce virtually identical
transient responses).
The transient response in Figure~\ref{fig:LTC3855_ex1_transient_response}(a)
marginally violates the output voltage regulation specification.
The load current increase causes the output voltage to drop to 
916mV (4mV violation), and the load current decrease causes the
output voltage to peak to 978mV (2mV margin).
%
Figure~\ref{fig:LTC3855_ex1_transient_response}(b) shows that voltage
regulation specification can be met by increasing the output capacitance.
The load current increase causes the output voltage to drop to 
925mV (5mV margin), and the load current decrease causes the
output voltage to peak to 977mV (3mV margin).

Figure~\ref{fig:LTC3855_ex2_transient_response} shows the transient
response for the dual-phase controller for two values of output
capacitance. The compensation network for the dual-phase controller
was adjusted until its response was similar to that of the
single-phase controller. The compensation values were changed
to; $R_1 = 20\text{k}\Omega$, $C_1 = 1500$pF and $C_2 = 150$pF.
%
The transient response in Figure~\ref{fig:LTC3855_ex2_transient_response}(a)
marginally violates the output voltage regulation specification.
The load current increase causes the output voltage to drop to 
921mV (1mV margin), and the load current decrease causes the
output voltage to peak to 982mV (2mV violation).
%
Figure~\ref{fig:LTC3855_ex2_transient_response}(b) shows that voltage
regulation specification can be met by increasing the output capacitance.
The load current increase causes the output voltage to drop to 
925mV (5mV margin), and the load current decrease causes the
output voltage to peak to 975mV (3mV margin).

The bulk output capacitance uses Sanyo POSCAP-series capacitors.
The single-phase transient response in
Figure~\ref{fig:LTC3855_ex1_transient_response}(a)
used $4\times470\mu$F 2R5TPF470M6L (2.5V, 6m$\Omega$, 4.4A$_{\rm RMS}$),
while Figure~\ref{fig:LTC3855_ex1_transient_response}(b) used
$4\times1000\mu$F ETPF1000M6H (2.5V, 6m$\Omega$, 5.6A$_{\rm RMS}$).
The dual-phase transient response in
Figure~\ref{fig:LTC3855_ex2_transient_response}(a)
used $8\times470\mu$F 2R5TPF470M6L (2.5V, 6m$\Omega$, 4.4A$_{\rm RMS}$),
while Figure~\ref{fig:LTC3855_ex2_transient_response}(b) used
$8\times1000\mu$F ETPF1000M6H (2.5V, 6m$\Omega$, 5.6A$_{\rm RMS}$).
The Sanyo POSCAP catalog lists parts with slightly lower ESR as
{\em in development}; ETPF470M5H (2.5V, 5m$\Omega$, 6.1A$_{\rm RMS}$),
and ETPF1000M5H (2.5V, 5m$\Omega$, 6.1A$_{\rm RMS}$).

The key design aspect that the transient response analysis has
determined, is that the (single-) dual-phase supply requires
(four) eight POSCAP capacitors to meet the output voltage regulation
specification.

% -----------------------------------------------------------------
% Single-phase transient LTspice circuit
% -----------------------------------------------------------------
%
\begin{landscape}
\setlength{\unitlength}{1mm}
\begin{figure}[p]
  \begin{center}
    \includegraphics[width=210mm]
    {figures/LTC3855_ex1_transient_circuit.pdf}
  \end{center}
  \caption{LTC3855 12V to 0.95V@20A single-phase LTspice transient response analysis circuit.}
  \label{fig:LTC3855_ex1_transient_circuit}
\end{figure}
\end{landscape}

% -----------------------------------------------------------------
% Dual-phase transient LTspice circuit
% -----------------------------------------------------------------
%
\begin{landscape}
\setlength{\unitlength}{1mm}
\begin{figure}[p]
  \begin{center}
    \includegraphics[width=210mm]
    {figures/LTC3855_ex2_transient_circuit.pdf}
  \end{center}
  \caption{LTC3855 12V to 0.95V@40A dual-phase LTspice transient response analysis circuit.}
  \label{fig:LTC3855_ex2_transient_circuit}
\end{figure}
\end{landscape}

% -----------------------------------------------------------------
% Single-phase transient response waveforms
% -----------------------------------------------------------------
%
\setlength{\unitlength}{1mm}
\begin{figure}[p]
  \begin{picture}(155,205)(0,0)
    \put(10,105){
    \includegraphics[width=0.87\textwidth]
    {figures/LTC3855_ex1_transient_response_a.pdf}}
    \put(10,0){
    \includegraphics[width=0.87\textwidth]
    {figures/LTC3855_ex1_transient_response_b.pdf}}
    \put(77,105){(a)}
    \put(77, 0){(b)}
  \end{picture}
  \caption{LTC3855 12V to 0.95V@20A single-phase supply LTspice transient response;
  (a) from $t=0$, and (b) zoomed view from $t=1.1$ms to 2.1ms.}
  \label{fig:LTC3855_ex1_transient_response_power_on}
\end{figure}

% -----------------------------------------------------------------
% Single-phase transients for 470uF and 1000uF
% -----------------------------------------------------------------
%
\setlength{\unitlength}{1mm}
\begin{figure}[p]
  \begin{picture}(155,205)(0,0)
    \put(10,105){
    \includegraphics[width=0.87\textwidth]
    {figures/LTC3855_ex1_transient_response_c.pdf}}
    \put(10,0){
    \includegraphics[width=0.87\textwidth]
    {figures/LTC3855_ex1_transient_response_d.pdf}}
    \put(77,105){(a)}
    \put(77, 0){(b)}
  \end{picture}
  \caption{LTC3855 12V to 0.95V@20A single-phase supply LTspice {\em improved} 
  transient response for output capacitances of;
  (a) $4\times470\mu$F, and (b) $4\times1000\mu$F, both with 6m$\Omega$ ESR.}
  \label{fig:LTC3855_ex1_transient_response}
\end{figure}

% -----------------------------------------------------------------
% Dual-phase transients for 470uF and 1000uF
% -----------------------------------------------------------------
%
\setlength{\unitlength}{1mm}
\begin{figure}[p]
  \begin{picture}(155,205)(0,0)
    \put(10,105){
    \includegraphics[width=0.87\textwidth]
    {figures/LTC3855_ex2_transient_response_a.pdf}}
    \put(10,0){
    \includegraphics[width=0.87\textwidth]
    {figures/LTC3855_ex2_transient_response_b.pdf}}
    \put(77,105){(a)}
    \put(77, 0){(b)}
  \end{picture}
  \caption{LTC3855 12V to 0.95V@40A dual-phase supply LTspice 
  transient response for output capacitances of;
  (a) $8\times470\mu$F, and (b) $8\times1000\mu$F, both with 6m$\Omega$ ESR.}
  \label{fig:LTC3855_ex2_transient_response}
\end{figure}

\clearpage
% -----------------------------------------------------------------
\subsubsection{Frequency Response Analysis}
% -----------------------------------------------------------------

Figures~\ref{fig:LTC3855_ex1_bode_circuit} 
and~\ref{fig:LTC3855_ex2_bode_circuit} show the LTspice circuits
used for analyzing the LTC3855 single-phase and dual-phase Bode
response\footnote{The Bode analysis run time for the
single-phase controller was 14 hours 20 minutes, while the run
time for the dual-phase controller was 26 hours 56 minutes.}.
%
Figures~\ref{fig:LTC3855_ex1_bode_response}
and~\ref{fig:LTC3855_ex2_bode_response} show the Bode responses.
The Bode responses reasonable open-loop bandwidths, i.e., relative
to the switching frequency of $f_{\rm SW} = 750$kHz, the open-loop bandwidth
of 60kHz, is slightly under the target of $f_{\rm SW}/10 = 75$kHz.
The open-loop gain could be increased to 75kHz if more 
compensation gain was acceptable at the switching frequency.

% -----------------------------------------------------------------
\subsubsection{Review and Discussion}
% -----------------------------------------------------------------

The 12V to 0.95A@40A dual-phase controller transient response in
Figure~\ref{fig:LTC3855_ex2_transient_response}(b)
($8 \times 1000\mu$F output capacitance) shows that the design
meets the output voltage regulation specification, and the Bode 
response in Figure~\ref{fig:LTC3855_ex2_bode_response} shows that
the control loop is stable. 
At this point in the design phase, the power supply design would
be accepted, with final adjustments occurring after hardware tests.

\clearpage
% -----------------------------------------------------------------
% LTspice circuit
% -----------------------------------------------------------------
%
\begin{landscape}
\setlength{\unitlength}{1mm}
\begin{figure}[p]
  \begin{center}
    \includegraphics[width=200mm]
    {figures/LTC3855_ex1_bode_circuit.pdf}
  \end{center}
  \caption{LTC3851A 12V to 0.95V@20A single-phase LTspice Bode response analysis circuit.}
  \label{fig:LTC3855_ex1_bode_circuit}
\end{figure}
\end{landscape}

\clearpage
% -----------------------------------------------------------------
% LTspice circuit
% -----------------------------------------------------------------
%
\begin{landscape}
\setlength{\unitlength}{1mm}
\begin{figure}[p]
  \begin{center}
    \includegraphics[width=200mm]
    {figures/LTC3855_ex2_bode_circuit.pdf}
  \end{center}
  \caption{LTC3855 12V to 0.95V@40A dual-phase LTspice Bode response analysis circuit.}
  \label{fig:LTC3855_ex2_bode_circuit}
\end{figure}
\end{landscape}

\clearpage
% -----------------------------------------------------------------
% Bode response
% -----------------------------------------------------------------
%
\setlength{\unitlength}{1mm}
\begin{figure}[p]
  \begin{picture}(155,190)(0,0)
    \put(15,100){
    \includegraphics[width=0.75\textwidth]
    {figures/LTC3855_ex1_bode_response_mag.pdf}}
    \put(15,4){
    \includegraphics[width=0.75\textwidth]
    {figures/LTC3855_ex1_bode_response_phase.pdf}}
    \put(75,  97){(a)}
    \put(75,   0){(b)}
  \end{picture}
  \caption{LTC3855 12V to 0.95V@20A single-phase supply Bode response;
  (a) magnitude, and (b) phase. 
  The open-loop gain has a cross-over frequency of 61kHz with
  84-degrees of phase margin. The compensation network has
  a gain of 10dB at the switching frequency, i.e., the
  compensation signal contains the output voltage ripple 
  amplified by 3.3 times.}
  \label{fig:LTC3855_ex1_bode_response}
\end{figure}

\clearpage
% -----------------------------------------------------------------
% Bode response
% -----------------------------------------------------------------
%
\setlength{\unitlength}{1mm}
\begin{figure}[p]
  \begin{picture}(155,190)(0,0)
    \put(15,100){
    \includegraphics[width=0.75\textwidth]
    {figures/LTC3855_ex2_bode_response_mag.pdf}}
    \put(15,4){
    \includegraphics[width=0.75\textwidth]
    {figures/LTC3855_ex2_bode_response_phase.pdf}}
    \put(75,  97){(a)}
    \put(75,   0){(b)}
  \end{picture}
  \caption{LTC3855 12V to 0.95V@40A dual-phase supply Bode response;
  (a) magnitude, and (b) phase. 
  The open-loop gain has a cross-over frequency of 66kHz with
  85-degrees of phase margin. The compensation network has
  a gain of 11dB at the switching frequency, i.e., the
  compensation signal contains the output voltage ripple 
  amplified by 3.5 times.}
  \label{fig:LTC3855_ex2_bode_response}
\end{figure}




\clearpage
% =================================================================
\subsection{LTC $\mu$Modules}
% =================================================================


The Linear Technology product guide {\em Power management solutions
for Altera's FPGAs, CPLDs and Structured 
ASICs}~\cite{Linear_Altera_Product_Guide_2012} contains
Linear Technology power supply recommendations for Altera devices
based on input supply voltage, output supply voltage, and output current.
The LTM4601A, LTM4611, LTM4628, and LTM4627 $\mu$Modules are recommended
for many of the high-current applications. 
%
The Linear Technology web site~\cite{Linear_Altera_Reference_Designs_2012}, and
the {\em Altera Development Tools Guide}~\cite{Arrow_Altera_Devtool_Guide_2010}
from Arrow Electronics, contain a list Altera development kits along with the
Linear Technology devices used on each kit.

The LTM4601A $\mu$Module is an integrated module containing a
controller, MOSFETs, inductor, output capacitance, and 
compensation~\cite{Linear_LTM4601A_2011}.
The data sheet indicates that to use the module, the user
{\em simply} determines the output voltage using a resistor, and
adds input capacitance and output capacitance. The following
examples demonstrate that in practice things are not that
simple, due mainly to the fact that the compensation network
is not user-controlled.

% =================================================================
\subsubsection{LTM4601A: 12V to 1.5V@12A (single-phase)}
% =================================================================

For a $V_{\rm IN} = 12$V to $V_{\rm OUT} = 1.5\text{V} \pm 75$mV at 
$I_{\rm OUT} = 12$A single-phase power supply design, the
{\em nominal} output ripple current of the LTM4601A is
%
\begin{equation}
\Delta I_{\rm OUT} = \frac{V_{\rm OUT}}{f_{\rm SW}L}\left(
1 - \frac{V_{\rm OUT}}{V_{\rm IN}}\right) =
\frac{1.5}{850\text{k}\times0.47\mu}\left(
1 - \frac{1.5}{12}\right) = 3.3\text{A}
\end{equation}
%
where the nominal switching frequency, $f_{\rm SW} = 850$kHz, and
the internal inductor value, $L = 0.47\mu$F.

The {\em nominal} total output capacitance requirement is
%
\begin{equation}
\begin{split}
C_{\rm OUT} &> \frac{LI_{\rm STEP}^2}{2V_{\rm OUT}\Delta V_{\rm OUT}}\\
&= \frac{0.47\mu\times 11^2}{2\times1.5\times65\text{m}}\\
&= 292\mu\text{F}
\end{split}
\end{equation}
%
%
where $I_{\rm STEP}=11$A and $\Delta V_{\rm OUT} = 65$mV was
used, since there will be additional ripple voltage due to 
the output capacitance ESR.

The {\em nominal} value of the output capacitance ESR is determined by
the allowable output voltage dip during a load step from minimum current
to maximum current, i.e.,
%
\begin{equation}
R_{\rm ESR} = \frac{\Delta V_{\rm OUT}}{I_{\rm STEP}} = 
\frac{65\text{m}}{11} = 5.9\text{m}\Omega
\end{equation}
%
The output capacitance requirements can be met using a
Sanyo POSCAP TPLF-series
330$\mu$F ETPLF330M6 (2.5V, 6m$\Omega$, 4.7A$_{\rm RMS}$)
or TPF-series
330$\mu$F 2TPF330M6 (2.0V, 6m$\Omega$, 4.4A$_{\rm RMS}$)
or 2R5TPF330M7L (2.5V, 7m$\Omega$, 4.4A$_{\rm RMS}$).
A 100$\mu$F ceramic can be used in parallel with the POSCAP.
%
The output voltage dip (peaking) for a 11A output current increase
(decrease) will be about 66mV (44mV). The ripple voltage will
be $6\text{m}\Omega \times 3.3\text{A} = 20$mV.

Figure~\ref{fig:LTM4601A_ex1_transient_circuit} shows the LTspice
circuit used for transient analysis of the single-phase supply.
Figure~\ref{fig:LTM4601A_ex1_transient_response} shows the
transient response.
%
The transient response in 
Figure~\ref{fig:LTM4601A_ex1_transient_response}(a) fails to meet
the output voltage regulation of $\pm75$mV; the response dips to
1.373V (52mV violation) and peaks at 1.625V (50mV) violation.
Figure~\ref{fig:LTM4601A_ex1_transient_response}(b) shows that
the output voltage regulation requirement can almost be met
by adding a feed-forward capacitor of 330pF;  the response dips to
1.419V (6mV violation) and peaks at 1.579V (4mV) violation.

The feed-forward capacitor reacts with the internal 
$R_{\rm INT} = 60.4\text{k}\Omega$
and $R_{\rm SET}$ resistors in the output voltage-to-error 
amplifier feedback path. The addition of the feed-forward
capacitor converts the feedback path from a simple attenuator
%
\begin{equation}
\alpha_{\rm DC} = \frac{V_{\rm REF}}{V_{\rm OUT}} =
\frac{R_{\rm SET}}{R_{\rm INT} + R_{\rm SET}}
\end{equation}
%
to a frequency dependent network
%
\begin{equation}
\begin{split}
\alpha(s) &= 
\frac{R_{\rm SET}}{R_{\rm INT} + R_{\rm SET}}\cdot
\frac{1+sR_{\rm INT}C_{\rm FF}}
{1+s(R_{\rm INT}||R_{\rm SET})C_{\rm FF}}\\
&=
\frac{R_{\rm SET}}{R_{\rm INT} + R_{\rm SET}}\cdot
\frac{1+sR_{\rm INT}C_{\rm FF}}
{1+sR_{\rm INT}C_{\rm FF}V_{\rm REF}/V_{\rm OUT}}
\end{split}
\end{equation}
%
where use was made of the parallel resistor ratio
%
\begin{equation}
R_{\rm INT}||R_{\rm SET} = \frac{R_{\rm INT}R_{\rm SET}}{R_{\rm INT}+R_{\rm SET}} 
= R_{\rm INT}\cdot\frac{V_{\rm REF}}{V_{\rm OUT}}
\end{equation}
%
and the $\alpha_{\rm DC}$ output voltage relationship.
%
The frequency dependent feedback network has zero and pole frequencies
%
\begin{equation}
\begin{split}
f_{\rm Z} &= \frac{1}{2\pi R_{\rm INT}C_{\rm FF}}\\
f_{\rm P} &= \frac{1}{2\pi R_{\rm INT}C_{\rm FF}V_{\rm REF}/V_{\rm OUT}}
= f_{\rm Z}\cdot\frac{V_{\rm OUT}}{V_{\rm REF}}
\end{split}
\end{equation}
%
where since $V_{\rm OUT} > V_{\rm REF}$, the zero precedes the
pole\footnote{The feed-forward capacitor location in
Figure~\ref{fig:LTM4601A_ex1_transient_circuit} is as shown in 
the LTM4601A data sheet. This location is however incorrect.
The feed-forward capacitor should be placed between the {\tt VOUT\_LCL} and
{\tt FB} pins so that it is placed across the 60.4k$\Omega$
resistor. The location recommended by Linear effectively places the
capacitor across the positive input to the voltage sense amplifier,
{\tt VOSNS+}, and the {\tt FB} pin (error amplifier negative input). 
This is not ideal, as it will cause an imbalance in the differential
input impedance of the voltage sense amplifier, since
{\tt VOSNS+} and {\tt VOSNS-} are no longer impedance matched.}.
%
The addition of a feed-forward capacitor to adjust the compensation network
is {\em not recommended}, as it increases the compensation gain {\em at
all frequencies above the zero}. This results in an increase in the
compensation gain at the switching frequency, which can lead to
control loop instability.

Figure~\ref{fig:LTM4601A_ex1_bode_circuit} shows the LTspice
circuit used for Bode response analysis of the single-phase 
supply.
Figure~\ref{fig:LTM4601A_ex1a_bode_response} shows the
Bode response\footnote{The Bode analysis time
was 1 hour and 30 minutes.} with no feed-forward capacitor
(actually $C_{\rm FF} = 1$fF (femto-farad)).
The Bode response in Figure~\ref{fig:LTM4601A_ex1a_bode_response}
shows that the LTM4601A controller has an open-loop cross-over
frequency of only 23kHz, which is four times lower than the
target cross-over frequency of $f_{\rm SW}/10 = 85$kHz. 
The solid red-line in Figure~\ref{fig:LTM4601A_ex1a_bode_response}
represents a best-guess of the error amplifier gain and
compensation components (the slight deviation in phase at
higher frequencies is most likely due to the voltage-sense
amplifier). Because the user has no access to the compensation
network, there is no way to adjust the mid-band gain, to improve
the open-loop cross-over frequency. The only option available
to the user, is the feed-forward capacitor.

The purpose of the feed-forward capacitor is to increase the
open-loop cross-over frequency. The feed-forward capacitor can
only do this, if the feed-forward zero is located below the
nominal cross-over frequency. The maximum increase in
cross-over frequency occurs if both the zero and pole are
place below the nominal cross-over frequency.
For the Bode response in 
Figure~\ref{fig:LTM4601A_ex1a_bode_response}, the zero should
be located below $V_{\rm REF}/V_{\rm OUT}$ of the nominal cross-over
frequency, i.e., below $0.6/1.5\times23\text{kHz} = 9$kHz.
A feed-forward capacitor of $C_{\rm FF} = 330$pF places the
feed-forward zero and pole at 8kHz and 20kHz.
With this arrangement, the closed-loop gain at the nominal
cross-over frequency will increase by $V_{\rm OUT}/V_{\rm REF} = 2.5$
(8dB), increasing the cross-over frequency to somewhere near 58kHz.
Figure~\ref{fig:LTM4601A_ex1b_bode_response} shows the
Bode response with a feed-forward capacitor of
$C_{\rm FF} = 330$pF, with a cross-over frequency of 53kHz.
The increase in cross-over frequency is what improved the transient
response in Figure~\ref{fig:LTM4601A_ex1_transient_response}(b).

As commented earlier, feed-forward compensation is {\em not
recommended}. A comparison of the Bode responses in 
Figures~\ref{fig:LTM4601A_ex1a_bode_response}
and~\ref{fig:LTM4601A_ex1b_bode_response} shows that the
compensation gain at the switching frequency started out
high at 8dB, and was increased to 16dB by feed-forward
compensation. This gain amplifies the output voltage ripple
signal in the error amplifier output, which can cause
instability in the control loop.
This gain can be reduced by moving high-frequency pole
in the compensation loop down, by adding a capacitor to the
\verb+COMP+ pin. The LTM4627 $\mu$Module data sheet shows
example circuits with this additional capacitor, however,
the data sheet has no details on why it is used, i.e, the
casual user of the part has no idea what the capacitor
on the \verb+COMP+ pin is for. The data sheets refer the
user to use LTpowerCAD for control loop optimization,
however, that software currently has no such feature.

% -----------------------------------------------------------------
% Single-phase transient LTspice circuit
% -----------------------------------------------------------------
%
\begin{landscape}
\setlength{\unitlength}{1mm}
\begin{figure}[p]
  \begin{center}
    \includegraphics[width=210mm]
    {figures/LTM4601A_ex1_transient_circuit.pdf}
  \end{center}
  \caption{LTM4601A 12V to 1.5V@12A single-phase LTspice transient response analysis circuit.}
  \label{fig:LTM4601A_ex1_transient_circuit}
\end{figure}
\end{landscape}

% -----------------------------------------------------------------
% Single-phase transients for CFF = 1fF and 330pF
% -----------------------------------------------------------------
%
\setlength{\unitlength}{1mm}
\begin{figure}[p]
  \begin{picture}(155,205)(0,0)
    \put(10,105){
    \includegraphics[width=0.87\textwidth]
    {figures/LTM4601A_ex1a_transient_response.pdf}}
    \put(10,0){
    \includegraphics[width=0.87\textwidth]
    {figures/LTM4601A_ex1b_transient_response.pdf}}
    \put(77,105){(a)}
    \put(77, 0){(b)}
  \end{picture}
  \caption{LTM4601A 12V to 1.5V@12A single-phase LTspice transient response
  for feed-forward capacitances of;
  (a) $C_{\rm FF} = 1$fF (i.e., none), and (b) $C_{\rm FF} = 330$pF.
  The feed-forward capacitance improves the transient response.}
  \label{fig:LTM4601A_ex1_transient_response}
\end{figure}

% -----------------------------------------------------------------
% Single-phase Bode response LTspice circuit
% -----------------------------------------------------------------
%
\begin{landscape}
\setlength{\unitlength}{1mm}
\begin{figure}[p]
  \begin{center}
    \includegraphics[width=200mm]
    {figures/LTM4601A_ex1_bode_circuit.pdf}
  \end{center}
  \caption{LTM4601A 12V to 1.5V@12A single-phase LTspice Bode response analysis circuit.}
  \label{fig:LTM4601A_ex1_bode_circuit}
\end{figure}
\end{landscape}

% -----------------------------------------------------------------
% Single-phase Bode response for CFF = 1fF
% -----------------------------------------------------------------
%
\setlength{\unitlength}{1mm}
\begin{figure}[p]
  \begin{picture}(155,190)(0,0)
    \put(15,100){
    \includegraphics[width=0.75\textwidth]
    {figures/LTM4601A_ex1a_bode_response_mag.pdf}}
    \put(15,4){
    \includegraphics[width=0.75\textwidth]
    {figures/LTM4601A_ex1a_bode_response_phase.pdf}}
    \put(75,  97){(a)}
    \put(75,   0){(b)}
  \end{picture}
  \caption{LTM4601A 12V to 1.5V@12A single-phase supply Bode response
  ($C_{\rm FF} = 1$fF); (a) magnitude, and (b) phase. 
  The open-loop gain has a cross-over frequency of 23kHz with
  66-degrees of phase margin. The compensation network has
  a gain of 8dB at the switching frequency, i.e., the
  compensation signal contains the output voltage ripple 
  amplified by 2.5 times.}
  \label{fig:LTM4601A_ex1a_bode_response}
\end{figure}

% -----------------------------------------------------------------
% Single-phase Bode response for CFF = 330pF
% -----------------------------------------------------------------
%
\setlength{\unitlength}{1mm}
\begin{figure}[p]
  \begin{picture}(155,190)(0,0)
    \put(15,100){
    \includegraphics[width=0.75\textwidth]
    {figures/LTM4601A_ex1b_bode_response_mag.pdf}}
    \put(15,4){
    \includegraphics[width=0.75\textwidth]
    {figures/LTM4601A_ex1b_bode_response_phase.pdf}}
    \put(75,  97){(a)}
    \put(75,   0){(b)}
  \end{picture}
  \caption{LTM4601A 12V to 1.5V@12A single-phase supply Bode response
  ($C_{\rm FF} = 330$pF); (a) magnitude, and (b) phase. 
  The open-loop gain has a cross-over frequency of 53kHz with
  101-degrees of phase margin. The compensation network has
  a gain of 16dB at the switching frequency, i.e., the
  compensation signal contains the output voltage ripple 
  amplified by 6.3 times.}
  \label{fig:LTM4601A_ex1b_bode_response}
\end{figure}

\clearpage
% =================================================================
\subsubsection{LTM4601A: 12V to 1.5V@24A (dual-phase)}
% =================================================================

Figure~\ref{fig:LTM4601A_ex2_transient_circuit} shows the LTspice
circuit used for transient analysis of the 12V to 1.5V@24A dual-phase
supply. The dual-phase supply has been configured per Figure 19 on
page 23 of the data sheet~\cite{Linear_LTM4601A_2011}; 
an LTM4601A-1 has been added as the second power-stage,
an LTC6908-1 oscillator generates the 0 and 180$^\circ$
clocks, the appropriate pins on the two controllers are
tied together, the output voltage set resistor has been
adjusted, the output capacitance has been doubled, and
the load currents have been doubled.

Figure~\ref{fig:LTM4601A_ex2_transient_response} shows the
transient response. The transient response in 
Figure~\ref{fig:LTM4601A_ex2_transient_response}(a) fails to meet
the output voltage regulation of $\pm75$mV; the response dips to
1.379V (46mV violation) and peaks at 1.630V (55mV) violation.
Figure~\ref{fig:LTM4601A_ex2_transient_response}(b) shows that
the output voltage regulation requirement can almost be met
by adding a feed-forward capacitor of 680pF;  the response dips to
1.426V (1mV margin) and peaks at 1.583V (8mV) violation.

The transient responses in Figure~\ref{fig:LTM4601A_ex2_transient_response}
show an instability in the control loop. 
Figure~\ref{fig:LTM4601A_ex2_transient_currents} shows
a slightly modified transient response, where only the load
current increase is applied. The waveform
\verb|I(L1)+I(L2)-I(Rload)-I(Istep)| is the inductor ripple
current, while the waveform
\verb+I(L1)-I(L2)+ is the difference in inductor waveforms.
Both inductor waveforms should be fairly uniform with time,
with variation only visible near the load current transients.
Figure~\ref{fig:LTM4601A_ex2_transient_currents} shows that
this is not the case; after each transient, the relative phase of
the inductor waveforms changes from the nominal 180-degree
phase until the waveforms are {\em in-phase} (the
\verb+I(L1)-I(L2)+ waveform goes to zero). When the inductor
waveforms are in-phase, the output ripple voltage doubles, and 
the ripple fed back into the error amplifier also doubles.
The addition of the feed-forward compensation capacitor
{\em makes things worse} at the error amplifier output,
since the compensation network gain is increased at the
switching frequency. These issues are specific to the
LTM4601A dual-phase controller. The LTC3885 dual-phase
controller in Section~\ref{sec:LTC3855} shows none of
these issues.

Figure~\ref{fig:LTM4601A_ex2_bode_circuit} shows the LTspice
circuit used for Bode response analysis of the dual-phase supply.
Figures~\ref{fig:LTM4601A_ex2a_bode_response}
and~\ref{fig:LTM4601A_ex2b_bode_response}
show the Bode responses with $C_{\rm FF} = 1$fF and
$C_{\rm FF} = 680$pF (approximately twice 330pF)
\footnote{The Bode analysis time
was 8 hours and 6 minutes for each Bode response.}.
The comments for the single-phase controller apply to the
Bode responses for the dual-phase controller.


% -----------------------------------------------------------------
% Dual-phase transient LTspice circuit
% -----------------------------------------------------------------
%
\begin{landscape}
\setlength{\unitlength}{1mm}
\begin{figure}[p]
  \begin{center}
    \includegraphics[width=210mm]
    {figures/LTM4601A_ex2_transient_circuit.pdf}
  \end{center}
  \caption{LTM4601A 12V to 1.5V@24A dual-phase LTspice transient response analysis circuit.}
  \label{fig:LTM4601A_ex2_transient_circuit}
\end{figure}
\end{landscape}

% -----------------------------------------------------------------
% Dual-phase transients for CFF = 1fF and 330pF
% -----------------------------------------------------------------
%
\setlength{\unitlength}{1mm}
\begin{figure}[p]
  \begin{picture}(155,205)(0,0)
    \put(10,105){
    \includegraphics[width=0.87\textwidth]
    {figures/LTM4601A_ex2a_transient_response.pdf}}
    \put(10,0){
    \includegraphics[width=0.87\textwidth]
    {figures/LTM4601A_ex2b_transient_response.pdf}}
    \put(77,105){(a)}
    \put(77, 0){(b)}
  \end{picture}
  \caption{LTM4601A 12V to 1.5V@24A dual-phase LTspice transient response
  for feed-forward capacitances of;
  (a) $C_{\rm FF} = 1$fF (i.e., none), and (b) $C_{\rm FF} = 680$pF.
  The feed-forward capacitance improves the transient response.}
  \label{fig:LTM4601A_ex2_transient_response}
\end{figure}

% -----------------------------------------------------------------
% Dual-phase transient currents for CFF = 1fF and 330pF
% -----------------------------------------------------------------
%
\setlength{\unitlength}{1mm}
\begin{figure}[p]
  \begin{picture}(155,205)(0,0)
    \put(10,105){
    \includegraphics[width=0.87\textwidth]
    {figures/LTM4601A_ex2a_transient_currents.pdf}}
    \put(10,0){
    \includegraphics[width=0.87\textwidth]
    {figures/LTM4601A_ex2b_transient_currents.pdf}}
    \put(77,105){(a)}
    \put(77, 0){(b)}
  \end{picture}
  \caption{LTM4601A 12V to 1.5V@24A dual-phase LTspice transient
  response {\em instability} for feed-forward capacitances of;
  (a) $C_{\rm FF} = 1$fF (i.e., none), and (b) $C_{\rm FF} = 680$pF.
  Shortly after the transient load increase, the inductor phases 
  become {\em aligned}.}
  \label{fig:LTM4601A_ex2_transient_currents}
\end{figure}

% -----------------------------------------------------------------
% Dual-phase Bode response LTspice circuit
% -----------------------------------------------------------------
%
\begin{landscape}
\setlength{\unitlength}{1mm}
\begin{figure}[p]
  \begin{center}
    \includegraphics[width=200mm]
    {figures/LTM4601A_ex2_bode_circuit.pdf}
  \end{center}
  \caption{LTM4601A 12V to 1.5V@24A dual-phase LTspice Bode response analysis circuit.}
  \label{fig:LTM4601A_ex2_bode_circuit}
\end{figure}
\end{landscape}

% -----------------------------------------------------------------
% Dual-phase Bode response for CFF = 1fF
% -----------------------------------------------------------------
%
\setlength{\unitlength}{1mm}
\begin{figure}[p]
  \begin{picture}(155,190)(0,0)
    \put(15,100){
    \includegraphics[width=0.75\textwidth]
    {figures/LTM4601A_ex2a_bode_response_mag.pdf}}
    \put(15,4){
    \includegraphics[width=0.75\textwidth]
    {figures/LTM4601A_ex2a_bode_response_phase.pdf}}
    \put(75,  97){(a)}
    \put(75,   0){(b)}
  \end{picture}
  \caption{LTM4601A 12V to 0.95V@24A dual-phase supply Bode response
  ($C_{\rm FF} = 1$fF); (a) magnitude, and (b) phase. 
  The open-loop gain has a cross-over frequency of 23kHz with
  64-degrees of phase margin. The compensation network has
  a gain of 8dB at the switching frequency, i.e., the
  compensation signal contains the output voltage ripple 
  amplified by 2.5 times.}
  \label{fig:LTM4601A_ex2a_bode_response}
\end{figure}

% -----------------------------------------------------------------
% Dual-phase Bode response for CFF = 680pF
% -----------------------------------------------------------------
%
\setlength{\unitlength}{1mm}
\begin{figure}[p]
  \begin{picture}(155,190)(0,0)
    \put(15,100){
    \includegraphics[width=0.75\textwidth]
    {figures/LTM4601A_ex2b_bode_response_mag.pdf}}
    \put(15,4){
    \includegraphics[width=0.75\textwidth]
    {figures/LTM4601A_ex2b_bode_response_phase.pdf}}
    \put(75,  97){(a)}
    \put(75,   0){(b)}
  \end{picture}
  \caption{LTM4601A 12V to 0.95V@24A dual-phase supply Bode response
  ($C_{\rm FF} = 680$pF); (a) magnitude, and (b) phase. 
  The open-loop gain has a cross-over frequency of 55kHz with
  104-degrees of phase margin. The compensation network has
  a gain of 16dB at the switching frequency, i.e., the
  compensation signal contains the output voltage ripple 
  amplified by 6.3 times.}
  \label{fig:LTM4601A_ex2b_bode_response}
\end{figure}

\clearpage
% =================================================================
\subsubsection{LTM4601A: 12V to 0.95V@12A (single-phase)}
% =================================================================

For a $V_{\rm IN} = 12$V to $V_{\rm OUT} = 0.95\text{V} \pm 30$mV at 
$I_{\rm OUT} = 12$A single-phase power supply design, the
{\em nominal} output ripple current of the LTM4601A is
%
\begin{equation}
\Delta I_{\rm OUT} = \frac{V_{\rm OUT}}{f_{\rm SW}L}\left(
1 - \frac{V_{\rm OUT}}{V_{\rm IN}}\right) =
\frac{0.95}{850\text{k}\times0.47\mu}\left(
1 - \frac{0.95}{12}\right) = 2.2\text{A}
\end{equation}
%
where the nominal switching frequency, $f_{\rm SW} = 850$kHz, and
the internal inductor value, $L = 0.47\mu$F.

The {\em nominal} total output capacitance requirement is
%
\begin{equation}
\begin{split}
C_{\rm OUT} &> \frac{LI_{\rm STEP}^2}{2V_{\rm OUT}\Delta V_{\rm OUT}}\\
&= \frac{0.47\mu\times 11^2}{2\times0.95\times25\text{m}}\\
&= 1197\mu\text{F}
\end{split}
\end{equation}
%
%
where $I_{\rm STEP}=11$A and $\Delta V_{\rm OUT} = 25$mV was
used, since there will be additional ripple voltage due to 
the output capacitance ESR.

The {\em nominal} value of the output capacitance ESR is determined by
the allowable output voltage dip during a load step from minimum current
to maximum current, i.e.,
%
\begin{equation}
R_{\rm ESR} = \frac{\Delta V_{\rm OUT}}{I_{\rm STEP}} = 
\frac{25\text{m}}{11} = 2.3\text{m}\Omega
\end{equation}
%
The output capacitance requirements can be met using three
Sanyo POSCAP TPLF-series
330$\mu$F ETPLF330M6 (2.5V, 6m$\Omega$, 4.7A$_{\rm RMS}$)
in parallel with a 100$\mu$F ceramic.
%
The output voltage dip (peaking) for a 11A output current increase
(decrease) will be about 22mV (28mV). The ripple voltage will
be $2\text{m}\Omega \times 2.2\text{A} = 5$mV.

The transient and Bode response circuits for the 0.95V@12A supply
are similar to those shown for the 1.5V@20A supply in 
Figures~\ref{fig:LTM4601A_ex1_transient_circuit}
and~\ref{fig:LTM4601A_ex1_bode_circuit}, with the
output capacitor multiplier increased \verb+Mout1 = 3+,
and the set resistor changed to 103.5k$\Omega$
(which can be implemented via the parallel resistors
169k$\Omega$ and 267k$\Omega$).
Figure~\ref{fig:LTM4601A_ex3_transient_response} shows the
transient response, while
Figures~\ref{fig:LTM4601A_ex3a_bode_response}
and~\ref{fig:LTM4601A_ex3b_bode_response}
show the Bode responses with $C_{\rm FF} = 1$fF and
$C_{\rm FF} = 330$pF\footnote{The Bode analysis time
was 1 hour and 5 minutes for each Bode response.}.

The open-loop cross-over frequency of the nominal supply is
a very low 16kHz (less than 1/50th of the switching frequency).
The maximum gain that feed-forward compensation can implement
is $V_{\rm OUT}/V_{\rm REF} = 1.58$ (4dB), which does not
help the transient response significantly.
Using the same feed-forward capacitance value as
the 1.5V controller, $C_{\rm FF} = 330$pF, puts the
zero and pole at 8kHz and 13kHz, which are both below the
cross-over, providing maximum compensation gain.
Alas, as the transient response and Bode response shows,
this is simply not enough to improve the response of the
supply.

\clearpage
% -----------------------------------------------------------------
% Single-phase transients for CFF = 1fF and 330pF
% -----------------------------------------------------------------
%
\setlength{\unitlength}{1mm}
\begin{figure}[p]
  \begin{picture}(155,205)(0,0)
    \put(10,105){
    \includegraphics[width=0.87\textwidth]
    {figures/LTM4601A_ex3a_transient_response.pdf}}
    \put(10,0){
    \includegraphics[width=0.87\textwidth]
    {figures/LTM4601A_ex3b_transient_response.pdf}}
    \put(77,105){(a)}
    \put(77, 0){(b)}
  \end{picture}
  \caption{LTM4601A 12V to 0.95V@12A single-phase LTspice transient response
  for feed-forward capacitances of;
  (a) $C_{\rm FF} = 1$fF (i.e., none), and (b) $C_{\rm FF} = 330$pF.
  The feed-forward capacitance improves the transient response.}
  \label{fig:LTM4601A_ex3_transient_response}
\end{figure}

% -----------------------------------------------------------------
% Bode response for CFF = 1fF
% -----------------------------------------------------------------
%
\setlength{\unitlength}{1mm}
\begin{figure}[p]
  \begin{picture}(155,190)(0,0)
    \put(15,100){
    \includegraphics[width=0.75\textwidth]
    {figures/LTM4601A_ex3a_bode_response_mag.pdf}}
    \put(15,4){
    \includegraphics[width=0.75\textwidth]
    {figures/LTM4601A_ex3a_bode_response_phase.pdf}}
    \put(75,  97){(a)}
    \put(75,   0){(b)}
  \end{picture}
  \caption{LTM4601A 12V to 0.95V@12A single-phase supply Bode response
  ($C_{\rm FF} = 1$fF); (a) magnitude, and (b) phase. 
  The open-loop gain has a cross-over frequency of 16kHz with
  54-degrees of phase margin. The compensation network has
  a gain of 12dB at the switching frequency, i.e., the
  compensation signal contains the output voltage ripple 
  amplified by 4.0 times.}
  \label{fig:LTM4601A_ex3a_bode_response}
\end{figure}

% -----------------------------------------------------------------
% Bode response for CFF = 330pF
% -----------------------------------------------------------------
%
\setlength{\unitlength}{1mm}
\begin{figure}[p]
  \begin{picture}(155,190)(0,0)
    \put(15,100){
    \includegraphics[width=0.75\textwidth]
    {figures/LTM4601A_ex3b_bode_response_mag.pdf}}
    \put(15,4){
    \includegraphics[width=0.75\textwidth]
    {figures/LTM4601A_ex3b_bode_response_phase.pdf}}
    \put(75,  97){(a)}
    \put(75,   0){(b)}
  \end{picture}
  \caption{LTM4601A 12V to 0.95V@12A single-phase supply Bode response
  ($C_{\rm FF} = 330$pF); (a) magnitude, and (b) phase. 
  The open-loop gain has a cross-over frequency of 22kHz with
  75-degrees of phase margin. The compensation network has
  a gain of 16dB at the switching frequency, i.e., the
  compensation signal contains the output voltage ripple 
  amplified by 6.3 times.}
  \label{fig:LTM4601A_ex3b_bode_response}
\end{figure}

\clearpage
% =================================================================
\subsubsection{LTM4601A discussion}
% =================================================================

The Linear Technology product guide {\em Power management solutions
for Altera's FPGAs, CPLDs and Structured 
ASICs}~\cite{Linear_Altera_Product_Guide_2012}
promotes the LTM4601A for use as the core supply for Altera
Stratix IV and V series FPGAs for input voltages of 12V
and load currents of 5 to 10A. The analysis in this section
has shown that the LTM4601A controller can only meet the output
voltage regulation requirements of a 1.5V VCCIO bank of an FPGA.
The controller transient response is too slow to meet the output voltage
regulation requirements of the FPGA core supply.

So why then does Linear Technology promote the
use of this controller for powering Altera devices?
If you read the LTM4601A data sheet carefully, you will see that
the transient analysis of the controller is performed 
with respect to a load step of {\em only} 6A
(see Table 2 on page 18~\cite{Linear_LTM4601A_2011}).
Under the restriction that the LTM4601A current
handling is {\em derated} to only half the rated output current,
the transient requirements for 0.95V@6A supply should be met.
Re-running the 0.95V supply transient analysis for a constant current
of 1A, and a load step of 6A, showed that the nominal supply still 
failed, but the supply with feed-forward compensation met 
the output voltage requirements.
Unfortunately, a Stratix IV GT device with 530K logic
elements can generate a 40A load step. A multi-phase
LTM4601A controller would be pointless, as it would
require 8 phases, about ten times the PCB real-estate
as an LTC3855 (or LTC3880) dual-phase controller, and would
likely suffer terminal phasing instabilities.




\clearpage
% -----------------------------------------------------------------
% Do the bibliography
% -----------------------------------------------------------------
%Note, you can't have spaces in the list of bibliography files
%
\bibliography{refs}
\bibliographystyle{plain}

% -----------------------------------------------------------------
\end{document}











