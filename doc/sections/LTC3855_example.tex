% =================================================================
\subsection{LTC3855: 12V to 0.95V@40A}
% =================================================================
\label{sec:LTC3855}

The Altera Stratix IV GT series of FPGAs uses a 0.95V core voltage,
with a voltage tolerance of $\pm$30mV. The EP4S100G5F45I1 device 
can generate load current steps of up to 40A. The following
sections contain the design of a dual-phase supply for conversion
of 12V to 0.95V@40A. The design analyses the requirements of a
single-phase for conversion of 12V to 0.95V@20A, and then
converts that to a dual-phase design. A comparison of the
single-phase to dual-phase design shows the advantages of
multi-phase controllers.

% -----------------------------------------------------------------
\subsubsection{Component Selection}
% -----------------------------------------------------------------

Power supply requirements;
%
\begin{itemize}
\item Input supply voltage, $V_{\rm IN} = 12$V
\item Output supply voltage, $V_{\rm OUT} = 0.95$V
\item Number of phases, $N_{\rm PH} = 2$
\item Output supply current (total), $I_{\rm OUT(TOTAL)} = 40$A
\item Output supply current (per phase), $I_{\rm OUT} = 20$A
\item Output supply voltage ripple, $\Delta V_{\rm OUT} = \pm 30$mV ($\pm3$\%)
\item Controller switching frequency, $f_{\rm SW} = 750$kHz
\end{itemize}
%
The controller components are selected based on the data
sheet procedure (see p11~\cite{Linear_LTC3855_2009});
%
\begin{enumerate}
\item Select the {\em target} current-sense resistor (implemented
via inductor DCR sensing).

\begin{equation}
R_{\rm SENSE} = 0.8\cdot\frac{V_{\rm MAX}}{(I_{\rm OUT} + \Delta I_{\rm OUT}/2)} =
0.8\cdot\frac{50\text{mV}}{22.7\text{A}} = 1.76\text{m}\Omega 
\end{equation}
%
where the 0.8 scale factor accounts for component tolerances (per the LTC3851A design),
and the per phase maximum current, $I_{\rm MAX} = I_{\rm OUT} + \Delta I_{\rm OUT}/2 = 22\text{A} + 5.3\text{A}/2 \approx 22.7$A, is calculated shortly.
%
\item Select $R_{\rm FREQ}$ to select the controller operating
frequency.

$R_{\rm FREQ} = 180\text{k}\Omega$ selects a switching frequency of
750kHz (the LTspice model was used to confirm this selection).
Because this controller is used for high load currents, with a 
tight output voltage tolerance, the open-loop bandwidth will need
to be high (60kHz or higher). Using the rule-of-thumb that the cross-over
is $f_{\rm SW}/10$, the switching frequency should be over 600kHz.
The higher switching frequency will result
in lower inductor ripple current, but higher MOSFET switching losses.

\item Inductor selection.

The {\em nominal} inductor value is determined based on a ripple current of
30\% of the maximum per phase output current, i.e., 
$\Delta I_{\rm OUT} = 0.3I_{\rm OUT} = 6.7$A. The inductor value 
is then
%
\begin{equation}
L = \frac{V_{\rm OUT}}{f_{\rm SW}\Delta I_{\rm OUT}}\left(
1 - \frac{V_{\rm OUT}}{V_{\rm IN}}\right) =
\frac{0.95}{750\text{k}\times6.7}\left(
1 - \frac{0.95}{12}\right) = 0.17\mu\text{H}
\end{equation}
%
A review of Vishay/Dale IHLP series inductors showed that the nearest
value is the IHLP-3232DZ-01 $0.22\mu\text{H}$ 
($R_{\rm DCR(MAX)} = 1.68\text{m}\Omega$).
For $L = 0.22\mu$H, $\Delta I_{\rm OUT} = 5.3$A and 
$I_{\rm MAX} = 22.7$A. The IHLP-3232DZ-01 inductor can easily
support a peak load current of 22.7A, resulting in a temperature 
rise of about $20^\circ$C.

The LTC3855 supports current-sensing via current-sense resistor or
inductor DCR sensing. This design will use DCR sensing to minimize
power loss. Transient tests with the LTspice model showed a DCR
sense RC-network with an attenuation of 0.8 resulted in a transient
response on the \verb+V(comp)+ pin that remained within the
linear region of the $I_{\rm TH}$ voltage 
(see p7~\cite{Linear_LTC3855_2009}). The sense network components
are
%
\begin{equation}
\begin{split}
C_{\rm S} &= 100\text{nF}\\
R_{\rm S} &= 1.3\text{k}\Omega\\
R_{\rm ATTEN} &= 5.23\text{k}\Omega
\end{split}
\end{equation}
%
where $C_{\rm S}$ is selected to give resistors on the k$\Omega$
range, and the resistances are the nearest 1-percent values.

\item Power MOSFET and D1 selection.

The LTspice example downloaded from the LTC3855 web page uses a
Renesas RJK0305DP for the main (top) MOSFET 
($R_{\rm DS(ON)} = 6.7\text{m}\Omega$), and a Renesas RJK0301DP
for the synchronous (bottom) MOSFET ($R_{\rm DS(ON)} = 2.3\text{m}\Omega$).
These components are also used in this example.

{\bf TODO}: power estimates from LTspice. Compare these
Renesas MOSFETs to the Fairchild and TI dual-asymmetric packaged
devices. Do they have lower power dissipation? They certainly
have lower PCB real-estate requirements.

\item Input capacitance selection.

TODO: how do I calculate the value and ESR? Is it just based on
RMS handling? LTC3855 data sheet p17 has comments on ESR, but
nothing on the amount of capacitance.

\item Output capacitance selection.

The {\em nominal} value of the output capacitance is determined by 
the amount required to absorb the energy in the inductor during a
load step from maximum current to minimum current, i.e., the amount
of capacitance {\em per phase} is
%
\begin{equation}
\begin{split}
C_{\rm OUT} &> \frac{LI_{\rm STEP}^2}
{(V_{\rm OUT} + \Delta V_{\rm OUT})^2 - V_{\rm OUT}^2} 
\approx \frac{LI_{\rm STEP}^2}{2V_{\rm OUT}\Delta V_{\rm OUT}}\\
&= \frac{0.22\mu\times 20^2}{2\times0.95\times25\text{m}}\\
&= 1853\mu\text{F}
\end{split}
\end{equation}
%
where $I_{\rm STEP}=20$A and $\Delta V_{\rm OUT} = 25$mV was
used, since there will be additional ripple voltage due to 
the output capacitance ESR.

The {\em nominal} value of the output capacitance ESR is determined by
the allowable output voltage dip during a load step from minimum current
to maximum current, i.e.,
%
\begin{equation}
R_{\rm ESR} = \frac{\Delta V_{\rm OUT}}{I_{\rm STEP}} = 
\frac{30\text{m}}{20} = 1.5\text{m}\Omega
\end{equation}
%
The output capacitance requirements are dominated by the low ESR.
The per phase requirements can be met using four 
470$\mu$F 2R5TPF470M6L (2.5V, 6m$\Omega$, 4.4A$_{\rm RMS}$)
Sanyo POSCAP TPF-series capacitors.
%
The output voltage dip (peaking) for a 20A output current increase
(decrease) will be about 30mV. The single-phase ripple voltage will
be $1.5\text{m}\Omega \times 5.3\text{A} = 8$mV.
The dual-phase supply will have reduced output ripple voltage.

\item Output voltage.

The output voltage is determined by the output feedback resistors
and the controller reference voltage, 
%
\begin{equation}
V_{\rm OUT} = V_{\rm REF}\left(1+\frac{R_{\rm a}}{R_{\rm b}}\right)
\end{equation}
%
where $V_{\rm REF} = 0.6$V, $R_{\rm a}$ is the resistor from the
output voltage to the controller $V_{\rm OSENSE}$ pin, and 
$R_{\rm b}$ is the resistor from the controller $V_{\rm OSENSE}$ 
pin to ground.
%
The nearest pair of 1-percent resistor values for an output voltage
of 0.95V are $R_{\rm a} = 10.2\text{k}\Omega$ and 
$R_{\rm b} = 17.4\text{k}\Omega$, resulting in an output
voltage of $V_{\rm OUT} = 0.952$V.
%
The output voltage can be adjusted in approximately 2mV steps
using 1-percent resistors. For example, $R_{\rm a} = 9.31\text{k}\Omega$ and 
$R_{\rm b} = 15.8\text{k}\Omega$, results in an output
voltage of $V_{\rm OUT} = 0.954$V, and 
$R_{\rm a} = 6.34\text{k}\Omega$ and 
$R_{\rm b} = 10.7\text{k}\Omega$, results in an output
voltage of $V_{\rm OUT} = 0.956$V.  
%
Selecting an output voltage that is slightly higher than the nominal
0.95V, helps to trade off the ESR dip voltage, against the capacitance
peaking voltage, which can be more easily reduced by increasing
the output capacitance value.

\item Soft-start capacitance.

The soft-start time is 
%
\begin{equation}
t_{\rm SS} = 0.6\text{V}\cdot\frac{C_{\rm SS}}{1.2\mu\text{A}}
\end{equation}
%
The soft-start time should be configured such that the current 
required to charge the output capacitance is within normal
operating conditions. The average current required to charge
the output capacitance to the output voltage within the
soft-start time (assuming no load current) is
%
\begin{equation}
I_{\rm SS} = \frac{C_{\rm OUT}V_{\rm OUT}}{t_{\rm SS}}
\end{equation}
%
A soft-start capacitance of $C_{\rm SS} = 1800$pF results in a
soft-start time of 0.9ms, and an average output current
of $4\times470\mu\times0.95/0.9\text{m} = 2$A, which is
well within operating conditions. The power-on current
waveform during the soft-start time can be seen in 
Figure~\ref{fig:LTC3855_ex1_transient_response_power_on}(a).

\item Compensation components.

The LTC3855 example design uses the compensation components
$R_1  = 18.2\text{k}\Omega$, $C_1 = 1000$pF, and
$C_2 = 150$pF. These components create a compensation
network with a zero frequency of
%
\begin{equation}
f_{\rm Z} = \frac{1}{2\pi R_1 C_1} = 8.7\text{kHz}
\end{equation}
%
and a pole frequency of
%
\begin{equation}
f_{\rm P} = \frac{1}{2\pi R_1 C_1C_2/(C_1+C_2)}
= 67.7\text{kHz}
\end{equation}
%
These components were used as the initial compensation components,
and LTspice transient analysis runs were used to adjust the 
components until an acceptable transient response was obtained.
\end{enumerate}


\clearpage
% -----------------------------------------------------------------
\subsubsection{Transient Response Analysis}
% -----------------------------------------------------------------

Figures~\ref{fig:LTC3855_ex1_transient_circuit} 
and~\ref{fig:LTC3855_ex2_transient_circuit} show the LTspice circuits
used for analyzing the LTC3855 single-phase and dual-phase transient
response\footnote{The transient analysis run time for the
single-phase controller was about 8 minutes, while the run time
for the dual-phase controller was about 15 minutes.}.
In the single-phase circuit, the second channel is disabled
by grounding the \verb+RUN+ pin. Relative to the single-phase
circuit, the dual-phase circuit has the following changes;
extra power-stage MOSFETs and inductor, pins are tied together
per the data sheet recommendations (\verb+COMP+, \verb+FB+, 
\verb+ILIM+, \verb+RUN+, \verb+SS+, and \verb+TEMP+), the soft-start
capacitor value was doubled (since the \verb+SS+ pins are tied
together, the soft-start current doubles), the compensation 
network was modified, and the output current load step was
doubled.

Figure~\ref{fig:LTC3855_ex1_transient_response_power_on} shows
the single-phase power-on transient response (with $4\times470\mu$F
output capacitance). The dual-phase response is not shown, as it
looks similar. The transient response of the single-phase circuit in
Figure~\ref{fig:LTC3855_ex1_transient_response_power_on}(b) shows
that the load current transients cause violations of the output 
voltage regulation specification.
The load current increase causes the output voltage to drop to 
912mV (8mV violation), and the load current decrease causes the
output voltage to peak to 988mV (8mV violation).
The poor transient response is due to low control loop bandwidth.

Figure~\ref{fig:LTC3855_ex1_transient_response} shows the transient
response with an improved compensation network; $R_1 = 40.2\text{k}\Omega$,
$C_1 = 560$pF and $C_2 = 82$pF. The improved compensation network
values were determined by increasing the compensation mid-band gain
by increasing $R_1$, and then reducing $C_1$ and $C_2$ by a similar
amount ($C_1 = 470$pF and $C_2 = 68$pF produce virtually identical
transient responses).
The transient response in Figure~\ref{fig:LTC3855_ex1_transient_response}(a)
marginally violates the output voltage regulation specification.
The load current increase causes the output voltage to drop to 
916mV (4mV violation), and the load current decrease causes the
output voltage to peak to 978mV (2mV margin).
%
Figure~\ref{fig:LTC3855_ex1_transient_response}(b) shows that voltage
regulation specification can be met by increasing the output capacitance.
The load current increase causes the output voltage to drop to 
925mV (5mV margin), and the load current decrease causes the
output voltage to peak to 977mV (3mV margin).

Figure~\ref{fig:LTC3855_ex2_transient_response} shows the transient
response for the dual-phase controller for two values of output
capacitance. The compensation network for the dual-phase controller
was adjusted until its response was similar to that of the
single-phase controller. The compensation values were changed
to; $R_1 = 20\text{k}\Omega$, $C_1 = 1500$pF and $C_2 = 150$pF.
%
The transient response in Figure~\ref{fig:LTC3855_ex2_transient_response}(a)
marginally violates the output voltage regulation specification.
The load current increase causes the output voltage to drop to 
921mV (1mV margin), and the load current decrease causes the
output voltage to peak to 982mV (2mV violation).
%
Figure~\ref{fig:LTC3855_ex2_transient_response}(b) shows that voltage
regulation specification can be met by increasing the output capacitance.
The load current increase causes the output voltage to drop to 
925mV (5mV margin), and the load current decrease causes the
output voltage to peak to 975mV (3mV margin).

The bulk output capacitance uses Sanyo POSCAP-series capacitors.
The single-phase transient response in
Figure~\ref{fig:LTC3855_ex1_transient_response}(a)
used $4\times470\mu$F 2R5TPF470M6L (2.5V, 6m$\Omega$, 4.4A$_{\rm RMS}$),
while Figure~\ref{fig:LTC3855_ex1_transient_response}(b) used
$4\times1000\mu$F ETPF1000M6H (2.5V, 6m$\Omega$, 5.6A$_{\rm RMS}$).
The dual-phase transient response in
Figure~\ref{fig:LTC3855_ex2_transient_response}(a)
used $8\times470\mu$F 2R5TPF470M6L (2.5V, 6m$\Omega$, 4.4A$_{\rm RMS}$),
while Figure~\ref{fig:LTC3855_ex2_transient_response}(b) used
$8\times1000\mu$F ETPF1000M6H (2.5V, 6m$\Omega$, 5.6A$_{\rm RMS}$).
The Sanyo POSCAP catalog lists parts with slightly lower ESR as
{\em in development}; ETPF470M5H (2.5V, 5m$\Omega$, 6.1A$_{\rm RMS}$),
and ETPF1000M5H (2.5V, 5m$\Omega$, 6.1A$_{\rm RMS}$).

The key design aspect that the transient response analysis has
determined, is that the (single-) dual-phase supply requires
(four) eight POSCAP capacitors to meet the output voltage regulation
specification.

% -----------------------------------------------------------------
% Single-phase transient LTspice circuit
% -----------------------------------------------------------------
%
\begin{landscape}
\setlength{\unitlength}{1mm}
\begin{figure}[p]
  \begin{center}
    \includegraphics[width=210mm]
    {figures/LTC3855_ex1_transient_circuit.pdf}
  \end{center}
  \caption{LTC3855 12V to 0.95V@20A single-phase LTspice transient response analysis circuit.}
  \label{fig:LTC3855_ex1_transient_circuit}
\end{figure}
\end{landscape}

% -----------------------------------------------------------------
% Dual-phase transient LTspice circuit
% -----------------------------------------------------------------
%
\begin{landscape}
\setlength{\unitlength}{1mm}
\begin{figure}[p]
  \begin{center}
    \includegraphics[width=210mm]
    {figures/LTC3855_ex2_transient_circuit.pdf}
  \end{center}
  \caption{LTC3855 12V to 0.95V@40A dual-phase LTspice transient response analysis circuit.}
  \label{fig:LTC3855_ex2_transient_circuit}
\end{figure}
\end{landscape}

% -----------------------------------------------------------------
% Single-phase transient response waveforms
% -----------------------------------------------------------------
%
\setlength{\unitlength}{1mm}
\begin{figure}[p]
  \begin{picture}(155,205)(0,0)
    \put(10,105){
    \includegraphics[width=0.87\textwidth]
    {figures/LTC3855_ex1_transient_response_a.pdf}}
    \put(10,0){
    \includegraphics[width=0.87\textwidth]
    {figures/LTC3855_ex1_transient_response_b.pdf}}
    \put(77,105){(a)}
    \put(77, 0){(b)}
  \end{picture}
  \caption{LTC3855 12V to 0.95V@20A single-phase supply LTspice transient response;
  (a) from $t=0$, and (b) zoomed view from $t=1.1$ms to 2.1ms.}
  \label{fig:LTC3855_ex1_transient_response_power_on}
\end{figure}

% -----------------------------------------------------------------
% Single-phase transients for 470uF and 1000uF
% -----------------------------------------------------------------
%
\setlength{\unitlength}{1mm}
\begin{figure}[p]
  \begin{picture}(155,205)(0,0)
    \put(10,105){
    \includegraphics[width=0.87\textwidth]
    {figures/LTC3855_ex1_transient_response_c.pdf}}
    \put(10,0){
    \includegraphics[width=0.87\textwidth]
    {figures/LTC3855_ex1_transient_response_d.pdf}}
    \put(77,105){(a)}
    \put(77, 0){(b)}
  \end{picture}
  \caption{LTC3855 12V to 0.95V@20A single-phase supply LTspice {\em improved} 
  transient response for output capacitances of;
  (a) $4\times470\mu$F, and (b) $4\times1000\mu$F, both with 6m$\Omega$ ESR.}
  \label{fig:LTC3855_ex1_transient_response}
\end{figure}

% -----------------------------------------------------------------
% Dual-phase transients for 470uF and 1000uF
% -----------------------------------------------------------------
%
\setlength{\unitlength}{1mm}
\begin{figure}[p]
  \begin{picture}(155,205)(0,0)
    \put(10,105){
    \includegraphics[width=0.87\textwidth]
    {figures/LTC3855_ex2_transient_response_a.pdf}}
    \put(10,0){
    \includegraphics[width=0.87\textwidth]
    {figures/LTC3855_ex2_transient_response_b.pdf}}
    \put(77,105){(a)}
    \put(77, 0){(b)}
  \end{picture}
  \caption{LTC3855 12V to 0.95V@40A dual-phase supply LTspice 
  transient response for output capacitances of;
  (a) $8\times470\mu$F, and (b) $8\times1000\mu$F, both with 6m$\Omega$ ESR.}
  \label{fig:LTC3855_ex2_transient_response}
\end{figure}

\clearpage
% -----------------------------------------------------------------
\subsubsection{Frequency Response Analysis}
% -----------------------------------------------------------------

Figures~\ref{fig:LTC3855_ex1_bode_circuit} 
and~\ref{fig:LTC3855_ex2_bode_circuit} show the LTspice circuits
used for analyzing the LTC3855 single-phase and dual-phase Bode
response\footnote{The Bode analysis run time for the
single-phase controller was 14 hours 20 minutes, while the run
time for the dual-phase controller was 26 hours 56 minutes.}.
%
Figures~\ref{fig:LTC3855_ex1_bode_response}
and~\ref{fig:LTC3855_ex2_bode_response} show the Bode responses.
The Bode responses reasonable open-loop bandwidths, i.e., relative
to the switching frequency of $f_{\rm SW} = 750$kHz, the open-loop bandwidth
of 60kHz, is slightly under the target of $f_{\rm SW}/10 = 75$kHz.
The open-loop gain could be increased to 75kHz if more 
compensation gain was acceptable at the switching frequency.

% -----------------------------------------------------------------
\subsubsection{Review and Discussion}
% -----------------------------------------------------------------

The 12V to 0.95A@40A dual-phase controller transient response in
Figure~\ref{fig:LTC3855_ex2_transient_response}(b)
($8 \times 1000\mu$F output capacitance) shows that the design
meets the output voltage regulation specification, and the Bode 
response in Figure~\ref{fig:LTC3855_ex2_bode_response} shows that
the control loop is stable. 
At this point in the design phase, the power supply design would
be accepted, with final adjustments occurring after hardware tests.

\clearpage
% -----------------------------------------------------------------
% LTspice circuit
% -----------------------------------------------------------------
%
\begin{landscape}
\setlength{\unitlength}{1mm}
\begin{figure}[p]
  \begin{center}
    \includegraphics[width=200mm]
    {figures/LTC3855_ex1_bode_circuit.pdf}
  \end{center}
  \caption{LTC3851A 12V to 0.95V@20A single-phase LTspice Bode response analysis circuit.}
  \label{fig:LTC3855_ex1_bode_circuit}
\end{figure}
\end{landscape}

\clearpage
% -----------------------------------------------------------------
% LTspice circuit
% -----------------------------------------------------------------
%
\begin{landscape}
\setlength{\unitlength}{1mm}
\begin{figure}[p]
  \begin{center}
    \includegraphics[width=200mm]
    {figures/LTC3855_ex2_bode_circuit.pdf}
  \end{center}
  \caption{LTC3855 12V to 0.95V@40A dual-phase LTspice Bode response analysis circuit.}
  \label{fig:LTC3855_ex2_bode_circuit}
\end{figure}
\end{landscape}

\clearpage
% -----------------------------------------------------------------
% Bode response
% -----------------------------------------------------------------
%
\setlength{\unitlength}{1mm}
\begin{figure}[p]
  \begin{picture}(155,190)(0,0)
    \put(15,100){
    \includegraphics[width=0.75\textwidth]
    {figures/LTC3855_ex1_bode_response_mag.pdf}}
    \put(15,4){
    \includegraphics[width=0.75\textwidth]
    {figures/LTC3855_ex1_bode_response_phase.pdf}}
    \put(75,  97){(a)}
    \put(75,   0){(b)}
  \end{picture}
  \caption{LTC3855 12V to 0.95V@20A single-phase supply Bode response;
  (a) magnitude, and (b) phase. 
  The open-loop gain has a cross-over frequency of 61kHz with
  84-degrees of phase margin. The compensation network has
  a gain of 10dB at the switching frequency, i.e., the
  compensation signal contains the output voltage ripple 
  amplified by 3.3 times.}
  \label{fig:LTC3855_ex1_bode_response}
\end{figure}

\clearpage
% -----------------------------------------------------------------
% Bode response
% -----------------------------------------------------------------
%
\setlength{\unitlength}{1mm}
\begin{figure}[p]
  \begin{picture}(155,190)(0,0)
    \put(15,100){
    \includegraphics[width=0.75\textwidth]
    {figures/LTC3855_ex2_bode_response_mag.pdf}}
    \put(15,4){
    \includegraphics[width=0.75\textwidth]
    {figures/LTC3855_ex2_bode_response_phase.pdf}}
    \put(75,  97){(a)}
    \put(75,   0){(b)}
  \end{picture}
  \caption{LTC3855 12V to 0.95V@40A dual-phase supply Bode response;
  (a) magnitude, and (b) phase. 
  The open-loop gain has a cross-over frequency of 66kHz with
  85-degrees of phase margin. The compensation network has
  a gain of 11dB at the switching frequency, i.e., the
  compensation signal contains the output voltage ripple 
  amplified by 3.5 times.}
  \label{fig:LTC3855_ex2_bode_response}
\end{figure}



