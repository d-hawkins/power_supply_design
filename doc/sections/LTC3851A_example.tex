% =================================================================
\subsection{LTC3851A: 12V to 1.5V@15A}
% =================================================================

The Linear Technology web page for the LTC1735 indicates that
the device is not recommended for new designs, and recommends
using the LTC3851 controller, which in turn recommends using
the LTC3851A~\cite{Linear_LTC3851A_2010}. The power supply 
design in this section is based on the reference design that
can be downloaded from the
\href{http://www.linear.com/product/LTC3851A}{LTC3851A web page}.

% -----------------------------------------------------------------
\subsubsection{Component Selection}
% -----------------------------------------------------------------

Power supply requirements;
%
\begin{itemize}
\item Input supply voltage, $V_{\rm IN} = 12$V
\item Output supply voltage, $V_{\rm OUT} = 1.5$V
\item Output supply current, $I_{\rm OUT} = 15$A
\item Output supply voltage ripple, $\Delta V_{\rm OUT} = \pm 75$mV ($\pm5$\%)
\item Controller switching frequency, $f_{\rm SW} = 500$kHz
\end{itemize}
%
The controller components are selected based on the data
sheet procedure (see p11~\cite{Linear_LTC3851A_2010});
%
\begin{enumerate}
\item Select the output current-sense resistor.

\begin{equation}
R_{\rm SENSE} = 0.8\cdot\frac{V_{\rm MAX}}{(I_{\rm OUT} + \Delta I_{\rm OUT}/2)} =
0.8\cdot\frac{53\text{mV}}{17.8\text{A}} = 2.38\text{m}\Omega \approx  2\text{m}\Omega 
\end{equation}
%
where $I_{\rm MAX} = I_{\rm OUT} + \Delta I_{\rm OUT}/2 \approx 17.8$A is
calculated shortly.
%
\item Select $R_{\rm FREQ}$ to select the controller operating
frequency.

$R_{\rm FREQ} = 60\text{k}\Omega$ selects a switching frequency of
500kHz (p4~\cite{Linear_LTC3851A_2010}). The LTspice model was confirmed
to generate this switching frequency.

\item Inductor selection.

The {\em nominal} inductor value is determined based on a ripple current of
30\% of the maximum output current, i.e., 
$\Delta I_{\rm OUT} = 0.3I_{\rm OUT} = 5$A. The inductor value 
is then
%
\begin{equation}
L = \frac{V_{\rm OUT}}{f_{\rm SW}\Delta I_{\rm OUT}}\left(
1 - \frac{V_{\rm OUT}}{V_{\rm IN}}\right) =
\frac{1.5}{500\text{k}\times5}\left(
1 - \frac{1.5}{12}\right) = 0.525\mu\text{H} \approx 0.5\mu\text{H}
\end{equation}
%
A review of Vishay/Dale IHLP series inductors showed that the nearest
value is the IHLP-3232DZ-01 $0.47\mu\text{H}$ 
($R_{\rm DCR(MAX)} = 2.62\text{m}\Omega$).
For $L = 0.47\mu$H, $\Delta I_{\rm OUT} = 5.8$A and 
$I_{\rm MAX} = 17.8$A. The IHLP-3232DZ-01 inductor can easily
support a peak load current of 17.8A, resulting in a temperature 
rise of about $20^\circ$C.

The LTC3851A supports current-sensing via either a current-sense
resistor, or via inductor DCR sensing. The DCR of the $0.47\mu\text{H}$
is slightly higher than the target sense resistance of $2\text{m}\Omega$,
so the DCR sense network needs to attenuate the current-sense
voltage by about 0.8. The use of a sense resistor versus the inductor
DCR will be compared during power (efficiency) analysis.

\item Power MOSFET and D1 selection.

The LTspice example downloaded from the LTC3851A web page uses a
Renesas RJK0305DP for the main (top) MOSFET 
($R_{\rm DS(ON)} = 6.7\text{m}\Omega$), and a Renesas RJK0301DP
for the synchronous (bottom) MOSFET ($R_{\rm DS(ON)} = 2.3\text{m}\Omega$).
These components are also used in this example.

Based on the LTC3851A data sheet equations, the power dissipated
in the synchronous MOSFET will be
%
\begin{equation}
P_{\rm SYNC} \approx \left(1 - \frac{V_{\rm OUT}}{V_{\rm IN}}\right)
I_{\rm MAX}^2R_{\rm DS(ON)} = 0.875\times17.8^2\times2.3\text{m} = 0.64\text{W}
\end{equation}
%
and with a channel-to-case thermal resistance of $\theta_{\rm ch-C} = 1.93^\circ$C/W,
the channel temperature rise will be $1.2^\circ$C. The power dissipated in the
main (top) MOSFET will be
%
\begin{equation}
\begin{split}
P_{\rm MAIN} &\approx \frac{V_{\rm OUT}}{V_{\rm IN}}I_{\rm MAX}^2R_{\rm DS(ON)} +
\frac{V_{\rm IN}^2I_{\rm MAX}R_{\rm DR}C_{\rm MILLER}}{2}\cdot
\left[ \frac{1}{V_{\rm INTVCC} - V_{\rm TH(MIN)}} + \frac{1}{V_{\rm TH(MIN)}}\right]\\
&= 0.125\times17.8^2\times6.7\text{m} +  ???\\
&= ???
\end{split}
\end{equation}
%
resulting in a junction temperature rise of ???

{\bf TODO}: fill in these numbers.

LTspice will be used to review these power dissipation estimates.

\item Input capacitance selection.

TODO: how do I calculate the value and ESR? Is it just based on
RMS handling? LTC3851A data sheet p17 has comments on ESR, but
nothing on the amount of capacitance.

\item Output capacitance selection.

The {\em nominal} value of the output capacitance is determined by 
the amount required to absorb the energy in the inductor during a
load step from maximum current to minimum current, i.e.,
%
\begin{equation}
\begin{split}
C_{\rm OUT} &> \frac{LI_{\rm STEP}^2}
{(V_{\rm OUT} + \Delta V_{\rm OUT})^2 - V_{\rm OUT}^2} 
\approx \frac{LI_{\rm STEP}^2}{2V_{\rm OUT}\Delta V_{\rm OUT}}\\
&= \frac{0.47\mu\times 15^2}{2\times1.5\times60\text{m}}\\
&= 588\mu\text{F}
\end{split}
\end{equation}
%
where $I_{\rm STEP}=15$A and $\Delta V_{\rm OUT} = 60$mV was
used, since there will be additional ripple voltage due to 
the output capacitance ESR.

The {\em nominal} value of the output capacitance ESR is determined by
the allowable output voltage dip during a load step from minimum current
to maximum current, i.e.,
%
\begin{equation}
R_{\rm ESR} = \frac{\Delta V_{\rm OUT}}{I_{\rm STEP}} = 
\frac{75\text{m}}{15} \approx 5\text{m}\Omega
\end{equation}
%
The output capacitance requirements can be met using a pair
of 330$\mu$F 2R5TPE330M9 (2.5V, 9m$\Omega$, 3.9A$_{\rm RMS}$)
Sanyo POSCAP TPE-series capacitors.
%
The output voltage dip for a 15A output current increase will
be around 68mV, the output voltage peaking for a 15A output
current decrease will be around 54mV, and the ripple
voltage will be approximately 25mV.

\item Output voltage.

The output voltage is determined by the output feedback resistors
and the controller reference voltage, 
%
\begin{equation}
V_{\rm OUT} = V_{\rm REF}\left(1+\frac{R_{\rm a}}{R_{\rm b}}\right)
\end{equation}
%
where $V_{\rm REF} = 0.8$V, $R_{\rm a}$ is the resistor from the
output voltage to the controller $V_{\rm OSENSE}$ pin, and 
$R_{\rm b}$ is the resistor from the controller $V_{\rm OSENSE}$ 
pin to ground.

The LTC3851A example design used $R_{\rm a} = 43.2\text{k}\Omega$ and 
$R_{\rm b} = 49.9\text{k}\Omega$, resulting in an output
voltage of $V_{\rm OUT} = 1.493$V.
%
The nearest pair of 1-percent resistor values for an output voltage
of 1.5V are actually $R_{\rm a} = 9.31\text{k}\Omega$ and 
$R_{\rm b} = 10.7\text{k}\Omega$, resulting in an output
voltage of $V_{\rm OUT} = 1.496$V.
%
This design retains the same resistor values as the LTC3851A example.

\item Soft-start capacitance.

The soft-start time is 
%
\begin{equation}
t_{\rm SS} = 0.8\text{V}\cdot\frac{C_{\rm SS}}{1.0\mu\text{A}}
\end{equation}
%
A soft-start capacitance of $C_{\rm SS} = 2$nF was selected 
to give a soft-start time of 1.6ms. This results in the 
LTspice simulation having similar power-on timing to the
LTC1735 example.

The voltage on the soft-start capacitor ramps the output voltage
during power on, and thus limits the power-on current.
The ramping of the soft-start voltage and the output voltage
can be viewed in the LTspice simulation by plotting the
soft-start voltage \verb+V(ss)+ and the voltage on the
error amplifier feedback input \verb+V(fb)+.

\item Compensation components.

The LTC3851A example design uses the compensation components
$R_1  = 6.49\text{k}\Omega$, $C_1 = 3300$pF, and
$C_2 = 470$pF. These components create a compensation
network with a zero frequency of
%
\begin{equation}
f_{\rm Z} = \frac{1}{2\pi R_1 C_1} = 7.4\text{kHz}
\end{equation}
%
and a pole frequency of
%
\begin{equation}
f_{\rm P} = \frac{1}{2\pi R_1 C_1C_2/(C_1+C_2)}
= 59.6\text{kHz}
\end{equation}
\end{enumerate}

\clearpage
% -----------------------------------------------------------------
\subsubsection{Transient Response Analysis}
% -----------------------------------------------------------------

Figure~\ref{fig:LTC3851A_ex1_transient_circuit} shows an LTspice
circuit for analyzing the transient response of the
LTC3851A 12V to 1.5V@15A power supply design.
Figure~\ref{fig:LTC3851A_ex1_transient_response} shows the transient
response of several controller waveforms. 
The top trace in Figure~\ref{fig:LTC3851A_ex1_transient_response}
show the output voltage, \verb+V(out)+, the feedback voltage
\verb+V(fb)+, and the soft-start voltage, \verb+V(ss)+. The
feedback voltage tracks the soft-start voltage until the
feedback voltage reaches the reference voltage of 
$V_{\rm REF} = 0.8$V, after which point, the output voltage is
in regulation. During the soft-start power-on time, the
error amplifier (compensation) output voltage, \verb+V(comp)+,
and the inductor current, \verb+I(L)+ are well below their
maximum values. If these signals were observed to be too high,
then the soft-start time would need to be increased.

Figure~\ref{fig:LTC3851A_ex1_transient_response} shows the transient
response for an output current load step from 1A to 15A (a 14A load step),
applied at $t = 2.0$ms and removed at $t = 2.4$ms. 
Figure~\ref{fig:LTC3851A_ex1_transient_response}(b) shows a zoomed view of
the controller response to the load step. The compensation
response, \verb+V(comp)+, has no overshoot or ringing, indicating
the closed-loop response is stable with good phase margin.
The inductor current ripple of $\Delta I_{\rm OUT} = 5.6$A
can be seen in the inductor current waveform, \verb+I(L)+,
in the bottom trace in Figure~\ref{fig:LTC3851A_ex1_transient_response}(b).
The AC component of the inductor current travels through the output
capacitance, and generates the output ripple voltage over the output
capacitance ESR.  This ripple voltage can be seen in \verb+V(out)+, in
the top trace in Figure~\ref{fig:LTC3851A_ex1_transient_response}(b). 
The peak-to-peak output ripple voltage matches the expected value of 
$5.6\text{A}\times4.5\text{m}\Omega=25$mV.
The output voltage has an average value of 1.493V.
The output voltage dip and peaking during the application and removal
of the output current step is about -110mV and +90mV respectively.
The top plot in
Figure~\ref{fig:LTC3851A_ex1_transient_response}(b)
contains traces showing the voltage transient limits at
1.425V and 1.575V. The output current load transient causes
output voltage transients that both violate the power supply
output voltage specification.

The transient response in 
Figure~\ref{fig:LTC3851A_ex1_transient_response}(b) is
control loop limited, i.e., the loop bandwidth needs to be increased
so that the voltage dip transient is output capacitor ESR dominated.
The next section measures the power supply control loop to determine
whether the control loop bandwidth can be increased, while maintaining
adequate phase margin.

% -----------------------------------------------------------------
% LTspice circuit
% -----------------------------------------------------------------
%
\begin{landscape}
\setlength{\unitlength}{1mm}
\begin{figure}[p]
  \begin{center}
    \includegraphics[width=210mm]
    {figures/LTC3851A_ex1_transient_circuit.pdf}
  \end{center}
  \caption{LTC3851A 12V to 1.5V@15A LTspice transient response analysis circuit.}
  \label{fig:LTC3851A_ex1_transient_circuit}
\end{figure}
\end{landscape}

% -----------------------------------------------------------------
% Transient response waveforms
% -----------------------------------------------------------------
%
\setlength{\unitlength}{1mm}
\begin{figure}[p]
  \begin{picture}(155,205)(0,0)
    \put(10,105){
    \includegraphics[width=0.87\textwidth]
    {figures/LTC3851A_ex1_transient_response_a.pdf}}
    \put(10,0){
    \includegraphics[width=0.87\textwidth]
    {figures/LTC3851A_ex1_transient_response_b.pdf}}
    \put(77,105){(a)}
    \put(77, 0){(b)}
  \end{picture}
  \caption{LTC3851A 12V to 1.5V@15A supply LTspice transient response;
  (a) from $t=0$, and (b) zoomed view from $t=1.7$ms to 2.7ms.}
  \label{fig:LTC3851A_ex1_transient_response}
\end{figure}

\clearpage
% -----------------------------------------------------------------
\subsubsection{Frequency Response Analysis}
% -----------------------------------------------------------------

Figure~\ref{fig:LTC3851A_ex1_bode_circuit} shows an LTspice
circuit for analyzing the Bode response of the
LTC3851A 12V to 1.5V@15A power supply design.
Figure~\ref{fig:LTC3851A_ex1_bode_response} shows the Bode 
response.
The Bode response shows that the loop gain has a
cross-over frequency of 41kHz with 69-degrees of phase margin.
This is consistent with the transient response observed in
Figure~\ref{fig:LTC3851A_ex1_transient_response}. 
The problem with the transient response though, is that it
does not meet the output voltage regulation requirements.
The Bode magnitude plot in
Figure~\ref{fig:LTC3851A_ex1_bode_response}(a) shows that the
compensation response has less than 0dB gain at the switching
frequency, indicating that there is some room for 
compensation network adjustment.

Table~\ref{fig:LTC3851A_ex1_compensation} shows the results of 
{\em designing} a compensation network using a MATLAB script.
The MATLAB script imports the control-to-output gain measured 
using LTspice, and then allows the user to adjust the compensation
response by specifying the desired cross-over frequency and
zero and pole locations. The script then calculates the nearest
1-percent resistor value and standard capacitor values.
Table~\ref{fig:LTC3851A_ex1_compensation} shows the compensation
parameters for the circuit in 
Figure~\ref{fig:LTC3851A_ex1_bode_response} and an alternative
set of parameters.

To improve the transient response of the power supply, the 
circuit in Figure~\ref{fig:LTC3851A_ex1_bode_circuit} was modified;
%
\begin{itemize}
\item The feedback resistors were changed to 
$R_{\rm a} = 1.65\text{k}\Omega$ and 
$R_{\rm b} = 1.87\text{k}\Omega$. This results in a slightly higher
output voltage of $V_{\rm OUT} = 1.506$V. This change helps 
make the transient response symmetric.
\item The compensation components were changed per 
Table~\ref{fig:LTC3851A_ex1_compensation} design number \#2.
\item A 100$\mu$F ceramic output capacitor was added
(TDK C5750X5R0J107M X5R $R_{\rm ESR} = 2\text{m}\Omega$)~\footnote{LTspice
has an internal database of capacitor manufacturer parts that is
accessed by right-clicking on the capacitor schematic symbol, and
then clicking on the {\em Select Capacitor} button.}.
\end{itemize}
%
The transient response of the modified circuit is shown in
Figure~\ref{fig:LTC3851A_ex2_transient_response}, while 
Figure~\ref{fig:LTC3851A_ex2_bode_response} shows the Bode response.
A comparison of the original transient response with the new
shows that the new response reacts much quicker to the load step
due to the increase in loop bandwidth and in the compensation
zero frequency, but has slightly more ringing in the response
due to the slightly lower phase-margin.

The modified power supply design has an improved transient response that
marginally violates the output voltage specification for the design.
At this point, the performance could be accepted subject to hardware
testing, or additional simulation could be performed. The next
section discussions some of the options.

\clearpage
% -----------------------------------------------------------------
% LTspice circuit
% -----------------------------------------------------------------
%
\begin{landscape}
\setlength{\unitlength}{1mm}
\begin{figure}[p]
  \begin{center}
    \includegraphics[width=200mm]
    {figures/LTC3851A_ex1_bode_circuit.pdf}
  \end{center}
  \caption{LTC3851A 12V to 1.5V@15A LTspice Bode response analysis circuit.}
  \label{fig:LTC3851A_ex1_bode_circuit}
\end{figure}
\end{landscape}

\clearpage
% -----------------------------------------------------------------
% Bode response
% -----------------------------------------------------------------
%
\setlength{\unitlength}{1mm}
\begin{figure}[p]
  \begin{picture}(155,190)(0,0)
    \put(15,100){
    \includegraphics[width=0.75\textwidth]
    {figures/LTC3851A_ex1_bode_response_mag.pdf}}
    \put(15,4){
    \includegraphics[width=0.75\textwidth]
    {figures/LTC3851A_ex1_bode_response_phase.pdf}}
    \put(75,  97){(a)}
    \put(75,   0){(b)}
  \end{picture}
  \caption{LTC3851A 12V to 1.5V@15A supply Bode response;
  (a) magnitude, and (b) phase. The LC and compensation
  {\em calculated} responses are shown by the black and red
  solid lines, while the LTspice {\em simulated} responses are
  shown by the black and red crosses. The 
  control-to-output and open-loop LTspice {\em simulated} responses
  are shown using both solid lines and crosses.
  The open-loop gain has a cross-over frequency of 41kHz with
  69-degrees of phase margin.}
  \label{fig:LTC3851A_ex1_bode_response}
\end{figure}

\clearpage
% -----------------------------------------------------------------
% Compensation design
% -----------------------------------------------------------------
%
\begin{table}[p]
\caption{LTC3851A 12V to 1.5V@15A supply compensation design.}
\label{fig:LTC3851A_ex1_compensation}
\begin{center}
\begin{tabular}{|l|c|c|c|c|}
\hline
\rule{0cm}{4mm}Component or Parameter & \multicolumn{4}{c|}{Compensation Design Number}\\
\cline{2-5}
\rule{0cm}{4mm}     & \#1 & \#2 (3,4) & \#5 & \#6\\
\hline
\hline
\multicolumn{5}{|l|}{\bf Compensation response}\\
\hline
& \hspace {20mm} &  \hspace {20mm} &  \hspace {20mm} &  \hspace {20mm} \\
$R_1$       & 6.49k$\Omega$ &  7.5k$\Omega$ &  7.15k$\Omega$ &  7.15k$\Omega$\\
$C_1$       & 3300pF        &  680pF        &  560pF         &  2700pF       \\
$C_2$       &  470pF        &  330pF        &  180pF         &  180pF        \\
&&&&\\
$f_{\rm Z}$ &  7.4kHz       & 31.2kHz       & 39.8kHz        &  8.24kHz       \\
$f_{\rm P}$ & 59.6kHz       & 95.5kHz       & 163.4kHz       & 131.9kHz      \\
&&&&\\
Compensation gain at $f_{\rm SW}$ & -2.84dB & 0.07dB & 5.06dB & 5.19dB\\
&&&&\\
\hline
\multicolumn{5}{|l|}{\bf Open-loop response}\\
\hline
&&&&\\
Cross-over frequency    & 41.6kHz      & 49.4kHz      & 71.6kHz      & 76.1kHz      \\ 
Cross-over phase-margin & 69.9$^\circ$ & 57.0$^\circ$ & 56.2$^\circ$ & 72.0$^\circ$ \\
&&&&\\
\hline
\end{tabular}
\end{center}
\end{table}

% -----------------------------------------------------------------
% Transient response (Example#2)
% -----------------------------------------------------------------
%
\setlength{\unitlength}{1mm}
\begin{figure}[p]
  \begin{center}
    \includegraphics[width=0.87\textwidth]
    {figures/LTC3851A_ex2_transient_response.pdf}
  \end{center}
  \caption{LTC3851A 12V to 1.5V@15A supply transient response for design \#2.
  The circuit modifications were; a slight increase in the output voltage,
  a new compensation network, and a ceramic output capacitor was added.
  Compare the response in this figure to that in
  Figure~\ref{fig:LTC3851A_ex1_transient_response}(b).
  Figure~\ref{fig:LTC3851A_ex2_bode_response} shows the Bode response.}
  \label{fig:LTC3851A_ex2_transient_response}
\end{figure}

\clearpage
% -----------------------------------------------------------------
% Bode response (Example#2)
% -----------------------------------------------------------------
%
\setlength{\unitlength}{1mm}
\begin{figure}[p]
  \begin{picture}(155,190)(0,0)
    \put(15,100){
    \includegraphics[width=0.75\textwidth]
    {figures/LTC3851A_ex2_bode_response_mag.pdf}}
    \put(15,4){
    \includegraphics[width=0.75\textwidth]
    {figures/LTC3851A_ex2_bode_response_phase.pdf}}
    \put(75,  97){(a)}
    \put(75,   0){(b)}
  \end{picture}
  \caption{LTC3851A 12V to 1.5V@15A supply Bode response for design \#2;
  (a) magnitude, and (b) phase. The modified circuit added additional
  output capacitance, which causes the control-to-output response to
  move lower in frequency. The open-loop gain has a cross-over frequency
  of 43kHz with 48-degrees of phase margin. The slight ringing seen in the
  transient response is a reflection of the lower phase-margin.}
  \label{fig:LTC3851A_ex2_bode_response}
\end{figure}

\clearpage
% -----------------------------------------------------------------
\subsubsection{Control Loop Optimization}
% -----------------------------------------------------------------

The transient response in Figure~\ref{fig:LTC3851A_ex2_transient_response}
marginally violates the voltage specification of the design.
This violation can be avoided by increasing the loop bandwidth.
The loop bandwidth is increased by first modifying the current-loop
gain, and then adjusting the compensation network.

The compensation voltage waveform in Figure~\ref{fig:LTC3851A_ex2_transient_response},
\verb+V(comp)+, peaks at slightly over 1.4V. The LTC3851A data sheet
{\em Maximum Peak Current Sense Threshold vs $I_{\rm TH}$ Voltage} plot 
(p6~\cite{Linear_LTC3851A_2010}) shows that 1.4V is close to the limit
of the linear region of the error amplifier output.
The compensation voltage response can be reduced by reducing the current feedback.
Reducing the current feedback causes a slight increase
in the control-to-output gain, due to the lowered current feedback,
causing a slight increase in the open-loop gain cross-over frequency.
The LTC3851A has two options for adjusting the current feedback;
the $I_{\rm LIM}$ pin and the current sense resistance. These two
options can be considered coarse adjustment and fine adjustment.

Coarse adjustment of the current feedback can be controlled using
the LTC3851A $I_{\rm LIM}$ pin; 0V, floating, and $\text{INTV}_{\rm CC}$
select the {\em nominal} current-sense limits of 30mV, 53mV, and 80mV. 
These correspond to high, medium, and low current gain settings.
The effect of the $I_{\rm LIM}$ pin can be understood by considering
the power supply just analyzed. The supply was designed for the
53mV current-sense range ($I_{\rm LIM}$ floating), resulting in the
selection of a 2m$\Omega$ current-sense resistor. If the
80mV current-sense range ($I_{\rm LIM}$ connected to $\text{INTV}_{\rm CC}$)
is now selected {\em without} changing the sense resistor, then the
compensation voltage waveform will be scaled by $53\text{mV}/80\text{mV} = 0.663$,
i.e., the current-sense gain has been lowered.

Fine adjustment of the current feedback by adjusting the
current-sense resistance, is not very practical in designs that use an
actual {\em resistor} for current-sensing, due to the fact that the
low-ohmic value resistors come in limited increments. Current-sensing using
inductor DC resistance (DCR) sensing is much more flexible, as the RC-sense network
can also include attenuation control, providing the flexibility
to scale the current-sense gain arbitrarily. 
%
For example, the power supply just analyzed can be converted to use 
inductor DCR sensing. The design of a DCR sensing network is discussed
on pages 12 and 13 of the LTC3851A data 
sheet~\cite{Linear_LTC3851A_2010}. Defining $\alpha$ as the required
current-sense attenuation, and using the reference designators
from Figure 2 in the data sheet, the component values are calculated
via;
%
\begin{equation}
\begin{split}
\alpha &= \frac{R_{\rm SNS}}{R_{\rm DCR}}\\
C_1 &= 100\text{nF}\\
R_1 &= \frac{L}{\alpha R_{\rm DCR}C_1}\\
R_2 &= R_1\cdot\frac{\alpha}{(1-\alpha)}
\end{split}
\end{equation}
%
where $C_1$ is arbitrarily selected. Table~\ref{tab:LTC3851A_ex3_dcr_sensing}
shows the components required to implement an effective current-sense
value of 2.0m$\Omega$ as used in the original supply design, and the component
values needed for an effective current-sense resistance of 1.5m$\Omega$.
The resistance values in the table have been converted to the
nearest 1-percent values. 

\begin{table}[t]
\caption{LTC3851A 12V to 1.5V@15A supply inductor DCR current-sense components.}
\label{tab:LTC3851A_ex3_dcr_sensing}
\begin{center}
\begin{tabular}{|c|c|c|}
\hline
\rule{0cm}{4mm}Parameter or & \multicolumn{2}{c|}{Nominal Sense Resistance} \\
\cline{2-3}
\rule{0cm}{4mm}Component    & $2.0\text{m}\Omega$ & $1.5\text{m}\Omega$ \\
\hline\hline
         &\hspace {20mm} &\hspace {20mm}\\
$\alpha$ & 0.769         & 0.577         \\
$C_1$    & 100nF         & 100nF         \\
$R_1$    & 2.37k$\Omega$ & 3.16k$\Omega$ \\
$R_2$    & 7.87k$\Omega$ & 4.32k$\Omega$ \\
&&\\
\hline
\end{tabular}
\end{center}
\end{table}

Figures~\ref{fig:LTC3851A_ex3_transient_response} 
and~\ref{fig:LTC3851A_ex3_bode_response} show the transient and Bode
response for the $R_{\rm SNS} = 2.0\text{m}\Omega$ design.
A comparison between the transient responses in
Figures~\ref{fig:LTC3851A_ex2_transient_response} 
and~\ref{fig:LTC3851A_ex3_transient_response} 
shows the transient responses are virtually indistinguishable,
i.e., DCR current-sensing performs identically to resistive
current-sensing.
A comparison between the Bode responses in
Figures~\ref{fig:LTC3851A_ex2_bode_response} 
and~\ref{fig:LTC3851A_ex3_bode_response} shows that the
control-to-output response differs in amplitude and phase 
at low-frequencies. The DCR sensing design has higher gain near 
DC, which will result in better output regulation.

Figure~\ref{fig:LTC3851A_ex4_transient_response} 
and~\ref{fig:LTC3851A_ex4_bode_response} show the transient and Bode
response for the $R_{\rm SNS} = 1.5\text{m}\Omega$ design.
A comparison between the transient responses in
Figures~\ref{fig:LTC3851A_ex3_transient_response} 
and~\ref{fig:LTC3851A_ex4_transient_response} shows a slightly
faster transient in the $R_{\rm SNS} = 1.5\text{m}\Omega$ design,
due to a slight increase in the open-loop cross-over frequency.
A comparison between the Bode responses in
Figures~\ref{fig:LTC3851A_ex3_bode_response} 
and~\ref{fig:LTC3851A_ex4_bode_response} shows how the
control-to-output response shifts up slightly in the
$1.5\text{m}\Omega$ design.

The $R_{\rm SNS} = 1.5\text{m}\Omega$ design still does not meet
the output voltage regulation specification.
The compensation network used to produce the transient responses in
Figures~\ref{fig:LTC3851A_ex2_transient_response},
~\ref{fig:LTC3851A_ex3_transient_response},
and~\ref{fig:LTC3851A_ex4_transient_response}
was designed so that the compensation gain at the switching-frequency
was 0dB. This feature can be seen in the Bode responses in
Figures~\ref{fig:LTC3851A_ex2_bode_response}(a),
~\ref{fig:LTC3851A_ex3_bode_response}(a),
and~\ref{fig:LTC3851A_ex4_bode_response}(a).
Since the output voltage ripple is low (due to the use of the
additional ceramic output capacitor), this requirement can be
relaxed, allowing the open-loop bandwidth to be increased.
Figures~\ref{fig:LTC3851A_ex5_transient_response}
and~\ref{fig:LTC3851A_ex5_bode_response} show the transient
and Bode response for a compensation network design that
allows gain at the switching frequency (see 
Table~\ref{fig:LTC3851A_ex1_compensation} design \#5 for the
component values). The transient response
now meets the output voltage regulation requirement.

Figures~\ref{fig:LTC3851A_ex6_transient_response}
and~\ref{fig:LTC3851A_ex6_bode_response} show the transient
and Bode response for design \#6, the {\em final} supply design.
Table~\ref{fig:LTC3851A_ex1_compensation} shows that relative to
design \#5, the value for $C_1$ was increased to 2700pF,
causing the compensation zero to move to lower frequency.
This change was made to show that it was the {\em increase
in open-loop cross-over frequency} that led to the design
meeting the output voltage regulation specification.
The location of the low-frequency zero determines the responsiveness
of the controller after the initial transient. You can see
the effect of moving the zero by comparing the transient
responses in Figures~\ref{fig:LTC3851A_ex5_transient_response}
and~\ref{fig:LTC3851A_ex6_transient_response}; note how the
time taken for the output voltage to settle back to the nominal
output voltage takes longer in design \#6 than in design \#5.
Moving the zero to lower frequency provides additional phase
margin, so the transient response of the compensation voltage
and output current in 
Figure~\ref{fig:LTC3851A_ex6_transient_response}
have virtually no ringing.

The 1ms transient responses shown in the figures in this section simulate
very quickly in LTspice ($\sim$1 minute). The Bode responses are
constructed from multiple transient simulation runs; each of the 19 
points in the Bode plot is a transient simulation of 10 periods of the
sinusoid stimulus or 1ms, whichever time is greater. The lowest frequencies
of 100Hz and 300Hz dominate the simulation time.
The typical run time of the Bode analysis on an Intel Core i7 Q820 
1.73GHz quad-core machine with 16GB of RAM was in excess of 4 hours,
i.e., the Bode plots for the 6 designs required over 24 hours of 
simulation time. The Bode plots are however critical to the 
{\em design} of the compensation network, so the simulation time
cannot be avoided (just try to avoid too many iterations!).

The subtle changes in the transient and Bode figures is best
viewed using the electronic version of this document. The figures
have all been scaled and located on the page identically, so that
you can see the change in response when changing pages.

% -----------------------------------------------------------------
% Transient response (Example#3)
% -----------------------------------------------------------------
%
\setlength{\unitlength}{1mm}
\begin{figure}[p]
  \begin{center}
    \includegraphics[width=0.87\textwidth]
    {figures/LTC3851A_ex3_transient_response.pdf}
  \end{center}
  \caption{LTC3851A 12V to 1.5V@15A supply transient response for design \#3
  (inductor DCR sensing, $R_{\rm SNS} = 2.0\text{m}\Omega$).
  The response is virtually indistinguishable from the sense resistor
  version in Figure~\ref{fig:LTC3851A_ex2_transient_response}.\newline}
  \label{fig:LTC3851A_ex3_transient_response}
\end{figure}

% -----------------------------------------------------------------
% Transient response (Example#4)
% -----------------------------------------------------------------
%
\setlength{\unitlength}{1mm}
\begin{figure}[p]
  \begin{center}
    \includegraphics[width=0.87\textwidth]
    {figures/LTC3851A_ex4_transient_response.pdf}
  \end{center}
  \caption{LTC3851A 12V to 1.5V@15A supply transient response for design \#4
  (inductor DCR sensing, $R_{\rm SNS} = 1.5\text{m}\Omega$).
  The smaller effective current-sense resistance reduces the response
  of {\tt V(comp)}, relative to that in
  Figure~\ref{fig:LTC3851A_ex3_transient_response}. The transient response
  is now slightly faster due to the increase in open-loop bandwidth,
  however, the transient extrema still exceed the specification.}
  \label{fig:LTC3851A_ex4_transient_response}
\end{figure}

% -----------------------------------------------------------------
% Transient response (Example#5)
% -----------------------------------------------------------------
%
\setlength{\unitlength}{1mm}
\begin{figure}[p]
  \begin{center}
    \includegraphics[width=0.87\textwidth]
    {figures/LTC3851A_ex5_transient_response.pdf}
  \end{center}
  \caption{LTC3851A 12V to 1.5V@15A supply transient response for design \#5
  (inductor DCR sensing, $R_{\rm SNS} = 1.5\text{m}\Omega$, with modified
   compensation network). The modified compensation network moves the
   low frequency zero in the compensation response higher, and the
   cross-over frequency higher. The transient response now meets
   the output voltage regulation specification.}
  \label{fig:LTC3851A_ex5_transient_response}
\end{figure}

% -----------------------------------------------------------------
% Transient response (Example#6)
% -----------------------------------------------------------------
%
\setlength{\unitlength}{1mm}
\begin{figure}[p]
  \begin{center}
    \includegraphics[width=0.87\textwidth]
    {figures/LTC3851A_ex6_transient_response.pdf}
  \end{center}
  \caption{LTC3851A 12V to 1.5V@15A supply transient response for design \#6
  (the {\em final} design). The zero in the compensation network was moved
   lower, resulting in a slightly longer settling time after each transient,
   but less ringing (more phase-margin), while still meeting the output voltage 
   regulation specification.}
  \label{fig:LTC3851A_ex6_transient_response}
\end{figure}

\clearpage
% -----------------------------------------------------------------
% Bode response (Example#3)
% -----------------------------------------------------------------
%
\setlength{\unitlength}{1mm}
\begin{figure}[p]
  \begin{picture}(155,190)(0,0)
    \put(15,100){
    \includegraphics[width=0.75\textwidth]
    {figures/LTC3851A_ex3_bode_response_mag.pdf}}
    \put(15,4){
    \includegraphics[width=0.75\textwidth]
    {figures/LTC3851A_ex3_bode_response_phase.pdf}}
    \put(75,  97){(a)}
    \put(75,   0){(b)}
  \end{picture}
  \caption{LTC3851A 12V to 1.5V@15A supply Bode response for design \#3
  (inductor DCR sensing, $R_{\rm SNS} = 2.0\text{m}\Omega$);
  (a) magnitude, and (b) phase.
  The open-loop gain has a cross-over frequency of 43kHz with
  48-degrees of phase margin.\newline\newline}
  \label{fig:LTC3851A_ex3_bode_response}
\end{figure}

\clearpage
% -----------------------------------------------------------------
% Bode response (Example#4)
% -----------------------------------------------------------------
%
\setlength{\unitlength}{1mm}
\begin{figure}[p]
  \begin{picture}(155,190)(0,0)
    \put(15,100){
    \includegraphics[width=0.75\textwidth]
    {figures/LTC3851A_ex4_bode_response_mag.pdf}}
    \put(15,4){
    \includegraphics[width=0.75\textwidth]
    {figures/LTC3851A_ex4_bode_response_phase.pdf}}
    \put(75,  97){(a)}
    \put(75,   0){(b)}
  \end{picture}
  \caption{LTC3851A 12V to 1.5V@15A supply Bode response for design \#4
  (inductor DCR sensing, $R_{\rm SNS} = 1.5\text{m}\Omega$);
  (a) magnitude, and (b) phase.
  The open-loop gain has a cross-over frequency of 55kHz with
  49-degrees of phase margin.\newline\newline}
  \label{fig:LTC3851A_ex4_bode_response}
\end{figure}

\clearpage
% -----------------------------------------------------------------
% Bode response (Example#5)
% -----------------------------------------------------------------
%
\setlength{\unitlength}{1mm}
\begin{figure}[p]
  \begin{picture}(155,190)(0,0)
    \put(15,100){
    \includegraphics[width=0.75\textwidth]
    {figures/LTC3851A_ex5_bode_response_mag.pdf}}
    \put(15,4){
    \includegraphics[width=0.75\textwidth]
    {figures/LTC3851A_ex5_bode_response_phase.pdf}}
    \put(75,  97){(a)}
    \put(75,   0){(b)}
  \end{picture}
  \caption{LTC3851A 12V to 1.5V@15A supply Bode response for design \#5
  (inductor DCR sensing, $R_{\rm SNS} = 1.5\text{m}\Omega$, compensation
   gain at the switching-frequency);
  (a) magnitude, and (b) phase.
  The open-loop gain has a cross-over frequency of 71kHz with
  56-degrees of phase margin.\newline\newline}
  \label{fig:LTC3851A_ex5_bode_response}
\end{figure}

\clearpage
% -----------------------------------------------------------------
% Bode response (Example#6)
% -----------------------------------------------------------------
%
\setlength{\unitlength}{1mm}
\begin{figure}[p]
  \begin{picture}(155,190)(0,0)
    \put(15,100){
    \includegraphics[width=0.75\textwidth]
    {figures/LTC3851A_ex6_bode_response_mag.pdf}}
    \put(15,4){
    \includegraphics[width=0.75\textwidth]
    {figures/LTC3851A_ex6_bode_response_phase.pdf}}
    \put(75,  97){(a)}
    \put(75,   0){(b)}
  \end{picture}
  \caption{LTC3851A 12V to 1.5V@15A supply Bode response for design \#6
  (inductor DCR sensing, $R_{\rm SNS} = 1.5\text{m}\Omega$, compensation
   gain at the switching-frequency, compensation zero moved to $\sim$10kHz);
  (a) magnitude, and (b) phase.
  The open-loop gain has a cross-over frequency of 76kHz with
  72-degrees of phase margin.\newline\newline}
  \label{fig:LTC3851A_ex6_bode_response}
\end{figure}

\clearpage
% -----------------------------------------------------------------
\subsubsection{Review and Discussion}
% -----------------------------------------------------------------

\noindent{\bf TODO}:
\begin{itemize}
\item What other design changes could be looked at? Go back to
Rdcr = 2.0mOhm, so that the compensation voltage was close to
full-scale, and then redo the compensation components to have
a high cross-over, with say 6dB of gain at the switching frequency. 
\item Sense resistor versus inductor DCR sensing
\item NTC inductor temperature compensation.
\item Review the power dissipation in the MOSFETs.
\item Can I get an efficiency report?
\item Add a comment about output loads based on current sinks rather
than resistive loads, per the comments in~\cite{Linear_DC247_1999}
\item Find ESR of ceramic caps and add a couple on the output.
\end{itemize}

