% =================================================================
\subsection{LTC $\mu$Modules}
% =================================================================


The Linear Technology product guide {\em Power management solutions
for Altera's FPGAs, CPLDs and Structured 
ASICs}~\cite{Linear_Altera_Product_Guide_2012} contains
Linear Technology power supply recommendations for Altera devices
based on input supply voltage, output supply voltage, and output current.
The LTM4601A, LTM4611, LTM4628, and LTM4627 $\mu$Modules are recommended
for many of the high-current applications. 
%
The Linear Technology web site~\cite{Linear_Altera_Reference_Designs_2012}, and
the {\em Altera Development Tools Guide}~\cite{Arrow_Altera_Devtool_Guide_2010}
from Arrow Electronics, contain a list Altera development kits along with the
Linear Technology devices used on each kit.

The LTM4601A $\mu$Module is an integrated module containing a
controller, MOSFETs, inductor, output capacitance, and 
compensation~\cite{Linear_LTM4601A_2011}.
The data sheet indicates that to use the module, the user
{\em simply} determines the output voltage using a resistor, and
adds input capacitance and output capacitance. The following
examples demonstrate that in practice things are not that
simple, due mainly to the fact that the compensation network
is not user-controlled.

% =================================================================
\subsubsection{LTM4601A: 12V to 1.5V@12A (single-phase)}
% =================================================================

For a $V_{\rm IN} = 12$V to $V_{\rm OUT} = 1.5\text{V} \pm 75$mV at 
$I_{\rm OUT} = 12$A single-phase power supply design, the
{\em nominal} output ripple current of the LTM4601A is
%
\begin{equation}
\Delta I_{\rm OUT} = \frac{V_{\rm OUT}}{f_{\rm SW}L}\left(
1 - \frac{V_{\rm OUT}}{V_{\rm IN}}\right) =
\frac{1.5}{850\text{k}\times0.47\mu}\left(
1 - \frac{1.5}{12}\right) = 3.3\text{A}
\end{equation}
%
where the nominal switching frequency, $f_{\rm SW} = 850$kHz, and
the internal inductor value, $L = 0.47\mu$F.

The {\em nominal} total output capacitance requirement is
%
\begin{equation}
\begin{split}
C_{\rm OUT} &> \frac{LI_{\rm STEP}^2}{2V_{\rm OUT}\Delta V_{\rm OUT}}\\
&= \frac{0.47\mu\times 11^2}{2\times1.5\times65\text{m}}\\
&= 292\mu\text{F}
\end{split}
\end{equation}
%
%
where $I_{\rm STEP}=11$A and $\Delta V_{\rm OUT} = 65$mV was
used, since there will be additional ripple voltage due to 
the output capacitance ESR.

The {\em nominal} value of the output capacitance ESR is determined by
the allowable output voltage dip during a load step from minimum current
to maximum current, i.e.,
%
\begin{equation}
R_{\rm ESR} = \frac{\Delta V_{\rm OUT}}{I_{\rm STEP}} = 
\frac{65\text{m}}{11} = 5.9\text{m}\Omega
\end{equation}
%
The output capacitance requirements can be met using a
Sanyo POSCAP TPLF-series
330$\mu$F ETPLF330M6 (2.5V, 6m$\Omega$, 4.7A$_{\rm RMS}$)
or TPF-series
330$\mu$F 2TPF330M6 (2.0V, 6m$\Omega$, 4.4A$_{\rm RMS}$)
or 2R5TPF330M7L (2.5V, 7m$\Omega$, 4.4A$_{\rm RMS}$).
A 100$\mu$F ceramic can be used in parallel with the POSCAP.
%
The output voltage dip (peaking) for a 11A output current increase
(decrease) will be about 66mV (44mV). The ripple voltage will
be $6\text{m}\Omega \times 3.3\text{A} = 20$mV.

Figure~\ref{fig:LTM4601A_ex1_transient_circuit} shows the LTspice
circuit used for transient analysis of the single-phase supply.
Figure~\ref{fig:LTM4601A_ex1_transient_response} shows the
transient response.
%
The transient response in 
Figure~\ref{fig:LTM4601A_ex1_transient_response}(a) fails to meet
the output voltage regulation of $\pm75$mV; the response dips to
1.373V (52mV violation) and peaks at 1.625V (50mV) violation.
Figure~\ref{fig:LTM4601A_ex1_transient_response}(b) shows that
the output voltage regulation requirement can almost be met
by adding a feed-forward capacitor of 330pF;  the response dips to
1.419V (6mV violation) and peaks at 1.579V (4mV) violation.

The feed-forward capacitor reacts with the internal 
$R_{\rm INT} = 60.4\text{k}\Omega$
and $R_{\rm SET}$ resistors in the output voltage-to-error 
amplifier feedback path. The addition of the feed-forward
capacitor converts the feedback path from a simple attenuator
%
\begin{equation}
\alpha_{\rm DC} = \frac{V_{\rm REF}}{V_{\rm OUT}} =
\frac{R_{\rm SET}}{R_{\rm INT} + R_{\rm SET}}
\end{equation}
%
to a frequency dependent network
%
\begin{equation}
\begin{split}
\alpha(s) &= 
\frac{R_{\rm SET}}{R_{\rm INT} + R_{\rm SET}}\cdot
\frac{1+sR_{\rm INT}C_{\rm FF}}
{1+s(R_{\rm INT}||R_{\rm SET})C_{\rm FF}}\\
&=
\frac{R_{\rm SET}}{R_{\rm INT} + R_{\rm SET}}\cdot
\frac{1+sR_{\rm INT}C_{\rm FF}}
{1+sR_{\rm INT}C_{\rm FF}V_{\rm REF}/V_{\rm OUT}}
\end{split}
\end{equation}
%
where use was made of the parallel resistor ratio
%
\begin{equation}
R_{\rm INT}||R_{\rm SET} = \frac{R_{\rm INT}R_{\rm SET}}{R_{\rm INT}+R_{\rm SET}} 
= R_{\rm INT}\cdot\frac{V_{\rm REF}}{V_{\rm OUT}}
\end{equation}
%
and the $\alpha_{\rm DC}$ output voltage relationship.
%
The frequency dependent feedback network has zero and pole frequencies
%
\begin{equation}
\begin{split}
f_{\rm Z} &= \frac{1}{2\pi R_{\rm INT}C_{\rm FF}}\\
f_{\rm P} &= \frac{1}{2\pi R_{\rm INT}C_{\rm FF}V_{\rm REF}/V_{\rm OUT}}
= f_{\rm Z}\cdot\frac{V_{\rm OUT}}{V_{\rm REF}}
\end{split}
\end{equation}
%
where since $V_{\rm OUT} > V_{\rm REF}$, the zero precedes the
pole\footnote{The feed-forward capacitor location in
Figure~\ref{fig:LTM4601A_ex1_transient_circuit} is as shown in 
the LTM4601A data sheet. This location is however incorrect.
The feed-forward capacitor should be placed between the {\tt VOUT\_LCL} and
{\tt FB} pins so that it is placed across the 60.4k$\Omega$
resistor. The location recommended by Linear effectively places the
capacitor across the positive input to the voltage sense amplifier,
{\tt VOSNS+}, and the {\tt FB} pin (error amplifier negative input). 
This is not ideal, as it will cause an imbalance in the differential
input impedance of the voltage sense amplifier, since
{\tt VOSNS+} and {\tt VOSNS-} are no longer impedance matched.}.
%
The addition of a feed-forward capacitor to adjust the compensation network
is {\em not recommended}, as it increases the compensation gain {\em at
all frequencies above the zero}. This results in an increase in the
compensation gain at the switching frequency, which can lead to
control loop instability.

Figure~\ref{fig:LTM4601A_ex1_bode_circuit} shows the LTspice
circuit used for Bode response analysis of the single-phase 
supply.
Figure~\ref{fig:LTM4601A_ex1a_bode_response} shows the
Bode response\footnote{The Bode analysis time
was 1 hour and 30 minutes.} with no feed-forward capacitor
(actually $C_{\rm FF} = 1$fF (femto-farad)).
The Bode response in Figure~\ref{fig:LTM4601A_ex1a_bode_response}
shows that the LTM4601A controller has an open-loop cross-over
frequency of only 23kHz, which is four times lower than the
target cross-over frequency of $f_{\rm SW}/10 = 85$kHz. 
The solid red-line in Figure~\ref{fig:LTM4601A_ex1a_bode_response}
represents a best-guess of the error amplifier gain and
compensation components (the slight deviation in phase at
higher frequencies is most likely due to the voltage-sense
amplifier). Because the user has no access to the compensation
network, there is no way to adjust the mid-band gain, to improve
the open-loop cross-over frequency. The only option available
to the user, is the feed-forward capacitor.

The purpose of the feed-forward capacitor is to increase the
open-loop cross-over frequency. The feed-forward capacitor can
only do this, if the feed-forward zero is located below the
nominal cross-over frequency. The maximum increase in
cross-over frequency occurs if both the zero and pole are
place below the nominal cross-over frequency.
For the Bode response in 
Figure~\ref{fig:LTM4601A_ex1a_bode_response}, the zero should
be located below $V_{\rm REF}/V_{\rm OUT}$ of the nominal cross-over
frequency, i.e., below $0.6/1.5\times23\text{kHz} = 9$kHz.
A feed-forward capacitor of $C_{\rm FF} = 330$pF places the
feed-forward zero and pole at 8kHz and 20kHz.
With this arrangement, the closed-loop gain at the nominal
cross-over frequency will increase by $V_{\rm OUT}/V_{\rm REF} = 2.5$
(8dB), increasing the cross-over frequency to somewhere near 58kHz.
Figure~\ref{fig:LTM4601A_ex1b_bode_response} shows the
Bode response with a feed-forward capacitor of
$C_{\rm FF} = 330$pF, with a cross-over frequency of 53kHz.
The increase in cross-over frequency is what improved the transient
response in Figure~\ref{fig:LTM4601A_ex1_transient_response}(b).

As commented earlier, feed-forward compensation is {\em not
recommended}. A comparison of the Bode responses in 
Figures~\ref{fig:LTM4601A_ex1a_bode_response}
and~\ref{fig:LTM4601A_ex1b_bode_response} shows that the
compensation gain at the switching frequency started out
high at 8dB, and was increased to 16dB by feed-forward
compensation. This gain amplifies the output voltage ripple
signal in the error amplifier output, which can cause
instability in the control loop.
This gain can be reduced by moving high-frequency pole
in the compensation loop down, by adding a capacitor to the
\verb+COMP+ pin. The LTM4627 $\mu$Module data sheet shows
example circuits with this additional capacitor, however,
the data sheet has no details on why it is used, i.e, the
casual user of the part has no idea what the capacitor
on the \verb+COMP+ pin is for. The data sheets refer the
user to use LTpowerCAD for control loop optimization,
however, that software currently has no such feature.

% -----------------------------------------------------------------
% Single-phase transient LTspice circuit
% -----------------------------------------------------------------
%
\begin{landscape}
\setlength{\unitlength}{1mm}
\begin{figure}[p]
  \begin{center}
    \includegraphics[width=210mm]
    {figures/LTM4601A_ex1_transient_circuit.pdf}
  \end{center}
  \caption{LTM4601A 12V to 1.5V@12A single-phase LTspice transient response analysis circuit.}
  \label{fig:LTM4601A_ex1_transient_circuit}
\end{figure}
\end{landscape}

% -----------------------------------------------------------------
% Single-phase transients for CFF = 1fF and 330pF
% -----------------------------------------------------------------
%
\setlength{\unitlength}{1mm}
\begin{figure}[p]
  \begin{picture}(155,205)(0,0)
    \put(10,105){
    \includegraphics[width=0.87\textwidth]
    {figures/LTM4601A_ex1a_transient_response.pdf}}
    \put(10,0){
    \includegraphics[width=0.87\textwidth]
    {figures/LTM4601A_ex1b_transient_response.pdf}}
    \put(77,105){(a)}
    \put(77, 0){(b)}
  \end{picture}
  \caption{LTM4601A 12V to 1.5V@12A single-phase LTspice transient response
  for feed-forward capacitances of;
  (a) $C_{\rm FF} = 1$fF (i.e., none), and (b) $C_{\rm FF} = 330$pF.
  The feed-forward capacitance improves the transient response.}
  \label{fig:LTM4601A_ex1_transient_response}
\end{figure}

% -----------------------------------------------------------------
% Single-phase Bode response LTspice circuit
% -----------------------------------------------------------------
%
\begin{landscape}
\setlength{\unitlength}{1mm}
\begin{figure}[p]
  \begin{center}
    \includegraphics[width=200mm]
    {figures/LTM4601A_ex1_bode_circuit.pdf}
  \end{center}
  \caption{LTM4601A 12V to 1.5V@12A single-phase LTspice Bode response analysis circuit.}
  \label{fig:LTM4601A_ex1_bode_circuit}
\end{figure}
\end{landscape}

% -----------------------------------------------------------------
% Single-phase Bode response for CFF = 1fF
% -----------------------------------------------------------------
%
\setlength{\unitlength}{1mm}
\begin{figure}[p]
  \begin{picture}(155,190)(0,0)
    \put(15,100){
    \includegraphics[width=0.75\textwidth]
    {figures/LTM4601A_ex1a_bode_response_mag.pdf}}
    \put(15,4){
    \includegraphics[width=0.75\textwidth]
    {figures/LTM4601A_ex1a_bode_response_phase.pdf}}
    \put(75,  97){(a)}
    \put(75,   0){(b)}
  \end{picture}
  \caption{LTM4601A 12V to 1.5V@12A single-phase supply Bode response
  ($C_{\rm FF} = 1$fF); (a) magnitude, and (b) phase. 
  The open-loop gain has a cross-over frequency of 23kHz with
  66-degrees of phase margin. The compensation network has
  a gain of 8dB at the switching frequency, i.e., the
  compensation signal contains the output voltage ripple 
  amplified by 2.5 times.}
  \label{fig:LTM4601A_ex1a_bode_response}
\end{figure}

% -----------------------------------------------------------------
% Single-phase Bode response for CFF = 330pF
% -----------------------------------------------------------------
%
\setlength{\unitlength}{1mm}
\begin{figure}[p]
  \begin{picture}(155,190)(0,0)
    \put(15,100){
    \includegraphics[width=0.75\textwidth]
    {figures/LTM4601A_ex1b_bode_response_mag.pdf}}
    \put(15,4){
    \includegraphics[width=0.75\textwidth]
    {figures/LTM4601A_ex1b_bode_response_phase.pdf}}
    \put(75,  97){(a)}
    \put(75,   0){(b)}
  \end{picture}
  \caption{LTM4601A 12V to 1.5V@12A single-phase supply Bode response
  ($C_{\rm FF} = 330$pF); (a) magnitude, and (b) phase. 
  The open-loop gain has a cross-over frequency of 53kHz with
  101-degrees of phase margin. The compensation network has
  a gain of 16dB at the switching frequency, i.e., the
  compensation signal contains the output voltage ripple 
  amplified by 6.3 times.}
  \label{fig:LTM4601A_ex1b_bode_response}
\end{figure}

\clearpage
% =================================================================
\subsubsection{LTM4601A: 12V to 1.5V@24A (dual-phase)}
% =================================================================

Figure~\ref{fig:LTM4601A_ex2_transient_circuit} shows the LTspice
circuit used for transient analysis of the 12V to 1.5V@24A dual-phase
supply. The dual-phase supply has been configured per Figure 19 on
page 23 of the data sheet~\cite{Linear_LTM4601A_2011}; 
an LTM4601A-1 has been added as the second power-stage,
an LTC6908-1 oscillator generates the 0 and 180$^\circ$
clocks, the appropriate pins on the two controllers are
tied together, the output voltage set resistor has been
adjusted, the output capacitance has been doubled, and
the load currents have been doubled.

Figure~\ref{fig:LTM4601A_ex2_transient_response} shows the
transient response. The transient response in 
Figure~\ref{fig:LTM4601A_ex2_transient_response}(a) fails to meet
the output voltage regulation of $\pm75$mV; the response dips to
1.379V (46mV violation) and peaks at 1.630V (55mV) violation.
Figure~\ref{fig:LTM4601A_ex2_transient_response}(b) shows that
the output voltage regulation requirement can almost be met
by adding a feed-forward capacitor of 680pF;  the response dips to
1.426V (1mV margin) and peaks at 1.583V (8mV) violation.

The transient responses in Figure~\ref{fig:LTM4601A_ex2_transient_response}
show an instability in the control loop. 
Figure~\ref{fig:LTM4601A_ex2_transient_currents} shows
a slightly modified transient response, where only the load
current increase is applied. The waveform
\verb|I(L1)+I(L2)-I(Rload)-I(Istep)| is the inductor ripple
current, while the waveform
\verb+I(L1)-I(L2)+ is the difference in inductor waveforms.
Both inductor waveforms should be fairly uniform with time,
with variation only visible near the load current transients.
Figure~\ref{fig:LTM4601A_ex2_transient_currents} shows that
this is not the case; after each transient, the relative phase of
the inductor waveforms changes from the nominal 180-degree
phase until the waveforms are {\em in-phase} (the
\verb+I(L1)-I(L2)+ waveform goes to zero). When the inductor
waveforms are in-phase, the output ripple voltage doubles, and 
the ripple fed back into the error amplifier also doubles.
The addition of the feed-forward compensation capacitor
{\em makes things worse} at the error amplifier output,
since the compensation network gain is increased at the
switching frequency. These issues are specific to the
LTM4601A dual-phase controller. The LTC3885 dual-phase
controller in Section~\ref{sec:LTC3855} shows none of
these issues.

Figure~\ref{fig:LTM4601A_ex2_bode_circuit} shows the LTspice
circuit used for Bode response analysis of the dual-phase supply.
Figures~\ref{fig:LTM4601A_ex2a_bode_response}
and~\ref{fig:LTM4601A_ex2b_bode_response}
show the Bode responses with $C_{\rm FF} = 1$fF and
$C_{\rm FF} = 680$pF (approximately twice 330pF)
\footnote{The Bode analysis time
was 8 hours and 6 minutes for each Bode response.}.
The comments for the single-phase controller apply to the
Bode responses for the dual-phase controller.


% -----------------------------------------------------------------
% Dual-phase transient LTspice circuit
% -----------------------------------------------------------------
%
\begin{landscape}
\setlength{\unitlength}{1mm}
\begin{figure}[p]
  \begin{center}
    \includegraphics[width=210mm]
    {figures/LTM4601A_ex2_transient_circuit.pdf}
  \end{center}
  \caption{LTM4601A 12V to 1.5V@24A dual-phase LTspice transient response analysis circuit.}
  \label{fig:LTM4601A_ex2_transient_circuit}
\end{figure}
\end{landscape}

% -----------------------------------------------------------------
% Dual-phase transients for CFF = 1fF and 330pF
% -----------------------------------------------------------------
%
\setlength{\unitlength}{1mm}
\begin{figure}[p]
  \begin{picture}(155,205)(0,0)
    \put(10,105){
    \includegraphics[width=0.87\textwidth]
    {figures/LTM4601A_ex2a_transient_response.pdf}}
    \put(10,0){
    \includegraphics[width=0.87\textwidth]
    {figures/LTM4601A_ex2b_transient_response.pdf}}
    \put(77,105){(a)}
    \put(77, 0){(b)}
  \end{picture}
  \caption{LTM4601A 12V to 1.5V@24A dual-phase LTspice transient response
  for feed-forward capacitances of;
  (a) $C_{\rm FF} = 1$fF (i.e., none), and (b) $C_{\rm FF} = 680$pF.
  The feed-forward capacitance improves the transient response.}
  \label{fig:LTM4601A_ex2_transient_response}
\end{figure}

% -----------------------------------------------------------------
% Dual-phase transient currents for CFF = 1fF and 330pF
% -----------------------------------------------------------------
%
\setlength{\unitlength}{1mm}
\begin{figure}[p]
  \begin{picture}(155,205)(0,0)
    \put(10,105){
    \includegraphics[width=0.87\textwidth]
    {figures/LTM4601A_ex2a_transient_currents.pdf}}
    \put(10,0){
    \includegraphics[width=0.87\textwidth]
    {figures/LTM4601A_ex2b_transient_currents.pdf}}
    \put(77,105){(a)}
    \put(77, 0){(b)}
  \end{picture}
  \caption{LTM4601A 12V to 1.5V@24A dual-phase LTspice transient
  response {\em instability} for feed-forward capacitances of;
  (a) $C_{\rm FF} = 1$fF (i.e., none), and (b) $C_{\rm FF} = 680$pF.
  Shortly after the transient load increase, the inductor phases 
  become {\em aligned}.}
  \label{fig:LTM4601A_ex2_transient_currents}
\end{figure}

% -----------------------------------------------------------------
% Dual-phase Bode response LTspice circuit
% -----------------------------------------------------------------
%
\begin{landscape}
\setlength{\unitlength}{1mm}
\begin{figure}[p]
  \begin{center}
    \includegraphics[width=200mm]
    {figures/LTM4601A_ex2_bode_circuit.pdf}
  \end{center}
  \caption{LTM4601A 12V to 1.5V@24A dual-phase LTspice Bode response analysis circuit.}
  \label{fig:LTM4601A_ex2_bode_circuit}
\end{figure}
\end{landscape}

% -----------------------------------------------------------------
% Dual-phase Bode response for CFF = 1fF
% -----------------------------------------------------------------
%
\setlength{\unitlength}{1mm}
\begin{figure}[p]
  \begin{picture}(155,190)(0,0)
    \put(15,100){
    \includegraphics[width=0.75\textwidth]
    {figures/LTM4601A_ex2a_bode_response_mag.pdf}}
    \put(15,4){
    \includegraphics[width=0.75\textwidth]
    {figures/LTM4601A_ex2a_bode_response_phase.pdf}}
    \put(75,  97){(a)}
    \put(75,   0){(b)}
  \end{picture}
  \caption{LTM4601A 12V to 0.95V@24A dual-phase supply Bode response
  ($C_{\rm FF} = 1$fF); (a) magnitude, and (b) phase. 
  The open-loop gain has a cross-over frequency of 23kHz with
  64-degrees of phase margin. The compensation network has
  a gain of 8dB at the switching frequency, i.e., the
  compensation signal contains the output voltage ripple 
  amplified by 2.5 times.}
  \label{fig:LTM4601A_ex2a_bode_response}
\end{figure}

% -----------------------------------------------------------------
% Dual-phase Bode response for CFF = 680pF
% -----------------------------------------------------------------
%
\setlength{\unitlength}{1mm}
\begin{figure}[p]
  \begin{picture}(155,190)(0,0)
    \put(15,100){
    \includegraphics[width=0.75\textwidth]
    {figures/LTM4601A_ex2b_bode_response_mag.pdf}}
    \put(15,4){
    \includegraphics[width=0.75\textwidth]
    {figures/LTM4601A_ex2b_bode_response_phase.pdf}}
    \put(75,  97){(a)}
    \put(75,   0){(b)}
  \end{picture}
  \caption{LTM4601A 12V to 0.95V@24A dual-phase supply Bode response
  ($C_{\rm FF} = 680$pF); (a) magnitude, and (b) phase. 
  The open-loop gain has a cross-over frequency of 55kHz with
  104-degrees of phase margin. The compensation network has
  a gain of 16dB at the switching frequency, i.e., the
  compensation signal contains the output voltage ripple 
  amplified by 6.3 times.}
  \label{fig:LTM4601A_ex2b_bode_response}
\end{figure}

\clearpage
% =================================================================
\subsubsection{LTM4601A: 12V to 0.95V@12A (single-phase)}
% =================================================================

For a $V_{\rm IN} = 12$V to $V_{\rm OUT} = 0.95\text{V} \pm 30$mV at 
$I_{\rm OUT} = 12$A single-phase power supply design, the
{\em nominal} output ripple current of the LTM4601A is
%
\begin{equation}
\Delta I_{\rm OUT} = \frac{V_{\rm OUT}}{f_{\rm SW}L}\left(
1 - \frac{V_{\rm OUT}}{V_{\rm IN}}\right) =
\frac{0.95}{850\text{k}\times0.47\mu}\left(
1 - \frac{0.95}{12}\right) = 2.2\text{A}
\end{equation}
%
where the nominal switching frequency, $f_{\rm SW} = 850$kHz, and
the internal inductor value, $L = 0.47\mu$F.

The {\em nominal} total output capacitance requirement is
%
\begin{equation}
\begin{split}
C_{\rm OUT} &> \frac{LI_{\rm STEP}^2}{2V_{\rm OUT}\Delta V_{\rm OUT}}\\
&= \frac{0.47\mu\times 11^2}{2\times0.95\times25\text{m}}\\
&= 1197\mu\text{F}
\end{split}
\end{equation}
%
%
where $I_{\rm STEP}=11$A and $\Delta V_{\rm OUT} = 25$mV was
used, since there will be additional ripple voltage due to 
the output capacitance ESR.

The {\em nominal} value of the output capacitance ESR is determined by
the allowable output voltage dip during a load step from minimum current
to maximum current, i.e.,
%
\begin{equation}
R_{\rm ESR} = \frac{\Delta V_{\rm OUT}}{I_{\rm STEP}} = 
\frac{25\text{m}}{11} = 2.3\text{m}\Omega
\end{equation}
%
The output capacitance requirements can be met using three
Sanyo POSCAP TPLF-series
330$\mu$F ETPLF330M6 (2.5V, 6m$\Omega$, 4.7A$_{\rm RMS}$)
in parallel with a 100$\mu$F ceramic.
%
The output voltage dip (peaking) for a 11A output current increase
(decrease) will be about 22mV (28mV). The ripple voltage will
be $2\text{m}\Omega \times 2.2\text{A} = 5$mV.

The transient and Bode response circuits for the 0.95V@12A supply
are similar to those shown for the 1.5V@20A supply in 
Figures~\ref{fig:LTM4601A_ex1_transient_circuit}
and~\ref{fig:LTM4601A_ex1_bode_circuit}, with the
output capacitor multiplier increased \verb+Mout1 = 3+,
and the set resistor changed to 103.5k$\Omega$
(which can be implemented via the parallel resistors
169k$\Omega$ and 267k$\Omega$).
Figure~\ref{fig:LTM4601A_ex3_transient_response} shows the
transient response, while
Figures~\ref{fig:LTM4601A_ex3a_bode_response}
and~\ref{fig:LTM4601A_ex3b_bode_response}
show the Bode responses with $C_{\rm FF} = 1$fF and
$C_{\rm FF} = 330$pF\footnote{The Bode analysis time
was 1 hour and 5 minutes for each Bode response.}.

The open-loop cross-over frequency of the nominal supply is
a very low 16kHz (less than 1/50th of the switching frequency).
The maximum gain that feed-forward compensation can implement
is $V_{\rm OUT}/V_{\rm REF} = 1.58$ (4dB), which does not
help the transient response significantly.
Using the same feed-forward capacitance value as
the 1.5V controller, $C_{\rm FF} = 330$pF, puts the
zero and pole at 8kHz and 13kHz, which are both below the
cross-over, providing maximum compensation gain.
Alas, as the transient response and Bode response shows,
this is simply not enough to improve the response of the
supply.

\clearpage
% -----------------------------------------------------------------
% Single-phase transients for CFF = 1fF and 330pF
% -----------------------------------------------------------------
%
\setlength{\unitlength}{1mm}
\begin{figure}[p]
  \begin{picture}(155,205)(0,0)
    \put(10,105){
    \includegraphics[width=0.87\textwidth]
    {figures/LTM4601A_ex3a_transient_response.pdf}}
    \put(10,0){
    \includegraphics[width=0.87\textwidth]
    {figures/LTM4601A_ex3b_transient_response.pdf}}
    \put(77,105){(a)}
    \put(77, 0){(b)}
  \end{picture}
  \caption{LTM4601A 12V to 0.95V@12A single-phase LTspice transient response
  for feed-forward capacitances of;
  (a) $C_{\rm FF} = 1$fF (i.e., none), and (b) $C_{\rm FF} = 330$pF.
  The feed-forward capacitance improves the transient response.}
  \label{fig:LTM4601A_ex3_transient_response}
\end{figure}

% -----------------------------------------------------------------
% Bode response for CFF = 1fF
% -----------------------------------------------------------------
%
\setlength{\unitlength}{1mm}
\begin{figure}[p]
  \begin{picture}(155,190)(0,0)
    \put(15,100){
    \includegraphics[width=0.75\textwidth]
    {figures/LTM4601A_ex3a_bode_response_mag.pdf}}
    \put(15,4){
    \includegraphics[width=0.75\textwidth]
    {figures/LTM4601A_ex3a_bode_response_phase.pdf}}
    \put(75,  97){(a)}
    \put(75,   0){(b)}
  \end{picture}
  \caption{LTM4601A 12V to 0.95V@12A single-phase supply Bode response
  ($C_{\rm FF} = 1$fF); (a) magnitude, and (b) phase. 
  The open-loop gain has a cross-over frequency of 16kHz with
  54-degrees of phase margin. The compensation network has
  a gain of 12dB at the switching frequency, i.e., the
  compensation signal contains the output voltage ripple 
  amplified by 4.0 times.}
  \label{fig:LTM4601A_ex3a_bode_response}
\end{figure}

% -----------------------------------------------------------------
% Bode response for CFF = 330pF
% -----------------------------------------------------------------
%
\setlength{\unitlength}{1mm}
\begin{figure}[p]
  \begin{picture}(155,190)(0,0)
    \put(15,100){
    \includegraphics[width=0.75\textwidth]
    {figures/LTM4601A_ex3b_bode_response_mag.pdf}}
    \put(15,4){
    \includegraphics[width=0.75\textwidth]
    {figures/LTM4601A_ex3b_bode_response_phase.pdf}}
    \put(75,  97){(a)}
    \put(75,   0){(b)}
  \end{picture}
  \caption{LTM4601A 12V to 0.95V@12A single-phase supply Bode response
  ($C_{\rm FF} = 330$pF); (a) magnitude, and (b) phase. 
  The open-loop gain has a cross-over frequency of 22kHz with
  75-degrees of phase margin. The compensation network has
  a gain of 16dB at the switching frequency, i.e., the
  compensation signal contains the output voltage ripple 
  amplified by 6.3 times.}
  \label{fig:LTM4601A_ex3b_bode_response}
\end{figure}

\clearpage
% =================================================================
\subsubsection{LTM4601A discussion}
% =================================================================

The Linear Technology product guide {\em Power management solutions
for Altera's FPGAs, CPLDs and Structured 
ASICs}~\cite{Linear_Altera_Product_Guide_2012}
promotes the LTM4601A for use as the core supply for Altera
Stratix IV and V series FPGAs for input voltages of 12V
and load currents of 5 to 10A. The analysis in this section
has shown that the LTM4601A controller can only meet the output
voltage regulation requirements of a 1.5V VCCIO bank of an FPGA.
The controller transient response is too slow to meet the output voltage
regulation requirements of the FPGA core supply.

So why then does Linear Technology promote the
use of this controller for powering Altera devices?
If you read the LTM4601A data sheet carefully, you will see that
the transient analysis of the controller is performed 
with respect to a load step of {\em only} 6A
(see Table 2 on page 18~\cite{Linear_LTM4601A_2011}).
Under the restriction that the LTM4601A current
handling is {\em derated} to only half the rated output current,
the transient requirements for 0.95V@6A supply should be met.
Re-running the 0.95V supply transient analysis for a constant current
of 1A, and a load step of 6A, showed that the nominal supply still 
failed, but the supply with feed-forward compensation met 
the output voltage requirements.
Unfortunately, a Stratix IV GT device with 530K logic
elements can generate a 40A load step. A multi-phase
LTM4601A controller would be pointless, as it would
require 8 phases, about ten times the PCB real-estate
as an LTC3855 (or LTC3880) dual-phase controller, and would
likely suffer terminal phasing instabilities.

