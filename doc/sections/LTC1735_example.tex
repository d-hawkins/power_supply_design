% =================================================================
\subsection{LTC1735: 12V to 3.3V@6A}
% =================================================================

Linear Technology power supply controller data sheets briefly
describe how to compensate the control loop, and then refer the
reader to Application Note AN76~\cite{Linear_AN76_1999} for more
details. AN76 discusses controller compensation with respect to 
the LTC1735, LTC1736, and LTC1628 controllers. The 
LTC1735~\cite{Linear_LTC1735_1998} is used as the 
controller in the following example design, so that the
recommendations in AN76 can be reviewed with respect to an
LTspice based design.

% -----------------------------------------------------------------
\subsubsection{Component Selection}
% -----------------------------------------------------------------

Power supply requirements;
%
\begin{itemize}
\item Input supply voltage, $V_{\rm IN} = 12$V
\item Output supply voltage, $V_{\rm OUT} = 3.3$V
\item Output supply current, $I_{\rm OUT} = 6$A
\item Output supply voltage ripple, $\Delta V_{\rm OUT} = \pm 165$mV ($\pm5$\%)
\item Controller switching frequency, $f_{\rm SW} = 500$kHz
\end{itemize}
%
The controller components are selected based on the data
sheet procedure (see p11~\cite{Linear_LTC1735_1998});
%
\begin{enumerate}
\item Select the output current-sense resistor.

\begin{equation}
R_{\rm SENSE} = \frac{50\text{mV}}{I_{\rm MAX}} =
\frac{50\text{mV}}{7\text{A}} = 7.14\text{m}\Omega \approx  7\text{m}\Omega 
\end{equation}
%
where $I_{\rm MAX} = I_{\rm OUT} + \Delta I_{\rm OUT}/2 \approx 7$A is calculated
shortly.
%
\item Select $C_{\rm OSC}$ to select the controller operating
frequency.

The LTspice model was used to determine the capacitor value that
resulted in a switching frequency of 500kHz, i.e., 
$C_{\rm OSC} = 30.1$pF.

\item Inductor selection.

The {\em nominal} inductor value is determined based on a ripple current of
30\% of the maximum output current, i.e., 
$\Delta I_{\rm OUT} = 0.3I_{\rm OUT} = 2$A. The inductor value 
is then
%
\begin{equation}
L = \frac{V_{\rm OUT}}{f_{\rm SW}\Delta I_{\rm OUT}}\left(
1 - \frac{V_{\rm OUT}}{V_{\rm IN}}\right) =
\frac{3.3}{500\text{k}\times2}\left(
1 - \frac{3.3}{12}\right) = 2.3925\mu\text{H} \approx 2.4\mu\text{H}
\end{equation}
%
A review of Vishay/Dale IHLP series inductors showed there were
no 2.4$\mu$H inductors that would support 6A of load current.
However, there were 2.2$\mu$H and 3.3$\mu$H parts,
eg., see the IHLP-2525CZ-01 and IHLP-3232DZ-01 series devices.
For $L = 2.2\mu$H, $\Delta I_{\rm OUT} = 2.2$A and 
$I_{\rm MAX} = 7.1$A, while for $L = 3.3\mu$H, 
$\Delta I_{\rm OUT} = 1.5$A and $I_{\rm MAX} = 6.75$A.

Although the larger inductor generates a smaller ripple current,
it increases the amount of output capacitance required to meet
the output voltage ripple requirement during a load transient,
so the smaller $L = 2.2\mu$H inductor is selected.
The LTC1735 does not support inductor DCR sensing, so the inductor
resistance just has to be low enough to keep the inductor
temperature increase under full load within reason.
A 6A load will generate a temperature increase of under $30^\circ$C
in either of the following Vishay/Dale devices;
IHLP-2525CZ-01 2.2$\mu$H $R_{\rm DCR(max)} = 20\text{m}\Omega$ or
IHLP-3232DZ-01 2.2$\mu$H $R_{\rm DCR(max)} = 17.7\text{m}\Omega$.

\item Power MOSFET and D1 selection.

The LTspice test jig for the LTC1735 used Vishay/Silconix
Si4410DY MOSFETs for both the top and bottom MOSFETs, and
used a 1N5818 diode over the synchronous (bottom) MOSFET.
These components were selected for use in this example.

Based on the LTC1735 data sheet equations, the power dissipated
in the synchronous MOSFET will be
%
\begin{equation}
P_{\rm SYNC} \approx \left(1 - \frac{V_{\rm OUT}}{V_{\rm IN}}\right)
I_{\rm MAX}^2R_{\rm DS(ON)} = 0.725\times7.1^2\times20\text{m} = 0.73\text{W}
\end{equation}
%
and with $\theta_{\rm JA} = 50^\circ$C/W, the junction temperature
rise will be $37^\circ$C. The power dissipated in the main (top) MOSFET
will be
%
\begin{equation}
\begin{split}
P_{\rm MAIN} &\approx \frac{V_{\rm OUT}}{V_{\rm IN}}I_{\rm MAX}^2R_{\rm DS(ON)} +
kV_{\rm IN}I_{\rm MAX}C_{\rm RSS}f_{\rm SW}\\
&= 0.275\times7.1^2\times20\text{m} + 
1.7\times12^2\times7.1\times ??? \times 500\text{k}\\
&= ???
\end{split}
\end{equation}
%
resulting in a junction temperature rise of ???

Page 26 has an example with an Si4412ADY and uses Css = 100pF. See if
I can find that data sheet. They then use an Si4410DY for the
synchronous MOSFET ... ah, so perhaps change the LTSpice simulation?

LTspice will be used to review these power dissipation estimates.

\item Input capacitance selection.

TODO: how do I calculate the value and ESR? Is it just based on
RMS handling? p21 recommends 20 to 40$\mu$F with an ESR of 10 
to 10m$\Omega$.

\item Output capacitance selection.

The {\em nominal} value of the output capacitance is determined by 
the amount required to absorb the energy in the inductor during a
load step from maximum current to minimum current, i.e.,
%
\begin{equation}
\begin{split}
C_{\rm OUT} &> \frac{LI_{\rm STEP}^2}
{(V_{\rm OUT} + \Delta V_{\rm OUT})^2 - V_{\rm OUT}^2} 
\approx \frac{LI_{\rm STEP}^2}{2V_{\rm OUT}\Delta V_{\rm OUT}}\\
&= \frac{2.2\mu\times 6^2}{2\times3.3\times0.1}\\
&= 120\mu\text{F}
\end{split}
\end{equation}
%
where $I_{\rm STEP}=6$A and $\Delta V_{\rm OUT} = 100$mV was
used, since there will be additional ripple voltage due to 
the output capacitance ESR.

The {\em nominal} value of the output capacitance ESR is determined by
the allowable output voltage dip during a load step from minimum current
to maximum current, i.e.,
%
\begin{equation}
R_{\rm ESR} = \frac{\Delta V_{\rm OUT}}{I_{\rm STEP}} = 
\frac{0.1}{6} \approx 17\text{m}\Omega
\end{equation}
%

The output capacitance requirements can be met using a single
Sanyo POSCAP TPE-series capacitor, eg., one of
150$\mu$F 6TPE150MI (6.3V, 18m$\Omega$, 2.8A$_{\rm RMS}$),
220$\mu$F 6TPE220MI (6.3V, 18m$\Omega$, 2.8A$_{\rm RMS}$), or
330$\mu$F 6TPE330MFL (6.3V, 15m$\Omega$, 3.1A$_{\rm RMS}$).

To minimize the location of the power-stage LC resonance,
the smallest value will be selected, i.e., 150$\mu$F with 
an ESR of 18m$\Omega$. The output voltage dip for a 6A output
current increase will be around 108mV, the output voltage peaking for
a 6A output current decrease will be around 67mV, and the ripple
voltage will be approximately 38mV.

\item Output voltage.

The output voltage is determined by the output feedback resistors
and the controller reference voltage, 
%
\begin{equation}
V_{\rm OUT} = V_{\rm REF}\left(1+\frac{R_{\rm a}}{R_{\rm b}}\right)
\end{equation}
%
where $V_{\rm REF} = 0.8$V, $R_{\rm a}$ is the resistor from the
output voltage to the controller $V_{\rm OSENSE}$ pin, and 
$R_{\rm b}$ is the resistor from the controller $V_{\rm OSENSE}$ 
pin to ground.

The nearest pair of 1-percent resistor values for a desired
output voltage of 3.3V are $R_{\rm a} = 3.57\text{k}\Omega$ and 
$R_{\rm b} = 1.15\text{k}\Omega$, resulting in an output
voltage of $V_{\rm OUT} = 3.28$V.

\item Soft-start capacitance.

A soft-start capacitance of $C_{\rm SS} = 800$pF was selected based
on an analysis of the LTspice transient simulation.
The simulation generates an output current load step once the 
controller has completed the soft-start sequence and the output
voltage has settled. The voltage on the soft-start capacitor 
sets the output current-limit for a short time after the
output voltage has settled. The output load step {\em should not}
be applied until after the soft-start capacitor voltage has exceeded
3.0V, otherwise the controller will enter current-limit, and
the voltage on the compensation pin will not reflect the
correct control loop dynamics.
This effect is shown shortly during the transient discussion.

\item Compensation components.

Page 21 of the LTC1735 data sheet indicates to use the compensation
components shown in Figure 1 as the starting point for the 
compensation design~\cite{Linear_LTC1735_1998}, i.e., 
$R_1  = 33\text{k}\Omega$, $C_1 = 330$pF, and
$C_2 = 100$pF. These components create a compensation
network with a zero frequency of
%
\begin{equation}
f_{\rm Z} = \frac{1}{2\pi R_1 C_1} = 14.6\text{kHz}
\end{equation}
%
and a pole frequency of
%
\begin{equation}
f_{\rm P} = \frac{1}{2\pi R_1 C_1C_2/(C_1+C_2)}
= 62.8\text{kHz}
\end{equation}
\end{enumerate}

\clearpage
% -----------------------------------------------------------------
\subsubsection{Transient Response Analysis}
% -----------------------------------------------------------------

Figure~\ref{fig:LTC1735_ex1_transient_circuit} shows an LTspice
circuit for analyzing the transient response of the
LTC1735 12V to 3.3V@6A power supply design.
Figure~\ref{fig:LTC1735_ex1_transient_response} shows the transient
response of several controller waveforms. The output voltage starts
to ramp up at $t = 0.93$ms, which is when the oscillator starts (the 
oscillator can be viewed by adding \verb+V(osc)+ to the plot).
The voltage on the soft-start capacitor sets the power-on 
current-limit, which the compensation voltage (error amplifier 
output), \verb+V(comp)+, then tracks;
in Figure~\ref{fig:LTC1735_ex1_transient_response}(a) \verb+V(comp)+
follows \verb+V(ss)-0.5+.

Figure~\ref{fig:LTC1735_ex1_transient_response} shows the transient
response for an output current load step from 1A to 6A (a 5A load step),
applied at $t = 2.0$ms and removed at $t = 2.4$ms. 
The load step is not applied until the soft-start voltage has 
risen above 3V, as otherwise the soft-start function
will limit \verb+V(comp)+  to \verb+V(ss)-0.5+, preventing the 
response at \verb+V(comp)+ being used to investigate the control-loop
dynamics. This relative timing of the soft-start and load step is only 
a concern in simulation. A hardware test of the controller would
apply the load step long after soft-start completes.

Figure~\ref{fig:LTC1735_ex1_transient_response}(b) shows a zoomed view of
the controller response to an output current load step. The compensation
response, \verb+V(comp)+, has no overshoot or ringing, indicating
the closed-loop response is stable with good phase margin.
The inductor current ripple of $\Delta I_{\rm OUT} = 2.1$A
can be seen in the inductor current waveform, \verb+I(L)+,
in the bottom trace in Figure~\ref{fig:LTC1735_ex1_transient_response}(b).
The AC component of the inductor current travels through the output
capacitance, and generates the output ripple voltage over the output
capacitance ESR.  This ripple voltage can be seen in \verb+V(out)+, in
the top trace in Figure~\ref{fig:LTC1735_ex1_transient_response}(b). 
The peak-to-peak output ripple voltage matches the expected value of 
$2.1\text{A}\times18\text{m}\Omega=38$mV.
The output voltage has an average value of 3.283V, as expected due to 
the use of 1-percent output resistors to set the output voltage. 
The output voltage dip and peaking during the application and removal
of the output current step is about -140mV and +120mV respectively.
These values are within the output voltage ripple specification,
however, due to the offset of the output voltage, the voltage dip
during the load current increase marginally violates the specification.
This issue can be rectified by using $R_{\rm a} = 10.7\text{k}\Omega$
and $R_{\rm b} = 3.4\text{k}\Omega$ for $V_{\rm OUT} = 3.318$V,
or  $R_{\rm a} = 46.4\text{k}\Omega$
and $R_{\rm b} = 14.7\text{k}\Omega$ for $V_{\rm OUT} = 3.325$V.
With this change, the power supply meets the design requirements.

The transient response of the output voltage changes depending on
the control-loop bandwidth. If the control-loop has high bandwidth,
then the output voltage dip would be dominated by the output load
step generating a voltage drop over the output capacitor ESR. 
The magnitude of the voltage dip for this example would have been
$5\text{A}\times18\text{m}\Omega=90$mV.
Since the output voltage dip of 140mV in 
Figure~\ref{fig:LTC1735_ex1_transient_response}(b) is larger than
the ESR voltage dip, the dip magnitude is being dominated by
the controller response time.
Although this dip could be reduced, since the output voltage
ripple is within the specifications, it can be accepted.


% -----------------------------------------------------------------
% LTspice circuit
% -----------------------------------------------------------------
%
\begin{landscape}
\setlength{\unitlength}{1mm}
\begin{figure}[p]
  \begin{center}
    \includegraphics[width=210mm]
    {figures/LTC1735_ex1_transient_circuit.pdf}
  \end{center}
  \caption{LTC1735 12V to 3.3V@6A LTspice transient response analysis circuit.}
  \label{fig:LTC1735_ex1_transient_circuit}
\end{figure}
\end{landscape}

% -----------------------------------------------------------------
% Transient response waveforms
% -----------------------------------------------------------------
%
\setlength{\unitlength}{1mm}
\begin{figure}[p]
  \begin{picture}(155,205)(0,0)
    \put(10,105){
    \includegraphics[width=0.87\textwidth]
    {figures/LTC1735_ex1_transient_response_a.pdf}}
    \put(10,0){
    \includegraphics[width=0.87\textwidth]
    {figures/LTC1735_ex1_transient_response_b.pdf}}
    \put(77,105){(a)}
    \put(77, 0){(b)}
  \end{picture}
  \caption{LTC1735 12V to 3.3V@6A supply LTspice transient response;
  (a) from $t=0$, and (b) zoomed view from $t=1.7$ms to 2.7ms.}
  \label{fig:LTC1735_ex1_transient_response}
\end{figure}

\clearpage
% -----------------------------------------------------------------
\subsubsection{Frequency Response Analysis}
% -----------------------------------------------------------------

This section uses an LTspice circuit to measure the power-supply
open-loop, control-to-output, compensation, and power-stage
frequency response. The frequency responses are then analyzed to
determine whether any circuit adjustments are required.

Figure~\ref{fig:LTC1735_ex1_bode_circuit} shows an LTspice
circuit for analyzing the Bode response of the
LTC1735 12V to 3.3V@6A power supply design. The {\em frequency}
response of the circuit is analyzed using multiple transient
({\em time}) analysis runs. The frequency response analysis
must be performed using multiple transient analysis runs, as
switched-mode power supplies are non-linear devices, and 
there is no linearized small-signal AC model (which is the model
needed for an LTspice \verb+.ac+ analysis). The frequency response
of a {\em physical} power supply can be measured using a
{\em frequency response analyzer} (FRA). The circuit in
Figure~\ref{fig:LTC1735_ex1_bode_circuit} uses the same
technique as an FRA; the feedback loop is broken at a 
low-impedance point, a small amplitude sine wave is injected 
into the loop, and the response at various locations around the
control loop is measured. The LTspice \verb+.measure+ statements in 
Figure~\ref{fig:LTC1735_ex1_bode_circuit} calculate the
gain and phase at each measurement frequency in the \verb+.step+
list, for the locations \verb+A+, \verb+B+, \verb+C+, and \verb+D+
in the circuit. The measurements at those locations are then 
converted into open-loop, control-to-output, compensation, and
LC response measurements.

Figure~\ref{fig:LTC1735_ex1_bode_response} shows the Bode response
for the circuit in Figure~\ref{fig:LTC1735_ex1_bode_circuit}.
The Bode response shows that the loop gain has a
cross-over frequency of 46kHz with 60-degrees of phase margin.
This is consistent with the good transient response observed in
Figure~\ref{fig:LTC1735_ex1_transient_response}. The Bode response
allows you to investigate the main control loop components;
the control-to-output (including the current-feedback loop) 
and the compensation responses.

The current-feedback loop converts the resonant LC response of 
the power-stage into a control-to-output response that is more
like a single-pole response (at least at low-frequencies). 
The control-to-output phase is a good indicator of how well the 
current-loop has been designed.
%
Figure~\ref{fig:LTC1735_ex1_bode_response}(b) shows how the close to
-180-degree phase in the LC response is reduced to under -90-degrees
in the control-to-output loop. This indicates a reasonable 
closed-loop current-feedback response.

The magnitude and phase of the control-to-output loop can be adjusted
using the current-sense resistor. A slightly smaller resistor will
cause slightly less signal to be fed back into the current-loop,
causing the control-to-output gain to increase and the control-to-output
phase to decrease. The increase in control-to-output gain can be
used to increase the cross-over frequency, however, due to the loss
of control-to-output phase, the compensation phase must be adjusted
to provide sufficient phase-margin. This adjustment in compensation
phase might not be possible due to the desire to keep the compensation
gain at the switching frequency at or below 0dB. Hence, there is a
tradeoff in the design of the current-loop and compensation
responses.

The compensation response in Figure~\ref{fig:LTC1735_ex1_bode_response}
meets the typical requirements for a compensation network; it
provides sufficient mid-band gain and phase to produce an
open-loop response with a high cross-over frequency with sufficient
phase-margin, and the gain at the switching frequency is no more than
0dB (this results in the same amount of ripple voltage on the error
amplifier output as on the power supply output).
Table~\ref{fig:LTC1735_ex1_compensation} shows the results of 
{\em designing} a compensation network using a MATLAB script.
The MATLAB script imports the control-to-output gain measured 
using LTspice, and then allows the user to adjust the compensation
response by specifying the desired cross-over frequency and
zero and pole locations. The script then calculates the nearest
1-percent resistor value and standard capacitor values.
Table~\ref{fig:LTC1735_ex1_compensation} shows the compensation
parameters for the circuit in 
Figure~\ref{fig:LTC1735_ex1_bode_response} and an alternative
set of parameters (that result in a very similar Bode response).
A transient response using the alternative parameters appeared
identical to that shown in Figure~\ref{fig:LTC1735_ex1_transient_response}.
The fact that the compensation could not be improved significantly
is a reflection that the original compensation was reasonable.

\clearpage
% -----------------------------------------------------------------
% LTspice circuit
% -----------------------------------------------------------------
%
\begin{landscape}
\setlength{\unitlength}{1mm}
\begin{figure}[p]
  \begin{center}
    \includegraphics[width=200mm]
    {figures/LTC1735_ex1_bode_circuit.pdf}
  \end{center}
  \caption{LTC1735 12V to 3.3V@6A LTspice Bode response analysis circuit.}
  \label{fig:LTC1735_ex1_bode_circuit}
\end{figure}
\end{landscape}

\clearpage
% -----------------------------------------------------------------
% Bode response
% -----------------------------------------------------------------
%
\setlength{\unitlength}{1mm}
\begin{figure}[p]
  \begin{picture}(155,190)(0,0)
    \put(15,100){
    \includegraphics[width=0.75\textwidth]
    {figures/LTC1735_ex1_bode_response_mag.pdf}}
    \put(15,4){
    \includegraphics[width=0.75\textwidth]
    {figures/LTC1735_ex1_bode_response_phase.pdf}}
    \put(75,  97){(a)}
    \put(75,   0){(b)}
  \end{picture}
  \caption{LTC1735 12V to 3.3V@6A supply Bode response;
  (a) magnitude, and (b) phase. The LC and compensation
  {\em calculated} responses are shown by the black and red
  solid lines, while the LTspice {\em simulated} responses are
  shown by the black and red crosses. The 
  control-to-output and open-loop LTspice {\em simulated} responses
  are shown using both solid lines and crosses.
  The open-loop gain has a cross-over frequency of 46kHz with
  60-degrees of phase margin.}
  \label{fig:LTC1735_ex1_bode_response}
\end{figure}

\clearpage
% -----------------------------------------------------------------
% Compensator design
% -----------------------------------------------------------------
%
\begin{table}
\caption{LTC1735 12V to 3.3V@6A supply compensation design.}
\label{fig:LTC1735_ex1_compensation}
\begin{center}
\begin{tabular}{|l|c|c|}
\hline
\rule{0cm}{4mm}Component or Parameter & \multicolumn{2}{c|}{Compensation Design Number}\\
\cline{2-3}
\rule{0cm}{4mm}     & \#1 & \#2\\
\hline
\hline
\multicolumn{3}{|l|}{\bf Compensation response}\\
\hline
& \hspace {20mm} &  \hspace {20mm} \\
$R_1$       & 33.0k$\Omega$ & 40.2k$\Omega$ \\
$C_1$       &  330pF        &  270pF        \\
$C_2$       &  100pF        &  100pF        \\
&&\\
$f_{\rm Z}$ & 14.6kHz       & 14.7kHz       \\
$f_{\rm P}$ & 62.8kHz       & 54.3kHz       \\
&&\\
Compensation gain at $f_{\rm SW}$ & 0.01dB & 0.02dB\\
&&\\
\hline
\multicolumn{3}{|l|}{\bf Open-loop response}\\
\hline
&&\\
Cross-over frequency    & 47.0kHz      & 49.8kHz      \\ 
Cross-over phase-margin & 61.0$^\circ$ & 56.8$^\circ$ \\
&&\\
\hline
\end{tabular}
\end{center}
\end{table}

\clearpage
% -----------------------------------------------------------------
\subsubsection{Review and Discussion}
% -----------------------------------------------------------------

\noindent{\bf TODO}:
\begin{itemize}
\item Review the power dissipation in the MOSFETs.
\item Can I get an efficiency report?
\item Add a comment about output loads based on current sinks rather
than resistive loads, per the comments in~\cite{Linear_DC247_1999}
\item Find ESR of ceramic caps and add a couple on the output.
\end{itemize}

