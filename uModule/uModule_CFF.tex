\documentclass[10pt,twoside]{article}

% Math symbols
\usepackage{amsmath}
\usepackage{amssymb}

% Headers/Footers
\usepackage{fancyhdr}

% Colors
\usepackage[usenames,dvipsnames]{color}

% Importing and manipulating graphics
\usepackage{graphicx}
\usepackage{subfig}
\usepackage{lscape}

% Misc packages
\usepackage{verbatim}
\usepackage{dcolumn}
\usepackage{ifpdf}
\usepackage{enumerate}

% PDF Bookmarks and hyperref stuff
\usepackage[
  bookmarks=true,
  bookmarksnumbered=true,
  colorlinks=true,
  filecolor=blue,
  linkcolor=blue,
  urlcolor=blue,
  hyperfootnotes=true
  citecolor=blue
]{hyperref}

% Improved hyperlinking to figures
% (include after hyperref)
\usepackage[all]{hypcap}

% Improved citation handling
% (include after the hyperref stuff)
\usepackage{cite}

% Pretty-print code
\usepackage{listings}

% -----------------------------------------------------------------
% Hyper-references
% -----------------------------------------------------------------
%
% This page has lots of good advice on hyper-references, including
% the use of the hypcap package and \phantomsection for generating
% labels to text.
%
% http://en.wikibooks.org/wiki/LaTeX/Labels_and_Cross-referencing
%
% -----------------------------------------------------------------
% Setup the margins
% -----------------------------------------------------------------
% Footer Template

% Set left margin - The default is 1 inch, so the following
% command sets a 1.25-inch left margin.
\setlength{\oddsidemargin}{0.25in}
\setlength{\evensidemargin}{0.25in}

% Set width of the text - What is left will be the right
% margin. In this case, right margin is
% 8.5in - 1.25in - 6in = 1.25in.
\setlength{\textwidth}{6in}

% Set top margin - The default is 1 inch, so the following
% command sets a 0.75-inch top margin.
%\setlength{\topmargin}{-0.25in}

% Set height of the header
\setlength{\headheight}{0.3in}

% Set vertical distance between the header and the text
\setlength{\headsep}{0.2in}

% Set height of the text
\setlength{\textheight}{8.5in}

% Set vertical distance between the text and the
% bottom of footer
\setlength{\footskip}{0.4in}

% -----------------------------------------------------------------
% Allow floats to take up more space on a page.
% -----------------------------------------------------------------

% see page 142 of the Companion for this stuff and the
% documentation for the fancyhdr package
\renewcommand{\textfraction}{0.05}
\renewcommand{\topfraction}{0.95}
\renewcommand{\bottomfraction}{0.95}
% dont make this too small
\renewcommand{\floatpagefraction}{0.35}
\setcounter{totalnumber}{5}

% -----------------------------------------------------------------
% Abbreviated symbols
% -----------------------------------------------------------------
\newcommand{\sinc}{\ensuremath{\,\text{sinc}}}
\newcommand{\rect}{\ensuremath{\,\text{rect}}}

% =================================================================
% The document starts here
% =================================================================
%
\begin{document}
\title{Linear Technology $\mu$Module Feed-forward Capacitor Placement}
\author{D. W. Hawkins (dwh@ovro.caltech.edu)}
\date{\today}
%\date{June 13, 2012}
\maketitle

% No header/footer on the first page
\thispagestyle{empty}

%\tableofcontents

% start the intro on an odd page
%\cleardoublepage
%\clearpage

% Set up the header/footer
\pagestyle{fancy}
\lhead{$\mu$Module Feed-forward Capacitor Placement}
\chead{}
\rhead{\today}
\lfoot{}
\cfoot{}
\rfoot{\thepage}
\renewcommand{\headrulewidth}{0.4pt}
\renewcommand{\footrulewidth}{0.4pt}

% Set the listings package language to Tcl
\lstset{language=Tcl}

% =================================================================
%\section{Discussion}
% =================================================================

The Linear Technology $\mu$Module series of integrated switch-mode
power supplies contains an incorrect recommendation for the
placement of the feed-forward capacitor in the voltage feedback
loop. Figure~\ref{fig:an76_control_loop} shows the control
loop for a current-mode controlled switch-mode power supply
(Figure 1 of Linear Technology
\href{http://cds.linear.com/docs/Application%20Note/an76.pdf}
{Application Note 76}).

Figure~\ref{fig:LTM4601_diagram} shows a block diagram of the
LTM4601/LTM4601-1 $\mu$Module (Figure 1 on page 9 of the
\href{http://cds.linear.com/docs/Datasheet/4601fc.pdf}
{LTM4601} data sheet).
Figure~\ref{fig:LTM4601_datasheet_circuits} shows two recommended
circuits from the LT4601 data sheet
(Figures 16 and 17 on page 22 of the
\href{http://cds.linear.com/docs/Datasheet/4601fc.pdf}
{LTM4601} data sheet).

Figure~\ref{fig:LTM4601_datasheet_circuits}(a) shows the
LTM4601-1; this part does not have an integrated
remote {\em differential voltage sense amplifier}.
Figure~\ref{fig:LTM4601_datasheet_circuits}(b) shows the
LTM4601; this part has an integrated remote
{\em differential voltage sense amplifier}.
The placement of the feed-forward capacitor in 
Figure~\ref{fig:LTM4601_datasheet_circuits}(a) is correct,
however, the placement of the feed-forward capacitor in 
Figure~\ref{fig:LTM4601_datasheet_circuits}(b) is
\textcolor{red}{\bf incorrect}. 
%
Figure~\ref{fig:LTM4601_equivalent_circuits_a} redraws the
circuits in Figure~\ref{fig:LTM4601_datasheet_circuits}
to help show why the placement of the feed-forward
capacitor in Figure~\ref{fig:LTM4601_datasheet_circuits}(b) is
\textcolor{red}{\bf incorrect}. 
Figure~\ref{fig:LTM4601_equivalent_circuits_b}
shows the correct placement.

The feed-forward capacitor {\em must} be placed on the
{\bf output} of the {\em differential voltage sense amplifier}
as shown in Figure~\ref{fig:LTM4601_equivalent_circuits_b},
not {\bf across} the {\em differential voltage sense amplifier}
as shown in Figure~\ref{fig:LTM4601_datasheet_circuits}(b). 
%
The placement in
Figure~\ref{fig:LTM4601_datasheet_circuits}(b)
defeats the operation of the
{\em differential voltage sense amplifier}; in a 
real-world circuit with common-mode noise on voltage sense
traces, this placement will couple common-mode noise from
the {\em differential voltage sense amplifier}
input $V_{\rm OSNS+}$, directly into the error amplifier
input $V_{\rm FB}$. This has the potential for causing
loop instability.
If the feed-forward capacitor is placed on the {\bf output}
of the {\em differential voltage sense amplifier},
as shown in Figure~\ref{fig:LTM4601_equivalent_circuits_b},
then the {\em differential voltage sense amplifier}
does its job and rejects common-mode noise, resulting 
in a clean signal at the error amplifier input.

\vskip5mm
{\bf All data sheets for Linear Technology $\mu$Modules with 
remote sensing of the output voltage need to be updated to 
show the correct placement of the feed-forward capacitor.}


\begin{figure}[p]
  \begin{center}
    \includegraphics[width=\textwidth]
    {figures/an76_control_loop.pdf}
  \end{center}
  \caption{Switch-mode power supply control loop (from AN76).
  The feed-forward capacitor $C_1$ is highlighted in red.}
  \label{fig:an76_control_loop}
\end{figure}

\begin{figure}[p]
  \begin{center}
    \includegraphics[width=\textwidth]
    {figures/ltm4601_diagram.pdf}
  \end{center}
  \caption{LTM4601/LTM4601-1 block diagram.}
  \label{fig:LTM4601_diagram}
\end{figure}

\setlength{\unitlength}{1mm}
\begin{figure}[p]
  \begin{picture}(155,205)(0,0)
    \put(10,0){
    \includegraphics[width=\textwidth]
    {figures/ltm4601_datasheet_circuits.pdf}}
    \put(40,107){(a) LTM4601-1 (no differential voltage sense amplifier).}
    \put(30,0){(b) LTM4601 with \textcolor{red}{incorrect} $C_{\rm FF}$ placement
    per the data sheet.}
  \end{picture}
  \caption{LTM4601/4601-1 data sheet circuits (Figures 16 and 17).}
  \label{fig:LTM4601_datasheet_circuits}
\end{figure}

\setlength{\unitlength}{1mm}
\begin{figure}[p]
  \begin{picture}(155,205)(0,0)
    \put(0,0){
    \includegraphics[width=\textwidth]
    {figures/ltm4601_equivalent_circuits_a.pdf}}
    \put(40,107){(a) LTM4601-1 (no differential voltage sense amplifier).}
    \put(30,0){(b) LTM4601 with \textcolor{red}{incorrect} $C_{\rm FF}$ placement
    per the data sheet.}
  \end{picture}
  \caption{LTM4601/4601-1 circuits (redrawn).}
  \label{fig:LTM4601_equivalent_circuits_a}
\end{figure}

\begin{figure}[p]
  \begin{center}
    \includegraphics[width=\textwidth]
    {figures/ltm4601_equivalent_circuits_b.pdf}
  \end{center}
  \caption{LTM4601 circuit with corrected feed-forward capacitor placement.}
  \label{fig:LTM4601_equivalent_circuits_b}
\end{figure}

\clearpage
% -----------------------------------------------------------------
% Do the bibliography
% -----------------------------------------------------------------
%Note, you can't have spaces in the list of bibliography files
%
\bibliography{refs}
\bibliographystyle{plain}

% -----------------------------------------------------------------
\end{document}











